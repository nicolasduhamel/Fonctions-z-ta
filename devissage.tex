\section{Seconde preuve du théorème \ref{thm_padique}}

On reprend les notations de la section \ref{gln}. Dans cette partie, on propose de présenter une autre preuve qui se base sur un dévissage de l'espace de Schwartz.

\subsection{Convergence de l'intégrale zêta}

Commençons par donner une seconde preuve de la convergence absolue. On va utiliser la
\begin{proposition}
Soit $f$ un coefficient de $\pi$. Alors il existe $c > 0$ et $d \in \mathbb{N}$ tel que pour tout $g \in G$, on ait $|f(g)| \leq c ||g||^d$, où $||g|| = max(|g_{ij}|, |\det g|)$.
\end{proposition}

\begin{proof}
Voir \cite[Corollaire I.4.4]{waldspurger}.
\end{proof}

On sépare l'intégrale
\begin{equation}
\int_G \phi(g)f(g)|\det g|^s dg
\end{equation}
en deux intégrales selon que $max(|g_{ij}|) < 1$ et $max(|g_{ij}|) \geq 1$. Comme $\phi$ est à support compact la deuxième intégrale converge absolument pour tout $s \in \mathbb{C}$. En ce qui concerne la première intégrale, on utilise la proposition pour montrer que
\begin{align}
\int_{max(|g_{ij}|) < 1}|\phi(g)f(g)|\det g|^s dg &\leq C \int_{GL_n(\mathbb{Z}_p)} |\det g|^{s-d} dg \\
&\leq C' \sum_{m_1 \leq ... \leq m_n} p^{-(m_1+...+m_n)(s-d)},
\end{align}
cette série géométrique converge pour $Re(s)$ assez grand.

\subsection{Équation fonctionnelle}
On veut montrer l'équation fonctionnelle suivante
\begin{equation}
\zeta(f, \phi, s) = \gamma(s) \zeta(\check{f}, \hat{\phi}, n-s),
\end{equation}
où $\gamma$ est une fonction rationnelle en $p^s$ et $\check{f}(g) = f(g^{-1})$.

Pour montrer cette équation fonctionnelle, on va utiliser la
\begin{propriete}
Les opérateurs $\zeta(., ., s)$ et $\zeta(\check{.}, \hat{.}, n-s)$ sont des opérateurs d'entrelacements, éléments de $\Hom_{G \times G} ( (\tilde{\pi} \boxtimes \pi) \otimes \mathcal{S}, |\det|_p^s \boxtimes |\det|_p^{-s})$.
\end{propriete}

On précise que l'action de $G \times G$ sur $\mathcal{S}$ est
$(g_1,g_2).\phi(x) = \phi(g_1^{-1} x g_2)$. De plus, on identifie l'ensemble des coefficients de $\pi$ avec l'espace $\tilde{V}\otimes V$; l'action de $G \times G$ sur $\tilde{\pi} \boxtimes \pi$ est $(g_1,g_2).f(g) = f(g_1^{-1} g g_2)$.

\begin{proof}
L'action de $G \times G$ sur $\zeta(f,\phi,s)$ donne
\begin{equation}
\int_G \phi(g_1^{-1}gg_2)f(g_1^{-1}gg_2)|\det g|_p^s dg.
\end{equation}
On effectue le changement de variable $g \mapsto g_1gg_2^{-1},$ le groupe $G$ étant unimodulaire l'intégrale devient
\begin{equation}
|\det g_1g_2^{-1}|^s\int_G \phi(g)f(g)|\det g|_p^s dg.
\end{equation}

D'autre part, l'action de $G \times G$ sur $\zeta(\check{f}, \hat{\phi}, n-s)$ donne
\begin{equation}
\label{intzetafourier}
\int_G \hat{\phi}_{g_1,g_2}(g)\check{f}_{g_1,g_2}(g)|\det g|_p^{n-s} dg,
\end{equation}
où l'on a noté $\phi_{g_1,g_2}(x) = \phi(g_1^{-1}xg_2)$ et $f_{g_1,g_2}(g) = f(g_1^{-1}gg_2).$

Un calcul immédiat, montre que $\check{f}_{g_1,g_2}(g) = \check{f}(g_2^{-1}gg_1)$. De plus,
\begin{equation}
\hat{\phi}_{g_1,g_2}(g) = \int_{M_n} \phi(g_1^{-1}xg_2) \psi(Tr(xg)) dx.
\end{equation}
Après le changement de variable $x \mapsto g_1xg_2^{-1}$ l'intégrale devient
\begin{equation}
|\det g_1^{-1}g_2|_p^n\int_{M_n} \phi(x) \psi(Tr(xg_2^{-1}gg_1)) dx,
\end{equation}
qui n'est autre que $|\det g_1g_2^{-1}|_p^n\hat{\phi}(g_2^{-1}gg_1)$.
L'intégrale (\ref{intzetafourier}) devient donc, après le changement de variable $g \mapsto g_2gg_1^{-1}$,
\begin{equation}|\det g_1^{-1}g_2|_p^n|\det g_2g_1^{-1}|_p^{n-s}\int_G \hat{\phi}(g)\check{f}(g)|\det g|_p^{n-s} dg.
\end{equation}
\end{proof}

Dans le but de comprendre l'espace $\Hom_{G \times G} ( (\tilde{\pi} \boxtimes \pi) \otimes \mathcal{S}, |\det|_p^s \boxtimes |\det|_p^{-s})$, on va décomposer $\mathcal{S}$ selon le rang des matrices. Soit $r$ un entier compris entier $1$ et $n$, on note $S_r$ l'espace des matrices $n \times n$ de rang $r$ et $S^{(r)}$ l'espace des matrices $n \times n$ de rang $< r$.

Si $X$ est un espace localement compact totalement discontinu, on note $C^\infty_c(X)$ l'espace des fonctions $f : X \rightarrow \mathbb{C}$ localement constantes à support compact. L'espace $\mathcal{S}$ est donc égal à $C^\infty_c(M_n)$.

Le groupe $G$ est un ouvert de $M_n$ et $M_n \setminus G = S^{(n)}$. Cette décomposition donne la suite exacte
\begin{equation}
\label{suiteexacte}
0 \rightarrow C^\infty_c(G) \rightarrow C^\infty_c(M_n) \rightarrow C^\infty_c(S^{(n)}) \rightarrow 0,
\end{equation}
où l'inclusion de $C^\infty_c(G)$ dans $C^\infty_c(M_n)$ se fait par extension par $0$ et l'application $C^\infty_c(M_n) \rightarrow C^\infty_c(S^{(n)})$ est l'application de restriction.

Cette suite exacte commute avec l'action de $G \times G$, on la voit donc comme une suite exacte de représentations de $G \times G$. On applique le foncteur $\Hom_{G \times G} (., (\pi \boxtimes \tilde{\pi}) \otimes (|\det|_p^s \boxtimes |\det|_p^{-s}))$, qui est exact à gauche, on en déduit alors l'inégalité suivante :
\begin{multline}
\dim \Hom_{G \times G} ((\tilde{\pi} \boxtimes \pi) \otimes \mathcal{S}, |.|_p^s) \leq \dim \Hom_{G \times G} ((\tilde{\pi} \boxtimes \pi) \otimes C^\infty_c(G), |.|_p^s) \\
+ \dim \Hom_{G \times G} ((\tilde{\pi} \boxtimes \pi) \otimes C^\infty_c(S^{(n)}), |.|_p^s),
\end{multline}
où l'on a abrégé $|.|_p^s = |\det|_p^s \boxtimes |\det|_p^{-s}$.

On décompose ensuite $S^{(n)}$ selon le rang $r$, ce qui donne, en utilisant le même raisonnement, que
\begin{equation}
\dim \Hom_{G \times G} ((\tilde{\pi} \boxtimes \pi) \otimes \mathcal{S}, |.|_p^s) \leq \sum_{r=0}^{n} \dim \Hom_{G \times G} ((\tilde{\pi} \boxtimes \pi) \otimes C^\infty_c(S_r), |.|_p^s).
\end{equation}

Il ne nous reste plus qu'à calculer la dimension de ces différents espaces, pour cela on dispose de la
\begin{proposition}
Pour $r=n$ $(S_r = G)$, on a
\begin{equation}
\dim \Hom_{G \times G} ( (\tilde{\pi} \boxtimes \pi) \otimes C_{c}^\infty(G), |.|_p^s) = 1;
\end{equation}
et pour $r < n$, on a
\begin{equation}
\Hom_{G \times G} ( (\tilde{\pi} \boxtimes \pi) \otimes C_{c}^\infty(S_r), |.|_p^s) = 0
\end{equation}
sauf pour un nombre fini de valeurs de $s$ modulo $\frac{2i\pi}{(n-r)\log p}\mathbb{Z}$.
\end{proposition}

\begin{proof}

Commençons par le cas $r=n$,
\begin{align}
\Hom_{G \times G}( (\tilde{\pi} \boxtimes \pi) \otimes C_c^\infty(G), |.|_p^s) &\simeq \Hom_{G \times G}( (\tilde{\pi} \boxtimes \pi) \otimes |.|_p^{-s}, C^\infty(G)) \\
&\simeq \Hom_H( (\tilde{\pi} \boxtimes \pi) \otimes |.|_p^{-s}, \mathbb{C}) \\
&\simeq \Hom_G(\tilde{\pi}, \tilde{\pi});
\end{align}
où le groupe $H$ désigne la diagonale de $G \times G$. Ce dernier espace est bien de dimension $1$ d'après le lemme de Schur.

Le premier isomorphisme provient de la dualité entre $C_c^\infty(G)$ et $C^\infty(G)$. Le deuxième isomorphisme est une application de la réciprocité de Frobenius avec l'identification $C^\infty(G) = Ind_H^{G \times G}(1)$. Pour finir, le dernier isomorphisme provient du fait que l'action diagonale de $H$ sur $\tilde{\pi} \boxtimes \pi$ correspond à l'action de $G$ sur $\tilde{\pi} \otimes \pi$ et que $|.|_p^{-s}$ est trivial sur $H$.

Passons au cas $r < n$, $S_r$ est l'orbite de $\begin{pmatrix} 
1_r & 0 \\
0 & 0 
\end{pmatrix}$ sous l'action de $G \times G$ par translation à gauche du premier facteur et translation à droite de l'inverse sur le second facteur. On calcule le stabilisateur,
\begin{equation}
H = Stab_{G \times G} \begin{pmatrix} 
1_r & 0 \\
0 & 0 
\end{pmatrix} = \left\lbrace \left(\begin{pmatrix} 
a & b \\
0 & c 
\end{pmatrix}, \begin{pmatrix} 
a & 0 \\
d & e 
\end{pmatrix} \right) \right\rbrace \subset G \times G,
\end{equation}
où $a$ décrit $GL_r(\mathbb{Q}_p)$; $c, e$ décrivent $GL_{n-r}(\mathbb{Q}_p)$; $b$ décrit $M_{r,n-r}(\mathbb{Q}_p)$ et $d$ décrit $M_{n-r,r}(\mathbb{Q}_p)$.
 
On note $P = MN$ le sous-groupe parabolique de $G$ des matrices de la forme $\begin{pmatrix} 
a & b \\
0 & c 
\end{pmatrix}$ et $\bar{P} = M\bar{N}$ le groupe parabolique opposé, alors $H \subset P \times \bar{P}$.

\begin{equation}
\label{isom}
\begin{split}
\Hom( (\tilde{\pi} \boxtimes \pi) \otimes C_c^\infty(S_r), |.|_p^s) &\simeq \Hom_{G \times G}( (\tilde{\pi} \boxtimes \pi) \otimes |.|_p^{-s}, Ind_H^{G \times G}(\delta_H)) \\
&\simeq \Hom_{M \times M}( (\tilde{\pi} \boxtimes \pi)_{N \times \bar{N}} \otimes |.|_p^{-s}, Ind_{(M \times M)\cap H}^{M \times M}(\delta_H)) \\
&\simeq \Hom_{(M \times M) \cap H}( (\tilde{\pi} \boxtimes \pi)_{N \times \bar{N}}, \delta_H \otimes |.|_p^s),
\end{split}
\end{equation}
où $\delta_H$ est le caractère modulaire de $H$.

Le premier isomorphisme provient de l'identification de $C_c^\infty(S_r)=c-Ind_H^{G \times G}(1)$ et de la dualité entre $c-Ind_H^{G \times G}(1)$ et $Ind_H^{G \times G}(\delta_H)$. Pour le deuxième isomorphisme, on utilise la transitivité de l'induction, $H \subset P \times \bar{P} \subset G \times G$, et l'adjonction entre $Ind_{P \times \bar{P}}^{G \times G}$ et le foncteur de Jacquet; en remarquant, que $N \times \bar{N}$ agit trivialement sur $|.|_p^{-s}$. Le dernier isomorphisme n'est autre que la réciprocité de Frobenius.

On utilise le fait que $(\tilde{\pi} \boxtimes \pi)_{N \times \bar{N}}$ est de longueur finie; en effet le foncteur de Jacquet préserve la longueur finie. Il existe donc des représentations admissibles $V_i$ de $M \times M$ telles que
\begin{equation}
0=V_0 \subset V_1 \subset ... \subset V_l = (\tilde{\pi} \boxtimes \pi)_{N \times \bar{N}},
\end{equation}
avec $V_i/V_{i-1}$ irréductibles.

En reprenant un raisonnement que l'on a déjà fait, la suite exacte de représentations de $M \times M$
\begin{equation}
0 \rightarrow V_{i-1} \rightarrow V_i \rightarrow V_i/V_{i-1} \rightarrow 0
\end{equation}
permet d'obtenir l'inégalité suivante :
\begin{equation}
\dim \Hom_{(M \times M)\cap H} ((\tilde{\pi} \boxtimes \pi)_{N \times \bar{N}}, |.|_p^s\delta_H) \leq \sum_{i=1}^{l} \dim \Hom_{(M \times M)\cap H} (V_i/V_{i-1}, |.|_p^s\delta_H).
\end{equation}

Il nous suffit donc de montrer que ces derniers espaces sont nuls sauf pour au plus une valeur de $s$ modulo $\frac{2i\pi}{(n-r)\log p}\mathbb{Z}$.

En tant que représentation irréductible de $M \times M \simeq GL_r^2(\mathbb{Q}_p) \times GL_{n-r}^2(\mathbb{Q}_p)$, on peut décomposer $V_i/V_{i-1} \otimes \delta_H^{-1}$ sous la forme
$\sigma^{(i)} \boxtimes (\tau_1^{(i)} \boxtimes \tau_2^{(i)})$, où $\sigma^{(i)}$ est une représentation irréductible de $GL_r^2(\mathbb{Q}_p)$ et $\tau_1^{(i)}, \tau_2^{(i)}$ sont des représentations irréductibles de $GL_{n-r}(\mathbb{Q}_p)$.

D'après le lemme de Schur, la représentation $\tau_2^{(i)}$ admet un caractère central $\omega^{(i)}$. On en déduit que
\begin{equation}
Hom_{(M \times M) \cap H} (V_i/V_{i-1}, |.|_p^s\delta_H) = 0,
\end{equation}
sauf si $\omega^{(i)} = |.|_p^{-(n-r)s}$ sur $\mathbb{Q}_p^\times$. Cette équation n'est en fait vérifiée que pour au plus une valeur de $s$ modulo $\frac{2i\pi}{(n-r)\log p}\mathbb{Z}$.
\end{proof}

Terminons la preuve de l'équation fonctionnelle. Rappelons que les opérateurs $\zeta(., ., s)$ et $\zeta(\check{.}, \hat{.}, n-s)$ sont des éléments de $\Hom_{G \times G} ( (\tilde{\pi} \boxtimes \pi) \otimes \mathcal{S}, |\det|_p^s \boxtimes |\det|_p^{-s})$, qui est de dimension $1$ sauf pour un nombre fini de valeurs de $s$ modulo $\sum_{r=0}^{n-1}\frac{2i\pi}{(n-r)\log p}\mathbb{Z}$.

Autrement dit, pour $s$ en dehors de cet ensemble de valeurs exceptionnelles, il existe $\gamma(s) \in \mathbb{C}$ tel que
\begin{equation}
\label{eqfun}
\zeta(., ., s) = \gamma(s) \zeta(\check{.}, \hat{.}, n-s).
\end{equation}

Les fonctions zêta étant des fonctions rationnelles en $p^s$ et l'ensemble des valeurs de $s$ pour lesquelles $\gamma$ est ainsi défini est dense pour la topologie de Zariski, on en déduit que l'on peut étendre $\gamma$ en une fonction rationnelle en $p^s$ pour laquelle l'équation (\ref{eqfun}) est vérifiée en tant qu'égalité de fonctions rationnelles en $p^s$.

\subsection{Familles de représentations}

On veut maintenant montrer que la fonction zêta est une fraction rationnelle où l'on peut choisir le dénominateur de façon indépendante de $f$ et de $\phi$.

Si l'on note $\pi_s = \pi \otimes |\det|^s$, alors $\zeta(f, \phi, s)$ est élément de $\Hom_{G \times G}((\tilde{\pi_s} \boxtimes \pi_s) \otimes \mathcal{S}(M), \mathbb{C})$. On va donc considérer $\pi_s$ comme une famille de représentations paramétrée par $s$.

\begin{definition}
Soit $B$ une $\mathbb{C}$-algèbre commutative noethérienne réduite. Un $(G, B)$-module (ou $B$-famille de représentation de $G$) est une représentation lisse $(\pi_B, V_B)$ de $G$ munie d'une structure de $B$-module plat, qui commute à l'action de $G$.
\end{definition}

On pose $B=\mathbb{C}[G/G^0]$ l'algèbre des polynômes sur la variété des caractères non ramifiés $X^*(G)$ de $G$. On définit alors un $(G,B)$-module $(\pi_B, V_B)$ par $V_B = V \otimes B$ et $\pi_B(g)(v \otimes b) = \pi(g)v \otimes gb$ pour $v \in V, b \in B$ et $g \in G$.

La variété des caractères non ramifiés est $X^*(G) = \left\lbrace |\det|^s, s \in \mathbb{C} \right\rbrace$, on retrouve ainsi la famille de représentation $\pi_s$.

Lorsque l'on restreint l'espace de Schwartz à $\mathcal{S}(G)$, la fonction zêta $\zeta(f, \phi, .)$ est un polynôme en $p^s$. On en déduit que $\zeta_{|\mathcal{S}(G)}$ (que l'on étend par linéarité à $B$) est un élément de $\Hom_{B, G \times G}((\tilde{\pi_B} \boxtimes \pi_B) \otimes \mathcal{S}(G), B)$.

On reprend la suite exacte (\ref{suiteexacte}) et on applique le foncteur $\Hom_{B, G \times G} (., \pi_B \boxtimes \tilde{\pi_B})$, ce qui nous donne
\begin{equation}
\begin{split}
0 \rightarrow \Hom_{B, G \times G}((\tilde{\pi_B} \boxtimes \pi_B) \otimes C^\infty_c(S^{(n)}), B) &\rightarrow \Hom_{B, G \times G}((\tilde{\pi_B} \boxtimes \pi_B) \otimes C^\infty_c(M_n), B) \\
&\rightarrow \Hom_{B, G \times G}((\tilde{\pi_B} \boxtimes \pi_B) \otimes C^\infty_c(G), B)
\end{split}
\end{equation}

Si la dernière flèche était surjective, cela montrerait immédiatement que $\zeta \in \Hom_{B, G \times G}((\tilde{\pi_B} \boxtimes \pi_B) \otimes C^\infty_c(M_n), B)$. Cependant, en général, elle n'est pas surjective. On doit donc utiliser le terme suivant de cette suite exacte, qui est $\Ext^1_{B, G \times G}((\tilde{\pi_B} \boxtimes \pi_B) \otimes C^\infty_c(S^{(n)}), B)$.

Le but est maintenant de trouver un élément de $\Hom_{B, G \times G}((\tilde{\pi_B} \boxtimes \pi_B) \otimes C^\infty_c(G), B)$ dont l'image dans $\Ext^1_{B, G \times G}((\tilde{\pi_B} \boxtimes \pi_B) \otimes C^\infty_c(S^{(n)}), B)$ est nulle, ce qui nous permettra de le relever en un élément de $\Hom_{B, G \times G}((\tilde{\pi_B} \boxtimes \pi_B) \otimes C^\infty_c(M_n), B)$. Plus exactement, on va montrer qu'il existe un élément non nul de $B$ qui agit par $0$ dans $\Ext^1_{B, G \times G}((\tilde{\pi_B} \boxtimes \pi_B) \otimes C^\infty_c(S^{(n)}), B)$.

Pour $r \in \left\lbrace 1, ..., n \right\rbrace$, on décompose l'espace $S^{(r)}$ des matrices de rang $< r$ en $S^{(r)}=S^{(r-1)} \cup S_{r-1}$ où $S_{r-1}$ est l'ouvert de $S^{(r)}$ des matrices de rang $r-1$, pour obtenir la suite exacte
\begin{equation}
0 \rightarrow C^\infty_c(S_{r-1}) \rightarrow C^\infty_c(S^{(r)}) \rightarrow C^\infty_c(S^{(r-1)}) \rightarrow 0.
\end{equation}
En prolongeant la suite exacte du foncteur $\Hom_{B, G \times G} (., \pi_B \boxtimes \tilde{\pi_B})$, on en déduit la suite exacte
\begin{equation}
\begin{split}
\Ext^1_{B, G \times G}((\tilde{\pi_B} \boxtimes \pi_B) \otimes C^\infty_c(S^{(r-1)}), B) &\rightarrow \Ext^1_{B, G \times G}((\tilde{\pi_B} \boxtimes \pi_B) \otimes C^\infty_c(S^{(r)}), B) \\ 
&\rightarrow \Ext^1_{B, G \times G}((\tilde{\pi_B} \boxtimes \pi_B) \otimes C^\infty_c(S_{r-1}), B)
\end{split}
\end{equation}

\begin{lemme}
Soit $M,N,P$ des $B$-module qui satisfont la suite exacte de $B$-modules
\begin{equation}
M \xrightarrow{\alpha} N \xrightarrow{\beta} P.
\end{equation}
Supposons qu'il existe $b_1, b_2 \in B$ tel que $b_1M = 0$ et $b_2P=0$. Alors $b_1b_2N=0$.
\end{lemme}

\begin{proof}
Soit $n \in N$. Alors $\beta(n) \in P$, donc $b_2\beta(n)=0$. On en déduit qu'il existe $m \in M$ tel que $\alpha(m)=b_2n$. Or $b_1m=0$, d'où $b_1b_2n=\alpha(b_1m)=0$.
\end{proof}

Ce lemme nous permet d'en déduire qu'il suffit de montrer que les espaces $\Ext^1_{B, G \times G}((\tilde{\pi_B} \boxtimes \pi_B) \otimes C^\infty_c(S_r), B)$ sont de torsion, pour $r \in \left\lbrace 0, ..., n-1 \right\rbrace$.

De plus, on dispose de l'isomorphisme suivant
\begin{equation}
\Ext^1_{B, G \times G}((\tilde{\pi_B} \boxtimes \pi_B) \otimes C^\infty_c(S_r), B) \simeq \Ext^1_{B, (M \times M) \cap H}((\tilde{\pi_B} \boxtimes \pi_B)_{N \times \bar{N}}, \delta_H).
\end{equation}
Pour justifier cet isomorphisme, on reprend la suite d'isomorphismes (\ref{isom}) et on montre que les isomorphismes successifs passent à $Ext^1$ en prenant une résolution projective.

Considérons l'action de $(z_1,z_2) \in Z(GL_{n-r}) \times Z(GL_{n-r})$ sur ce dernier. Il agit par $z_1^{-1}z_2 \in G/G_0$ sur $(\tilde{\pi}_B \boxtimes \pi_B)_{N \times \bar{N}}$ et par $1$ sur $\delta_H$. On en déduit que $z_1^{-1}z_2-1 \in B$ agit par $0$ dans $\Ext^1_{B, (M \times M) \cap H}((\tilde{\pi_B} \boxtimes \pi_B)_{N \times \bar{N}}, \delta_H)$. Si l'on choisi $z_1 \neq z_2 \mod G_0$, on en déduit que $\Ext^1_{B, G \times G}((\tilde{\pi_B} \boxtimes \pi_B) \otimes C^\infty_c(S_r), B)$ est de torsion.

Terminons le raisonnement. Notons $Q$ le polynôme qui correspond à l'élément de $B$ qui agit par $0$ dans $\Ext^1_{B, G \times G}((\tilde{\pi_B} \boxtimes \pi_B) \otimes C^\infty_c(S^{(n)}), B)$. Alors par $B$-linéarité, on en déduit que l'image de $Q\zeta_{|\mathcal{S}(G)}$ dans $\Ext^1_{B, G \times G}((\tilde{\pi_B} \boxtimes \pi_B) \otimes C^\infty_c(S^{(n)}), B)$ est nulle. Ce qui permet d'en déduire que $Q\zeta_{|\mathcal{S}(G)}$ se relève en un élément de $\Hom_{B, G \times G}((\tilde{\pi_B} \boxtimes \pi_B) \otimes \mathcal{S}(M_n), B)$, que l'on note $\zeta_0$. Alors $\zeta_0(s)$ est un élément de $\Hom_{G \times G} ( (\tilde{\pi} \boxtimes \pi) \otimes \mathcal{S}, |\det|_p^s \boxtimes |\det|_p^{-s})$, qui est de dimension $1$ pour presque tout $s \in \mathbb{C}$, donc il est proportionnel à $\zeta(., ., s)$. De plus, la restriction de $\zeta_0$ à $\mathcal{S}(G)$ est $Q\zeta_{|\mathcal{S}(G)}$. On en déduit que $\zeta_0(s) = Q(p^s, p^{-s})\zeta(.,.,s)$ pour presque tout $s$; c'est un polynôme en $p^s$ et $p^{-s}$. Ce qui montre que l'on peut bien choisir le dénominateur indépendant de $f$ et $\phi$; $Q$ convient.