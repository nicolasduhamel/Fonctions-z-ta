\documentclass{amsart}

\usepackage[utf8]{inputenc}
\usepackage[T1]{fontenc}
\usepackage[francais]{babel}
%\usepackage{fouriernc}
\usepackage{amssymb}
\usepackage{amsmath}
\usepackage{amsthm}
\usepackage{mathtools}
\usepackage{hyperref}
\usepackage{graphics}
\usepackage{enumerate}

\usepackage{eulervm}

\newtheorem{proposition}{Proposition}
\newtheorem{propriete}{Propriété}
\newtheorem{definition}{Définition}
\newtheorem{theoreme}{Théorème}
\newtheorem{lemme}{Lemme}

\DeclareMathOperator{\Hom}{\mathnormal{Hom}}

\begin{document}

\title{Équation fonctionnelle}
\maketitle

%\bibliographystyle{siam}
%\bibliography{biblio}

Soit $n \geq 1$ un entier et $p$ un nombre premier. Dans la suite, on notera $G = GL_n(\mathbb{Q}_p)$, $dg$ une mesure de Haar sur $G$ et $(\pi, V)$ une représentation admissible irréductible de $G$.

Les coefficients de $\pi$ sont les fonctions de la forme $g \in G \mapsto <\pi(g)v, \tilde{v}>$, où $v \in V$ et $\tilde{v} \in \tilde{V}$.

On note $M_n$ l'ensemble des matrices $n \times n$ à coefficients dans $\mathbb{Q}_p$ et $\mathcal{S}$ l'ensemble des fonctions $\phi : M_n \rightarrow \mathbb{C}$ localement constantes à support compact.

Si $f$ est un coefficient de $\pi$, $\phi \in \mathcal{S}$ et $s \in \mathbb{C}$, on pose
\begin{equation}
\zeta(f, \phi, s) = \int_G \phi(g)f(g)|\det g|_p^s dg.
\end{equation}

On fixe un caractère $\psi$ de $\mathbb{Q}_p^\times$ et on pose
\begin{equation}
\hat{\phi}(y) = \int_{M_n} \phi(x) \psi(Tr(xy)) dx,
\end{equation}
où $dx$ est une mesure de Haar sur $M_n$.

On veut montrer l'équation fonctionnelle suivante
\begin{equation}
\zeta(f, \phi, s) = \gamma(s) \zeta(\check{f}, \hat{\phi}, 1-s),
\end{equation}
où $\gamma$ est une fonction rationnelle en $p^s$ et $\check{f}(g) = f(g^{-1})$.

Pour montrer cette équation fonctionnelle, on va utiliser la
\begin{propriete}
Les opérateurs $\zeta(., ., s)$ et $\zeta(\check{.}, \hat{.}, 1-s)$ sont des opérateurs d'entrelacements, éléments de $\Hom_{G \times G} ( (\pi \boxtimes \tilde{\pi}) \otimes \mathcal{S}, |\det|_p^s \boxtimes |\det|_p^{-s})$.
\end{propriete}

On précise que l'action de $G \times G$ sur $\mathcal{S}$ est
$(g_1,g_2).\phi(x) = \phi(g_1^{-1} x g_2)$. De plus, on identifie l'ensemble des coefficients de $\pi$ avec l'espace $V\otimes \tilde{V}$; l'action de $G \times G$ sur $\pi \boxtimes \tilde{\pi}$ est $(g_1,g_2).f(g) = f(g_1^{-1} g g_2)$.

\begin{proof}
L'action de $G \times G$ sur $\zeta(f,\phi,s)$ donne
\begin{equation}
\int_G \phi(g_1^{-1}gg_2)f(g_1^{-1}gg_2)|\det g|_p^s dg.
\end{equation}
On effectue le changement de variable $g \mapsto g_1gg_2^{-1},$ le groupe $G$ étant unimodulaire l'intégrale devient
\begin{equation}
|\det g_1g_2^{-1}|^s\int_G \phi(g)f(g)|\det g|_p^s dg.
\end{equation}

D'autre part, l'action de $G \times G$ sur $\zeta(\check{f}, \hat{\phi}, 1-s)$ donne
\begin{equation}
\label{intzetafourier}
\int_G \hat{\phi_{g_1,g_2}}(g)\check{f_{g_1,g_2}}(g)|\det g|_p^{1-s} dg,
\end{equation}
où l'on a noté $\phi_{g_1,g_2}(x) = \phi(g_1^{-1}xg_2)$ et $f_{g_1,g_2}(g) = f(g_1^{-1}gg_2).$

Un calcul immédiat, montre que $\check{f_{g_1,g_2}}(g) = \check{f}(g_2^{-1}gg_1)$. De plus,
\begin{equation}
\hat{\phi_{g_1,g_2}}(g) = \int_{M_n} \phi(g_1^{-1}xg_2) \psi(Tr(xg)) dx.
\end{equation}
Après le changement de variable $x \mapsto g_1xg_2^{-1}$ l'intégrale devient
\begin{equation}
|\det g_1^{-1}g_2|_p\int_{M_n} \phi(x) \psi(Tr(xg_2^{-1}gg_1)) dx,
\end{equation}
qui n'est autre que $|\det g_1g_2^{-1}|\hat{\phi}(g_2^{-1}gg_1)$.
L'intégrale (\ref{intzetafourier}) devient donc, après le changement de variable $g \mapsto g_2gg_1^{-1}$,
\begin{equation}|\det g_1^{-1}g_2|_p|\det g_2g_1^{-1}|_p^{1-s}\int_G \hat{\phi}(g)\check{f}(g)|\det g|_p^{1-s} dg.
\end{equation}
\end{proof}

Dans le but de comprendre l'espace $\Hom_{G \times G} ( (\pi \boxtimes \tilde{\pi}) \otimes \mathcal{S}, |\det|_p^s \boxtimes |\det|_p^{-s})$, on va décomposer $\mathcal{S}$ selon le rang des matrices. Soit $r$ un entier compris entier $1$ et $n$, on note $S_r$ l'espace des matrices $n \times n$ de rang $r$ et $S^{(r)}$ l'espace des matrices $n \times n$ de rang $< r$.

Si $X$ est un espace localement compact totalement discontinu, on note $C^\infty_c(X)$ l'espace des fonctions $f : X \rightarrow \mathbb{C}$ localement constantes à support compact. L'espace $\mathcal{S}$ est donc égal à $C^\infty_c(M_n)$.

Le groupe $G$ est un ouvert de $M_n$ et $M_n \setminus G = S^{(n)}$. Cette décomposition donne la suite exacte
\begin{equation}
0 \rightarrow C^\infty_c(G) \rightarrow C^\infty_c(M_n) \rightarrow C^\infty_c(S^{(n)}) \rightarrow 0,
\end{equation}
où l'inclusion de $C^\infty_c(G)$ dans $C^\infty_c(M_n)$ se fait par extension par $0$ et l'application $C^\infty_c(M_n) \rightarrow C^\infty_c(S^{(n)})$ est l'application de restriction.

Cette suite exacte commute avec l'action de $G \times G$, on la voit donc comme une suite exacte de représentations de $G \times G$. On applique le foncteur $\Hom_{G \times G} (., (\pi \boxtimes \tilde{\pi}) \otimes (|\det|_p^s \boxtimes |\det|_p^{-s}))$, qui est exact à gauche, on en déduit alors l'inégalité suivante :
\begin{multline}
\dim \Hom_{G \times G} ((\pi \boxtimes \tilde{\pi}) \otimes \mathcal{S}, |.|_p^s) \leq \dim \Hom_{G \times G} ((\pi \boxtimes \tilde{\pi}) \otimes C^\infty_c(G), |.|_p^s) \\
+ \dim \Hom_{G \times G} ((\pi \boxtimes \tilde{\pi}) \otimes C^\infty_c(S^{(n)}), |.|_p^s),
\end{multline}
où l'on a abrégé $|.|_p^s = |\det|_p^s \boxtimes |\det|_p^{-s}$.

On décompose ensuite $S^{(n)}$ selon le rang $r$, ce qui donne, en utilisant le même raisonnement, que
\begin{equation}
\dim \Hom_{G \times G} ((\pi \boxtimes \tilde{\pi}) \otimes \mathcal{S}, |.|_p^s) \leq \sum_{r=0}^{n} \dim \Hom_{G \times G} ((\pi \boxtimes \tilde{\pi}) \otimes C^\infty_c(S_r), |.|_p^s).
\end{equation}

Il ne nous reste plus qu'à calculer la dimension de ces différents espaces, pour cela on dispose de la
\begin{proposition}
Pour $r=n$ $(S_r = G)$, on a
\begin{equation}
\dim \Hom_{G \times G} ( (\pi \boxtimes \tilde{\pi}) \otimes C_{c}^\infty(G), |.|_p^s) = 1;
\end{equation}
et pour $r < n$, on a
\begin{equation}
\Hom_{G \times G} ( (\pi \boxtimes \tilde{\pi}) \otimes C_{c}^\infty(S_r), |.|_p^s) = 0
\end{equation}
sauf pour un nombre fini de valeurs de $s$ modulo $\frac{2i\pi}{(n-r)\log p}\mathbb{Z}$.
\end{proposition}

\begin{proof}

Commençons par le cas $r=n$,
\begin{align}
\Hom_{G \times G}( (\pi \boxtimes \tilde{\pi}) \otimes C_c^\infty(G), |.|_p^s) &\simeq \Hom_{G \times G}( (\pi \boxtimes \tilde{\pi}) \otimes |.|_p^{-s}, C^\infty(G)) \\
&\simeq \Hom_H( (\pi \boxtimes \tilde{\pi}) \otimes |.|_p^{-s}, \mathbb{C}) \\
&\simeq \Hom_G(\pi, \pi);
\end{align}
où le groupe $H$ désigne la diagonale de $G \times G$. Ce dernier espace est bien de dimension $1$ d'après le lemme de Schur.

Le premier isomorphisme provient de la dualité entre $C_c^\infty(G)$ et $C^\infty(G)$. Le deuxième isomorphisme est une application de la réciprocité de Frobenius avec l'identification $C^\infty(G) = Ind_H^{G \times G}(1)$. Pour finir, le dernier isomorphisme provient du fait que l'action diagonale de $H$ sur $\pi \boxtimes \tilde{\pi}$ correspond à l'action de $G$ sur $\pi \otimes \tilde{\pi}$ et que $|.|_p^{-s}$ est trivial sur $H$.

Passons au cas $r < n$, $S_r$ est l'orbite de $\begin{pmatrix} 
1_r & 0 \\
0 & 0 
\end{pmatrix}$ sous l'action de $G \times G$ par translation à gauche du premier facteur et translation à droite de l'inverse sur le second facteur. On calcule le stabilisateur,
\begin{equation}
H = Stab_{G \times G} \begin{pmatrix} 
1_r & 0 \\
0 & 0 
\end{pmatrix} = \left\lbrace \left(\begin{pmatrix} 
a & b \\
0 & c 
\end{pmatrix}, \begin{pmatrix} 
a & 0 \\
d & e 
\end{pmatrix} \right) \right\rbrace \subset G \times G,
\end{equation}
où $a$ décrit $GL_r(\mathbb{Q}_p)$; $c, e$ décrivent $GL_{n-r}(\mathbb{Q}_p)$; $b$ décrit $M_{r,n-r}(\mathbb{Q}_p)$ et $d$ décrit $M_{n-r,r}(\mathbb{Q}_p)$.
 
On note $P = MN$ le sous-groupe parabolique de $G$ des matrices de la forme $\begin{pmatrix} 
a & b \\
0 & c 
\end{pmatrix}$ et $\bar{P} = M\bar{N}$ le groupe parabolique opposé, alors $H \subset P \times \bar{P}$.

\begin{align}
\Hom( (\pi \boxtimes \tilde{\pi}) \otimes C_c^\infty(S_r), |.|_p^s) &\simeq \Hom_{G \times G}( (\pi \boxtimes \tilde{\pi}) \otimes |.|_p^{-s}, Ind_H^{G \times G}(\delta_H)) \\
&\simeq \Hom_{M \times M}( (\pi \boxtimes \tilde{\pi})_{N \times \bar{N}} \otimes |.|_p^{-s}, Ind_{(M \times M)\cap H}^{M \times M}(\delta_H)) \\
&\simeq \Hom_{(M \times M) \cap H}( (\pi \boxtimes \tilde{\pi})_{N \times \bar{N}}, \delta_H \otimes |.|_p^s),
\end{align}
où $\delta_H$ est le caractère modulaire de $H$.

Le premier isomorphisme provient de l'identification de $C_c^\infty(S_r)=c-Ind_H^{G \times G}(1)$ et de la dualité entre $c-Ind_H^{G \times G}(1)$ et $Ind_H^{G \times G}(\delta_H)$. Pour le deuxième isomorphisme, on utilise la transitivité de l'induction, $H \subset P \times \bar{P} \subset G \times G$, et l'adjonction entre $Ind_{P \times \bar{P}}^{G \times G}$ et le foncteur de Jacquet; en remarquant, que $N \times \bar{N}$ agit trivialement sur $|.|_p^{-s}$. Le dernier isomorphisme n'est autre que la réciprocité de Frobenius.

On utilise le fait que $(\pi \boxtimes \tilde{\pi})_{N \times \bar{N}}$ est de longueur finie; en effet le foncteur de Jacquet préserve la longueur finie. Il existe donc des représentations admissibles $V_i$ de $M \times M$ telles que
\begin{equation}
0=V_0 \subset V_1 \subset ... \subset V_l = (\pi \boxtimes \tilde{\pi})_{N \times \bar{N}},
\end{equation}
avec $V_i/V_{i-1}$ irréductibles.

En reprenant un raisonnement que l'on a déjà fait, la suite exacte de représentations de $M \times M$
\begin{equation}
0 \rightarrow V_{i-1} \rightarrow V_i \rightarrow V_i/V_{i-1} \rightarrow 0
\end{equation}
permet d'obtenir l'inégalité suivante :
\begin{equation}
\dim \Hom_{(M \times M)\cap H} ((\pi \boxtimes \tilde{\pi})_{N \times \bar{N}}, |.|_p^s\delta_H) \leq \sum_{i=1}^{l} \dim \Hom_{(M \times M)\cap H} (V_i/V_{i-1}, |.|_p^s\delta_H).
\end{equation}

Il nous suffit donc de montrer que ces derniers espaces sont nuls sauf pour au plus une valeur de $s$ modulo $\frac{2i\pi}{(n-r)\log p}\mathbb{Z}$.

En tant que représentation irréductible de $M \times M \simeq GL_r^2(\mathbb{Q}_p) \times GL_{n-r}^2(\mathbb{Q}_p)$, on peut décomposer $V_i/V_{i-1} \otimes \delta_H^{-1}$ sous la forme
$\sigma^{(i)} \boxtimes (\tau_1^{(i)} \boxtimes \tau_2^{(i)})$, où $\sigma^{(i)}$ est une représentation irréductible de $GL_r^2(\mathbb{Q}_p)$ et $\tau_1^{(i)}, \tau_2^{(i)}$ sont des représentations irréductibles de $GL_{n-r}(\mathbb{Q}_p)$.

D'après le lemme de Schur, la représentation $\tau_2^{(i)}$ admet un caractère central $\omega^{(i)}$. On en déduit que
\begin{equation}
Hom_{(M \times M) \cap H} (V_i/V_{i-1}, |.|_p^s\delta_H) = 0,
\end{equation}
sauf si $\omega^{(i)} = |.|_p^{-(n-r)s}$ sur $\mathbb{Q}_p^\times$. Cette dernière équation ne peut être vérifiée que pour au plus une valeur de $s$ modulo $\frac{2i\pi}{(n-r)\log p}\mathbb{Z}$.
\end{proof}

Terminons la preuve de l'équation fonctionnelle. Rappelons que les opérateurs $\zeta(., ., s)$ et $\zeta(\check{.}, \hat{.}, 1-s)$ sont des éléments de $\Hom_{G \times G} ( (\pi \boxtimes \tilde{\pi}) \otimes \mathcal{S}, |\det|_p^s \boxtimes |\det|_p^{-s})$, qui est de dimension $1$ sauf pour un nombre fini de valeurs de $s$ modulo $\sum_{r=0}^{n-1}\frac{2i\pi}{(n-r)\log p}\mathbb{Z}$.

Autrement dit, pour $s$ en dehors de cet ensemble de valeurs exceptionnelles, il existe $\gamma(s) \in \mathbb{C}$ tel que
\begin{equation}
\label{eqfun}
\zeta(., ., s) = \gamma(s) \zeta(\check{.}, \hat{.}, 1-s).
\end{equation}

Les fonctions zêta étant des fonctions rationnelles en $p^s$ et l'ensemble des valeurs de $s$ pour lesquelles $\gamma$ est ainsi défini est dense pour la topologie de Zariski, on en déduit que l'on peut étendre $\gamma$ en une fonction rationnelle en $p^s$ pour laquelle l'équation (\ref{eqfun}) est vérifiée comme égalité de fonctions rationnelles en $p^s$. 
\end{document}