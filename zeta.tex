\documentclass{amsart}

\usepackage[utf8]{inputenc}
\usepackage[T1]{fontenc}
\usepackage[francais]{babel}
\usepackage{bbm}
\usepackage{amssymb}
\usepackage{amsmath}
\usepackage{amsthm}
\usepackage{mathtools}
\usepackage{hyperref}
\usepackage{graphics}
\usepackage{enumerate}

\usepackage{eulervm}

\newtheorem{proposition}{Proposition}
\newtheorem{propriete}{Propriété}
\newtheorem{definition}{Définition}
\newtheorem{theoreme}{Théorème}
\newtheorem{lemme}{Lemme}

\DeclareMathOperator{\Hom}{\mathnormal{Hom}}
\DeclareMathOperator{\Ext}{\mathnormal{Ext}}

\begin{document}

\title{Fonctions zêta sur $GL_n$}
\maketitle

Ce mémoire est consacré à la théorie de Godement-Jacquet \cite{godement-jacquet} des fonctions zêtas, qui est une généralisation des résultats de Tate \cite{tate} sur $GL_1$. Cette théorie montre que l'on peut définir la fonction $L$ attachée à un représentation de $GL_n$ comme le dénominateur communs de fonctions zêta dont on montrera quelles sont des fractions rationnelles.

Dans la section 1, on rappelle l'essentiel des résultats sur $GL_1$, les méthodes de cette section seront ensuite généralisées à $GL_n$. Dans la section 2, on présente la théorie locale de Godement-Jacquet; on se ramène au cas d'une représentation supercuspidale et on utilise le fait que les coefficients sont à support compact modulo le centre dans ce cas. Dans la section 3, on présente une seconde preuve du théorème \ref{thm_padique}, la preuve consiste en un dévissage de l'espace de Schwartz, suivant une suggestion de Raphaël Beuzart-Plessis. Dans la section 4, on explique les modifications nécessaires dans le cas archimédien. Pour finir, la section 5 est consacré à la théorie globale de Godement-Jacquet avec la définition de la fonction $L$ attachée à une représentation cuspidale et les premières propriétés de ces fonctions $L$ (holomorphie, équation fonctionnelle).

Ce mémoire a été effectué sous la direction de Raphaël Beuzart-Plessis.

\tableofcontents

\section{Fonctions zêta sur $GL_1(\mathbb{Q}_p)$}

Cette section décrit la théorie de Tate \cite{tate} des fonctions zêta sur $GL_1(\mathbb{Q}_p)$. Dans la section \ref{gln}, on aura besoin de supposer que $n > 1$ et on aura aussi besoin des résultats pour $n=1$.

\subsection{Caractères de $\mathbb{Q}_p$}

Commençons par décrire l'ensemble des caractères de $\mathbb{Q}_p$. Pour ce faire, on dispose du

\begin{lemme}
Soit $\psi$ un caractère non trivial de $\mathbb{Q}_p$, alors les caractères de $\mathbb{Q}_p$ sont de la forme $x \mapsto \psi(xy)$ avec $y \in \mathbb{Q}_p$.
\end{lemme}

Pour avoir une description complète des caractères de $\mathbb{Q}_p$, il ne nous reste plus qu'à exhiber un caractère non trivial.

Soit $x \in \mathbb{Q}_p$, alors il existe $\mu \in \mathbb{N}$ tel que $p^\mu x \in \mathbb{Z}_p$. On note $m$ un entier tel que $m = p^\mu x \mod p^\mu$. On pose alors $\lambda(x) = \frac{m}{p^\mu}$, c'est un rationnel bien déterminé $\mod 1$. L'application $x \mapsto e^{2i\pi \lambda(x)}$ est un caractère non trivial de $\mathbb{Q}_p$.

On note $\mathcal{S}(\mathbb{Q}_p)$ l'ensemble des fonctions $\phi : \mathbb{Q}_p \rightarrow \mathbb{C}$ localement constantes à support compact, on l'appelle l'espace de Schwartz de $\mathbb{Q}_p$.

Pour $\phi \in \mathcal{S}(\mathbb{Q}_p)$, on définit la transformée de Fourier de $\phi$ par la formule
\begin{equation}
\hat{\phi}(y) = \int_{\mathbb{Q}_p} \phi(x) e^{-2i\pi \lambda(xy)}dx, y \in \mathbb{Q}_p,
\end{equation}
où $dx$ est une mesure de Haar sur $\mathbb{Q}_p$. On choisit la mesure de Haar de façon à avoir la formule d'inversion $\hat{\hat{\phi}}(x)=\phi(-x)$.

\subsection{Fonctions zêta, équation fonctionnelle}
\begin{definition}
Pour $\phi \in \mathcal{S}(\mathbb{Q}_p), \omega$ caractère de $\mathbb{Q}_p^\times$ et $s \in \mathbb{C}$, on pose
\begin{equation}
\zeta(\omega,\phi,s) = \int_{\mathbb{Q}_p^\times} \phi(x) \omega(x) |x|_p^s d^\times x,
\end{equation}
où $d^\times x$ est une mesure de Haar sur $\mathbb{Q}_p$. On choisit la mesure de Haar normalisé de telle façon à avoir $vol(\mathbb{Z}_p^\times)=1$.
\end{definition}

\begin{lemme}
\label{convergence-gl1}
L'intégrale définissant la fonction zêta est absolument convergente pour $Re(s)$ assez grand.
\end{lemme}

\begin{proof}
Quitte à remplacer $\omega$ par $\omega|.|_p^{-s_0}$, on peut supposer que $\omega$ est unitaire. On sépare l'intégrale en deux selon un intégrale pour $|x|_p > 1$ et l'autre pour $|x|_p \leq 1$. 

Pour ce qui est de la première intégrale, elle est absolument convergente car $\phi$ est à support compact. Quant à la seconde intégrale, elle est bornée par
\begin{equation}
\int_{|x|_p \leq 1} |x|_p^{Re(s)}dx=\sum_{k=0}^{\infty} p^{-kRe(s)},
\end{equation}
à une constante près, car $\phi$ est bornée. Cette dernière intégrale est convergente pour $Re(s) > 0$.
\end{proof}

Dans la suite, on veut montrer que les fonctions zêta peuvent être prolongées analytiquement en des fonctions méromorphes. On montre aussi que les fonctions zêta vérifient une équation fonctionnelle. On commence par le
\begin{lemme}
\label{lemme_fun}
Soient $\phi_1, \phi_2 \in \mathcal{S}(\mathbb{Q}_p), \omega$ un caractère de $\mathbb{Q}_p^\times$ et $s \in \mathbb{C}$. Alors
\begin{equation}
\zeta(\omega, \phi_1, s)\zeta(\omega^{-1}, \hat{\phi}_2, 1-s)=\zeta(\omega^{-1}, \hat{\phi}_1, 1-s)\zeta(\omega, \phi_2, s),
\end{equation}
cette équation étant valable dans le domaine d'absolue convergence.
\end{lemme}

\begin{proof}
Développons le membre de gauche,
\begin{align}
\zeta(\omega, \phi_1, s)\zeta(\omega^{-1}, \hat{\phi}_2, 1-s) &= \int_{{(\mathbb{Q}_p^\times)}^2} \phi_1(x_1)\hat{\phi}_2(x_2)\omega(x_1x_2^{-1})|x_1x_2^{-1}|_p^s|x_2|_p d^\times x_1 d^\times x_2 \\
&= \int_{{(\mathbb{Q}_p^\times)}^2} \phi_1(x_1)\hat{\phi}_2(x_1x_2)\omega(x_2^{-1})|x_2|_p^{-s}|x_1x_2|_p d^\times x_1 d^\times x_2.
\end{align}
On ne considère maintenant que la partie qui dépend de la variable $x_1$ et on utilise la formule définissant la transformée de Fourier, on obtient
\begin{align}
\int_{\mathbb{Q}_p^\times} \phi_1(x_1)\hat{\phi}_2(x_1x_2)|x_1|_p d^\times x_1 &=\int_{\mathbb{Q}_p^\times} \int_{\mathbb{Q}_p} \phi_1(x_1) \phi_2(y)|x_1|_p e^{-2i\pi \lambda(x_1x_2 y)} dy d^\times x_1 \\
&= \frac{p}{p-1}\int_{{(\mathbb{Q}_p)}^2} \phi_1(x_1)\phi_2(y) e^{-2i\pi \lambda(x_1x_2 y)} dy dx_1 \\
&= \int_{\mathbb{Q}_p^\times} \hat{\phi}_1(x_1x_2)\phi_2(x_1)|x_1|_p d^\times x_1.
\end{align}

D'autre part, en développant le membre de droite de l'équation, on a
\begin{equation}
\zeta(\omega^{-1}, \hat{\phi}_1, 1-s)\zeta(\omega, \phi_2, s) =  \int_{{(\mathbb{Q}_p^\times)}^2} \hat{\phi}_1(x_1x_2)\phi_2(x_1)|x_1|_p \omega(x_2^{-1})|x_2|_p^{1-s} d^\times x_1 d^\times x_2.
\end{equation}

On en déduit l'égalité $\zeta(\omega, \phi_1, s)\zeta(\omega^{-1}, \hat{\phi}_2, 1-s) = \zeta(\omega^{-1}, \hat{\phi}_1, 1-s)\zeta(\omega, \phi_2, s)$.
\end{proof}

On démontre maintenant le
\begin{theoreme}
Soit $\phi \in \mathcal{S}(\mathbb{Q}_p), \omega$ un caractère de $\mathbb{Q}_p^\times$ et $s \in \mathbb{C}$. La fonction $\zeta(\phi, \omega, .)$ peut être prolongée analytiquement en une fonction méromorphe sur $\mathbb{C}$. De plus, elle vérifie
\begin{equation}
\label{eq_gl1}
\zeta(\omega^{-1}, \hat{\phi},  1-s)=\gamma(\omega, s)\zeta(\omega,\phi, s),
\end{equation}
où $\gamma$ est une fonction méromorphe indépendante de $\phi$.
\end{theoreme}

\begin{proof}
On suppose tout d'abord que l'équation (\ref{eq_gl1}) est vérifiée pour une fonction $\phi_0 \in \mathcal{S}(\mathbb{Q}_p)$ telle que le quotient $\frac{\zeta(\omega^{-1}, \hat{\phi}_0 1-s)}{\zeta(\omega, \phi_0, s)}$ est défini. D'après le lemme (\ref{lemme_fun}), on en déduit que
\begin{equation}
\zeta(\omega^{-1}, \hat{\phi}, 1-s) = \gamma(\omega,s)\zeta(\omega, \phi, s),
\end{equation}
où $\gamma(\omega,s) = \frac{\zeta(\omega^{-1}, \hat{\phi}_0, 1-s)}{\zeta(\omega, \phi_0, s)}$. Ce qui montre l'équation fonctionnelle pour toutes les fonctions $\phi \in \mathcal{S}(\mathbb{Q}_p)$. La suite de cette section, on montre que cette fonction $\phi_0$ existe bien et on calcule le facteur $\gamma$.
\end{proof}

\subsection{Calcul du facteur $\gamma$}

Le caractère $\omega$ est trivial sur un ensemble de la forme $1+p^m \mathbb{Z}_p$, on suppose $m$ minimal pour cette propriété. L'entier $p^m$ est appelé le conducteur de $\omega$.

On choisit alors pour $\phi_0$ la fonction telle que $\phi_0(x)=e^{2i\pi \lambda(x)}$ si $x \in p^{-m}\mathbb{Z}_p$ et $\phi_0(x)=0$ sinon.

\begin{lemme}
\begin{equation}
\hat{\phi}_0(x) = \left\{
    \begin{array}{ll}
        p^m & \mbox{si } x = 1 \mod p^m, \\
        0 & \mbox{sinon.}
    \end{array}
\right.
\end{equation}
\end{lemme}

Dans le cas où le caractère $\omega$ est non ramifié, $m=0$,
\begin{align}
\zeta(\omega, \phi_0, s)&=\frac{1}{1-\omega(p)p^{-s}},\\
\zeta(\omega^{-1}, \hat{\phi}_0, 1-s) &= \frac{1}{1-\omega(p)^{-1}p^{s-1}}, \\
\gamma(\omega,s) &= \frac{1-\omega(p)p^{-s}}{1-\omega(p)^{-1}p^{s-1}}.
\end{align}

Dans le cas ramifié, $m > 0$,
\begin{align}
\zeta(\omega, \phi_0, s) &= p^{ms}\int_{p^{-m}\mathbb{Z}_p^\times} \omega(x)e^{2i\pi \lambda(x)}d^\times x, \\
\zeta(\omega^{-1}, \hat{\phi}_0, s) &= p^m\int_{1+p^{m}\mathbb{Z}_p} d^\times x, \\
\gamma(\omega,s) &= c p^{m(1-s)},
\end{align}
où $c$ est une constante non nulle.

Pour finir cette partie sur $GL_1(\mathbb{Q}_p)$, on définit la fonction $L$ d'un caractère $\omega$.
\begin{definition}
Lorsque $\omega$ est non ramifié, on pose
\begin{equation}
L(\omega,s) = \frac{1}{1-\omega(p)p^{-s}}.
\end{equation}
Si $\omega$ est ramifié, on pose $L(\omega,s)=1$.
\end{definition}

La proposition suivante nous sera utile dans la suite.
\begin{proposition}
Le quotient $\frac{\zeta(\omega, \phi, s)}{L(\omega,s)}$ est un polynôme en $p^s$ et $p^{-s}$.
\end{proposition}

\begin{proof}
On écrit $\phi = \phi_1 + \alpha \phi_0$, avec $\phi_1(0)=0$ et $\alpha = \phi(0)$. Alors le support de $\phi_1$ est inclus dans l'union d'un nombre fini d'ensembles de la forme $p^i \mathbb{Z}_p^\times$. On en déduit que $\zeta(\omega, \phi_1, s)$ est un polynôme en $p^s$ et $p^{-s}$. 

D'autre part, on a $\zeta(\omega, \phi_0, s) = L(\omega, s)$ dans le cas non ramifié et $\zeta(\omega, \phi_0, s) = cp^{ms}$ dans le cas ramifié, où $c$ est une constante non nulle. Ce qui nous permet de conclure que le quotient $\frac{\zeta(\omega, \phi, s)}{L(\omega,s)}$ est un polynôme en $p^s$ et $p^{-s}$.
\end{proof}
\section{Fonctions zêta sur $GL_n(\mathbb{Q}_p), n > 1$}

Dans la suite, on notera $G = GL_n(\mathbb{Q}_p)$, $dg$ une mesure de Haar sur $G$ et $(\pi, V)$ une représentation admissible irréductible de $G$. On pose $K=GL_n(\mathbb{Z}_p)$, c'est un sous-groupe compact maximal de $G$.

\begin{definition}
Une représentation $\pi : G \rightarrow GL(V)$ sur un $\mathbb{C}$-espace vectoriel $V$ est dite admissible si elle vérifie :
\begin{itemize}
\item Pour tout $v \in V$, le stabilisateur de $v$ dans $G$, $\left\lbrace g \in G, \pi(g)v = v \right\rbrace$, est un sous-groupe ouvert de $G$,
\item Pour tout sous-groupe ouvert $H$ de $G$, le sous-espace
\begin{equation*}
V^H=\left\lbrace v \in V, \pi(h)v = v, \forall h \in H \right\rbrace
\end{equation*}
des vecteurs stable par $H$ est de dimension fini.
\end{itemize}
\end{definition}

Les coefficients de $\pi$ sont les fonctions de la forme $g \in G \mapsto <\pi(g)v, \tilde{v}>$, où $v \in V$ et $\tilde{v} \in \tilde{V}$. Alors $\check{f}(g)=f(g^{-1})=<v, \tilde{\pi}(g)\tilde{v}>$ est un coefficient de $\tilde{\pi}$.

On note $M_n$ l'ensemble des matrices $n \times n$ à coefficients dans $\mathbb{Q}_p$ et $\mathcal{S}$ l'ensemble des fonctions $\phi : M_n \rightarrow \mathbb{C}$ localement constantes à support compact.

Si $f$ est un coefficient de $\pi$, $\phi \in \mathcal{S}$ et $s \in \mathbb{C}$, on pose
\begin{equation}
\zeta(f, \phi, s) = \int_G \phi(g)f(g)|\det g|_p^s dg.
\end{equation}

On fixe un caractère non trivial $\psi$ de $\mathbb{Q}_p$ et on pose
\begin{equation}
\hat{\phi}(y) = \int_{M_n} \phi(x) \psi(Tr(xy)) dx,
\end{equation}
où $dx$ est une mesure de Haar sur $M_n$, normalisée telle que $\hat{\hat{\phi}}(x)=\phi(-x)$.

L'objectif de cette section est de montrer le
\begin{theoreme}
\label{thm_padique}
\begin{enumerate}
\item Il existe $s_0 \in \mathbb{R}$ tel que pour tout $s \in \mathbb{C}$ vérifiant $Re (s) > s_0$, $\phi \in \mathcal{S}$ et $f$ un coefficient de $\pi$, les intégrales
\begin{align}
\zeta(f, \phi, s) &= \int_G \phi(g)f(g)|\det g|_p^s dg \\
\zeta(\check{f}, \phi, s) &= \int_G \phi(g)\check{f}(g)|\det g|_p^s dg
\end{align}
convergent absolument.
\item Ces intégrales sont des fonctions rationnelles en $p^{-s}$. Plus précisément, il existe des polynômes $Q$ et $\tilde{Q}$ indépendant de $f$ et $\phi$ avec $Q(0)\neq 0$ (respectivement $\tilde{Q}(0)\neq 0$) et des polynômes $\Xi(f, \phi, s)$, $\tilde{\Xi}(\check{f}, \phi, s)$ en $p^{s}$ et $p^{-s}$ tel que
\begin{align}
\zeta(f, \phi, s+\frac{1}{2}(n-1)) &= \frac{\Xi(f, \phi, s)}{Q(p^{-s})}, \\
\zeta(\check{f}, \phi, s+\frac{1}{2}(n-1)) &= \frac{\tilde{\Xi}(\check{f}, \phi, s)}{\tilde{Q}(p^{-s})},
\end{align}
pour tout $s \in \mathbb{C}$, $\phi \in \mathcal{S}$ et $f$ coefficient de $\pi$.
\item On peut choisir un nombre fini, de coefficients $f_i$ de $\pi$ (respectivement $\tilde{\pi}$) et de fonctions $\phi_i \in \mathcal{S}$, telles que $\sum_i \Xi(f_i, \phi_i, s)$ (respectivement $\sum_i \tilde{\Xi}(f_i, \phi_i, s)$ soit une constante non nulle.
\item Il existe une fonction $\epsilon(s, \pi, \psi)$, qui est à une constante prés une puissance de $p^{-s}$, telle que
\begin{equation}
\label{epsilon}
\tilde{\Xi}(\check{f}, \hat{\phi}, 1-s) = \epsilon(s, \pi, \psi)\Xi(f, \phi, s),
\end{equation}
pour tout $s\in \mathbb{C}$, $\phi \in \mathcal{S}$ et $f$ coefficient de $\pi$.
\end{enumerate}
\end{theoreme}

On normalise $Q$ et $\tilde{Q}$ tel que $Q(0)=\tilde{Q}(0)=1$, on pose alors
\begin{equation}
L(s, \pi) = \frac{1}{Q(p^{-s})}, \quad L(s, \tilde{\pi}) = \frac{1}{\tilde{Q}(p^{-s})}.
\end{equation}

L'existence de la fonction $\epsilon(s, \pi, \psi)$ est équivalente à l'existence d'une fonction méromorphe $\gamma(s,\pi,\psi)$ telle que
\begin{equation}
\zeta(\check{f}, \hat{\phi}, 1-s+\frac{1}{2}(n-1))=\gamma(s, \pi, \psi)\zeta(f, \phi, s),
\end{equation}
pour tout $\phi \in \mathcal{S}$ et $f$ coefficient de $\pi$. Ces deux fonctions étant reliées par la relation
\begin{equation}
\label{gammaepsilon}
\epsilon(s,\pi,\psi)=\gamma(s,\pi,\psi)\frac{L(s,\pi)}{L(1-s,\tilde{\pi})}.
\end{equation}
En effet, supposons l'existence de $\gamma(s,\pi,\psi)$ alors $\epsilon(s,\pi,\psi)$ vérifie 
\begin{equation}
\tilde{\Xi}(\check{f}, \hat{\phi}, 1-s) = \epsilon(s, \pi, \psi)\Xi(f, \phi, s).
\end{equation}
On a de plus une égalité similaire avec $\epsilon(s,\tilde{\pi},\psi)$,
\begin{equation}
\Xi(f, s, \hat{\hat{\phi}}, s)=\epsilon(1-s, \tilde{\pi}, \psi)\tilde{\Xi}(\check{f}, \hat{\phi}, 1-s).
\end{equation}
Il ne nous reste plus qu'à utiliser la formule $\hat{\hat{\phi}}(x)=\phi(-x)$ pour obtenir la relation
\begin{equation}
\epsilon(s, \pi, \psi)\epsilon(1-s, \tilde{\pi}, \psi)=\omega(-1),
\end{equation}
où $\omega$ est le caractère de $\mathbb{Q}_p^\times$ tel que $\pi(z)=\omega(z)1$ pour $z\in \mathbb{Q}_p^\times$. D'après (2) et (3) du théorème, $\epsilon(s, \pi, \psi)$ est alors un polynôme en $p^s$ et $p^{-s}$, on en déduit que $\epsilon(s, \pi, \psi)$ est une puissance de $p^{-s}$ à constante prés.

\subsection{Réduction au cas supercuspidal}

Si $\pi$ est une représentation admissible (non nécessairement irréductible) de $G$, les assertions du théorème font sens pour $\pi$ et $\tilde{\pi}$, mais peuvent être fausse si $\pi$ n'est pas irréductible.

Supposons le théorème vrai pour $\pi$ et $\tilde{\pi}$. Soit $\sigma$ une sous-représentation irréductible de $\pi$. Alors les coefficients de $\sigma$ sont de la forme $<\pi(g)v,\tilde{v}>$ avec $v\in V$ et $\tilde{v} \in \tilde{V}$. Cependant, toutes ces fonctions ne sont pas des coefficients de $\sigma$. On en déduit la
\begin{proposition}
Il existe des polynômes $R$ et $\tilde{R}$ en $p^{-s}$ tel que
\begin{align}
L(s,\sigma)&=R(p^{-s})L(s,\pi), \\
L(s,\tilde{\sigma})&=\tilde{p^{-s}}L(s,\tilde{\pi}).
\end{align}
De plus,
\begin{equation}
\gamma(s,\sigma,\psi)=\gamma(s,\pi,\psi).
\end{equation}
\end{proposition}

Soit $P$ un sous-groupe parabolique propre maximal de $G$ et $U$ son radical unipotent alors $P/U \simeq G' \times G''$, où l'on note $G'=GL_{n'}(\mathbb{Q}_p)$ et $G''=GL_{n''}(\mathbb{Q}_p)$.

Soit $\sigma'$ (respectivement $\sigma''$) une représentation admissible de $G'$ (respectivement $G''$). On ne les suppose pas irréductible, on suppose cependant qu'ils admettent des caractères centraux $\omega'$ et $\omega''$. Alors $\sigma' \boxtimes \sigma''$ est naturellement une représentation de $P/U$, donc une représentation de $P$ triviale sur $U$.
\begin{proposition}
Notons $\pi = Ind_P^G(\sigma' \boxtimes \sigma'')$. Supposons le théorème vrai pour $\sigma'$ et $\sigma''$. Alors le théorème est vrai pour $\pi$. De plus, on a
\begin{align}
L(s,\pi)&=L(s,\sigma')L(s,\sigma''), \\
L(s,\tilde{\pi})&=L(s,\tilde{\sigma}')L(s,\tilde{\sigma}''), \\
\epsilon(s,\pi,\psi)&=\epsilon(s,\sigma',\psi)\epsilon(s,\sigma'',\psi).
\end{align}
\end{proposition}

\begin{proof}
On notera $M'=M_{n'}(\mathbb{Q}_p)$ et $M''=M_{n''}(\mathbb{Q}_p)$. Soit $f$ un coefficient de $\pi$, $\phi \in \mathcal{S}$ et $s \in \mathbb{C}$.

L'espace vectoriel $V$ sur lequel $\pi$ agit est l'espace des fonctions $v : G \rightarrow W$ localement constante qui vérifient
\begin{equation}
v(pg)=\delta_P^{\frac{1}{2}}(p)(\sigma' \boxtimes \sigma'')(p)v(g),
\end{equation}
où $\delta_P$ est le caractère modulaire de $P$ et $W$ est l'espace vectoriel sur lequel $\sigma' \boxtimes \sigma''$ agit.

Le coefficient $f$ est alors de la forme
\begin{align}
f(g)&=<\pi(g)v,\tilde{v}> \\
&= \int_K <v(kg),\tilde{v}(k)>_W dk.
\end{align}

Posons $t=s+\frac{1}{2}(n-1)$, $t'=s+\frac{1}{2}(n'-1)$ et $t''=s+\frac{1}{2}(n''-1)$. L'intégrale zêta est donc
\begin{equation}
\zeta(f,\phi,s)=\int_G \phi(g)|\det g|_p^t \int_K <v(kg),\tilde{v}(k)>dk dg.
\end{equation}
On échange l'ordre d'intégration et on fait le changement de variables $g \mapsto k^{-1}g$, on obtient
\begin{equation}
\label{integrale1}
\int_K \int_G \phi(k^{-1}g)|\det g|^t<v(g),\tilde{v}(k)>dg dk.
\end{equation}
On utilise la décomposition de Cartan pour écrire $g \in G$ sous la forme $g = \begin{pmatrix} 
g' & u \\
0 & g'' 
\end{pmatrix} k'$, où $g' \in G'$, $g'' \in G''$, $u \in U$ et $k' \in K$. On peut alors décomposer la mesure de Haar de $G$ en fonction des mesures de Haar de $G'$, $G''$, $U$ et $K$. En effet,
\begin{equation}
dg = |\det g'|^{-n''}dg'dg''dudk'.
\end{equation}
L'expression (\ref{integrale1}) devient
\begin{equation}
\label{integrale2}
\begin{split}
\int_K \int_{G' \times G'' \times U \times K} &\phi(k^{-1}\begin{pmatrix} 
g' & u \\
0 & g'' 
\end{pmatrix} k') |\det g'|^{t'}|\det g''|^{t''} \\
&<(\sigma'(g') \boxtimes \sigma''(g''))v(k'), \tilde{v}(k)> dg' dg'' du dk' dk.
\end{split}
\end{equation}

Le facteur $<(\sigma'(g') \boxtimes \sigma''(g''))v(k'), \tilde{v}(k)>$ est un coefficient de $\sigma' \boxtimes \sigma''$, donc est une combinaison linéaire de produits de coefficients de $\sigma'$ et de coefficients de $\sigma''$ :
\begin{equation}
<(\sigma'(g') \boxtimes \sigma''(g''))v(k'), \tilde{v}(k)> = \sum_{i=1}^l \lambda_i(k,k')f_i'(g')f_i''(g''),
\end{equation}
où les fonctions $\lambda_i : K \times K \rightarrow \mathbb{C}$ sont localement constante et les $f_i'$ (respectivement $f_i''$) sont des coefficients de $\sigma'$ (respectivement $\sigma''$).

D'autre part, la fonction
\begin{equation}
(x' \in M', x'' \in M'') \mapsto \int_U \phi(k^{-1}\begin{pmatrix} 
x' & u \\
0 & x'' 
\end{pmatrix} k') du
\end{equation}
est un élément de l'espace de Schwartz $\mathcal{S}(M' \times M'')$. On peut donc l'écrire sous la forme
\begin{equation}
\label{u_integrale}
\int_U \phi(k^{-1}\begin{pmatrix} 
x' & u \\
0 & x'' 
\end{pmatrix} k') du = \sum_{j=1}^{l'} \mu_j(k, k')\phi_j'(x')\phi_j''(x''),
\end{equation}
où les $\mu_j$ sont localement constantes et $\phi_j' \in \mathcal{S}(M')$ (respectivement $\phi_j'' \in \mathcal{S}(M'')$).

En remplaçant ces expressions dans l'intégrale (\ref{integrale2}), on trouve
\begin{equation}
\label{zetainduite}
\zeta(f, \phi, t) = \sum_{i,j=1}^{l,l'} \int_{K \times K} \lambda_i(k,k')\mu_j(k,k') dk dk' \zeta(f_i',\phi_j',t') \zeta(f_i'',\phi_j'',t'').
\end{equation}

D'après les hypothèses faites sur $\sigma'$ et $\sigma''$, les intégrales définissant les $\zeta(f_i',\phi_j',t')$ (respectivement $\zeta(f_i'',\phi_j'',t'')$) sont absolument convergentes pour $Re(s)$ assez grande. Ce qui justifie à posteriori les calculs que l'on vient de faire et prouve la partie (1) du théorème pour $\pi$.

D'après (\ref{zetainduite}) et les hypothèses faites sur $\sigma'$ et $\sigma''$, on obtient la relation
\begin{equation}
\zeta(f,\phi,s)=\sum_{i,j=1}^{l,l'}c_{i,j}\Xi(f_i',\phi_j',s)L(s,\sigma')\Xi(f_i'',\phi_j'',s)L(s,\sigma'').
\end{equation}
Ce qui prouve la partie (2) du théorème pour $\pi$.

Passons à la partie (4) du théorème. La valeur $\zeta(\check{f}, \hat{\phi}, t)$ s'obtient en remplaçant $f$ par $\check{f}$, ce qui remplace les $f_i'$ et $f''_i$ en $\check{f}_i'$ et $\check{f}_i''$, et $\phi$ en $\hat{\phi}$. Voyons maintenant comment ce dernier changement affecte l'intégrale. Montrons que l'équation (\ref{u_integrale}) se transforme en
\begin{equation}
\int_U \hat{\phi}(k'^{-1}\begin{pmatrix} 
x' & u \\
0 & x'' 
\end{pmatrix} k) du = \sum_{j=1}^{l'} \mu_j(k, k')\hat{\phi}_j'(x')\hat{\phi}_j''(x'').
\end{equation}
En effet, 
\begin{align}
\int_U \hat{\phi}(k'^{-1}\begin{pmatrix} 
x' & u \\
0 & x'' 
\end{pmatrix} k) du &= \int_U \int_{M_n} \phi(k^{-1}xk')\psi(Tr(\begin{pmatrix} 
x_1 & x_2 \\
x_3 & x_4 
\end{pmatrix}\begin{pmatrix} 
x' & u \\
0 & x'' 
\end{pmatrix})dxdu \\
&= \int \phi(k^{-1}\begin{pmatrix} 
x_1 & x_2 \\
0 & x_4 
\end{pmatrix}k')\psi(x_1x'+x_4x'')dx_1dx_2dx_4 \\
&= \sum_{j=1}^{l'} \mu_j(k, k')\hat{\phi}_j'(x')\hat{\phi}_j''(x'').
\end{align}
La première égalité s'obtient en considérant la transformée de Fourier en les variables $(x_3, u)$. La dernière s'obtient en appliquant la transformée de Fourier sur $M'\times M''$ à l'équation (\ref{u_integrale}).

Ces considérations nous donnent une égalité similaire à (\ref{zetainduite}),
\begin{equation}
\zeta(\check{f},\phi,1-s+\frac{1}{2}(n-1))=\sum_{i,j=1}^{l,l'}c_{i,j}\Xi(\check{f}_i',\hat{\phi}_j',1-s)L(1-s,\tilde{\sigma}')\Xi(\check{f}_i'',\hat{\phi}_j'',1-s)L(1-s,\tilde{\sigma}'').
\end{equation}
On obtient ainsi l'équation fonctionnelle
\begin{equation}
\tilde{\Xi}(\check{f}, \hat{\phi}, 1-s)=\epsilon(s, \sigma', \psi)\epsilon(s,\sigma'',\psi)\Xi(f,\phi,s),
\end{equation}
on en déduit que $\epsilon(s,\pi,\psi)=\epsilon(s, \sigma', \psi)\epsilon(s,\sigma'',\psi)$ et la partie (4) du théorème pour $\pi$.

Il ne reste plus qu'à prouver la partie (3). Il suffit de montrer que si l'on fixe $\phi' \in \mathcal{S}(M')$, $\phi'' \in \mathcal{S}(M'')$ et $f$ (respectivement $f'$) coefficient de $\sigma'$ (respectivement $\sigma''$) alors il existe $\phi \in \mathcal{S}(M)$ et $f$ coefficient de $\pi$ tel que
\begin{equation}
\zeta(f,\phi,t)=\zeta(f',\phi',t')\zeta(f'',\phi'',t'').
\end{equation}
En effet, le calcul du produit des fonctions zêta $\zeta(f',\phi',t')\zeta(f'',\phi'',t'')$ donne
\begin{equation}
\label{zetaprod}
\int_{G' \times G''} \phi'(g')\phi''(g'')f'(g')f''(g'')|\det g'|_p^{t'}|\det g''|_p^{t''}dg'dg''.
\end{equation}
On choisit alors $\phi \in \mathcal{S}(M)$ de la forme $\begin{pmatrix} 
x' & u \\
v & x'' 
\end{pmatrix} \mapsto \phi'(x')\phi''(x'')\phi_0(u)\phi_1(v)$, où $\phi_1 \in \mathcal{S}(M_{n'',n'})$ vérifie $\phi_1(0)=1$ et $\phi_0 \in \mathcal{S}(M_{n',n''})$ est d'intégrale $1$. Avec ce choix, on a
\begin{equation}
\int_U \phi(\begin{pmatrix} 
g' & u \\
0 & g'' 
\end{pmatrix})=\phi'(g')\phi''(g'').
\end{equation}
De plus, il existe une fonction localement constante $\eta : K \rightarrow \mathbb{C}$ telle que
\begin{equation}
\int_{U \times K} \phi(\begin{pmatrix} 
g' & u \\
0 & g'' 
\end{pmatrix}k)\eta(k)dudk = \phi(g')\phi(g'').
\end{equation}
On pose aussi $f(g)=\delta_P^{\frac{1}{2}}(\begin{pmatrix} 
g' & u \\
0 & g'' 
\end{pmatrix})\eta(k)f(g')f(g'')$, alors $f$ est bien un coefficient de $\pi$. De plus, en intégrant sur $U \times K$ l'expression (\ref{zetaprod}) devient
\begin{equation}
\int_G \phi(\begin{pmatrix} 
g' & u \\
0 & g'' 
\end{pmatrix}k)f(g)|\deg g|_p^t\delta_P(\begin{pmatrix} 
g' & u \\
0 & g'' 
\end{pmatrix})dg'dg''dudk,
\end{equation}
qui est bien $\zeta(f,\phi,t)$. Ce qui termine la preuve de la proposition.
\end{proof}
\subsection{Représentation supercuspidale}

Dans cette partie, on suppose que $\pi$ est une représentation supercuspidale irréductible de $G$. Avant d'aller plus loin, commençons par rappeler un résultat fondamental sur les représentations supercuspidales.

\begin{proposition}
Les coefficients de $\pi$ sont à support compact modulo $\mathbb{Q}_p^\times$.
\end{proposition}

Soit $f$ un coefficient de $\pi$ et $\phi \in \mathcal{S}$, alors il existe un sous-groupe compact $K'$ de $G$ tel que $f$ et $\phi$ sont invariants à gauche par $K'$. De plus, le support de $f$ est, d'après la proposition, à support compact modulo $\mathbb{Q}_p^\times$. Il existe donc un nombre fini d'éléments $(g_i)_{1 \leq i \leq N}$ de $G$ tel que
\begin{equation}
supp(f) \subset \cup_{i=1}^N K'\mathbb{Q}_p^\times g_i.
\end{equation}

On en déduit que
\begin{equation}
\zeta(f,\phi,s) = \frac{vol(K')}{vol(K' \cap \mathbb{Q}_p^\times)} \sum_{i=1}^N f(g_i)|\det g_i|_p^s \int_{\mathbb{Q}_p^\times} \phi(xg_i)|x|_p^{ns}\omega(x)dx,
\end{equation}
où $\omega$ est le caractère central de $\pi$. Cette dernière intégrale est absolument convergente pour $Re(s) > 0$. De plus, le quotient $\frac{\zeta(f,\phi,s)}{L(ns,\omega)}$ est un polynôme en $p^s$ et $p^{-s}$. Ce qui prouve les parties (1) et (2) du théorème pour $\pi$.

Posons $G^0=\left\lbrace g \in G, |\det g|_p = 1 \right\rbrace$, alors $G^0 \cap \mathbb{Q}_p^\times = \mathbb{Z}_p^\times$ est compact. On choisit $\phi \in \mathcal{S}$ tel que $\phi(g) = \overline{f(g)}$ si $g \in G^0$ et $\phi(g)=0$ sinon. Alors
\begin{equation}
\zeta(f, \phi, s) = \int_{G^0} f(g)\overline{f(g)} dg > 0
\end{equation}
est une constante non nulle, ce qui prouve la partie (3) du théorème pour $\pi$. Il ne nous reste plus qu'à montrer l'équation fonctionnelle.

Commençons par définir l'opérateur zêta,
\begin{equation}
\zeta(\pi, \phi, s) = \int_G \phi(g)|\det g|_p^s\pi(g) dg.
\end{equation}
C'est l'opérateur dont les coefficients sont exactement les $\zeta(f, \phi,s)$ pour $f$ coefficient de $\pi$.

Posons $\mathcal{S}_0=\left\lbrace \phi \in \mathcal{S}| supp(\phi), supp(\hat{\phi}) \subset G \right\rbrace$. Le résultat qui va nous permettre de prouver l'équation fonctionnelle est la
\begin{proposition}
\label{pre-eq-func}
Pour $\phi \in \mathcal{S}, \phi' \in \mathcal{S}_0$, on a
\begin{equation}
\zeta(\check{\pi}, \hat{\phi}', n-s)\zeta(\pi,\phi,s) = \zeta(\pi, \phi',s)\zeta(\check{\pi}, \hat{\phi}, n-s),
\end{equation}
où $\check{\pi}(g) = \pi(g^{-1})$.
\end{proposition}

\begin{proof}
La proposition est une conséquence immédiate du
\begin{lemme}
Soit $\phi \in \mathcal{S}, \phi' \in \mathcal{S}_0, v \in V$ et $\tilde{v} \in \tilde{V}$, pour $0 < Re(s) < n$, les intégrales
\begin{align}
&\int_G \int_G \phi(g)\hat{\phi}'(h)<\pi(g)v,\tilde{\pi}(h)\tilde{v}>|\det g|_p^s|\det h|_p^{n-s}dg dh, \\
&\int_G \int_G \hat{\phi}(g)\phi'(h)<\pi(g^{-1}v,\tilde{\pi}(h^{-1})\tilde{v}>|\det g|_p^{n-s}|\det h|_p^s dg dh,
\end{align}
sont absolument convergentes et coïncident. De plus, ces intégrales sont les coefficients des opérateurs $\zeta(\check{\pi}, \hat{\phi}', n-s)\zeta(\pi,\phi,s)$ et $\zeta(\pi, \phi',s)\zeta(\check{\pi}, \hat{\phi}, n-s)$.
\end{lemme}

\end{proof}

\begin{proposition}
Pour $s \in \mathbb{C}$, il existe un opérateur $\gamma(s) : V \rightarrow V$ tel que
\begin{equation}
\zeta(\check{\pi}, \hat{\phi}, n-s) = \gamma(s)\zeta(\pi,\phi,s), \forall \phi \in \mathcal{S}_0.
\end{equation}
De plus, l'opérateur $\gamma(s)$ est un scalaire.
\end{proposition}

\begin{proof}
Unicité : On choisit $\phi \in \mathcal{S}_0$ tel que $\zeta(\pi,\phi,s)=Id_V$, alors $\gamma(s)=\zeta(\check{\pi}, \hat{\phi}, n-s)$.

Existence : Il faut démontrer que les différents $\phi \in \mathcal{S}_0$ tel que $\zeta(\pi,\phi,s)=Id_V$ donnent un même opérateur $\zeta(\check{\pi},\hat{\phi}, n-s)$. Soit $\phi_1,\phi_2 \in \mathcal{S}_0$ tel que $\zeta(\pi,\phi_1,s)=\zeta(\pi,\phi_2,s)=Id_v$. D'après la proposition (\ref{pre-eq-func}), on en déduit que $\zeta(\check{\pi},\hat{\phi_1}, n-s)=\zeta(\check{\pi},\hat{\phi_2}, n-s)$.

On pose $\gamma(s)=\zeta(\check{\pi},\hat{\phi}_0, n-s)$ pour $\phi_0 \in \mathcal{S}_0$ tel que $\zeta(\pi,\phi,s)=Id_V$. Alors, d'après la proposition (\ref{pre-eq-func}),
\begin{align}
\gamma(s)\zeta(\pi,\phi,s) &= \zeta(\check{\pi},\hat{\phi}_0, n-s)\zeta(\pi,\phi,s) \\
&= \zeta(\pi,\phi_0,s)\zeta(\check{\pi}, \hat{\phi}, n-s) \\
&= \zeta(\check{\pi}, \hat{\phi},s),
\end{align}
pour tout $\phi \in \mathcal{S}$.

Montrons maintenant que $\gamma(s) \in \Hom_G(\pi,\pi)$, le lemme de Schur nous permet de conclure que $\gamma(s)$ est un scalaire.

Pour $\phi \in \mathcal{S}$, on pose $\phi_h = \phi(h.)$. Alors $\hat{\phi}_h = |\det h|_p^{-n}\hat{\phi}(.h^{-1})$. Ce qui nous permet d'obtenir
\begin{align}
\zeta(\check{\pi}, \hat{\phi}_h, n-s)&=|\det h|_p^{-n}\zeta(\check{\pi}, \hat{\phi}(.h^{-1}),n-s) \\
&= |\det h|_p^{-s}\pi(h^{-1})\zeta(\check{pi}, \hat{\phi}, n-s) \\
&= |\det h|_p^{-s}\pi(h^{-1})\gamma(s)\zeta(\pi,\phi,s).
\end{align}
D'autre part, on a
\begin{equation}
\gamma(s)\zeta(\pi,\phi_h,s)=|\det h|_p^{-s}\gamma(s)\pi(h^{-1})\zeta(\pi,\phi,s).
\end{equation}
Par unicité de l'opérateur $\gamma(s)$, on en déduit que $\pi(h^{-1})\gamma(s)=\gamma(s)\pi(h^{-1})$. Autrement dit, $\gamma(s) \in \Hom_G(\pi,\pi)$.
\end{proof}

\begin{lemme}
Soit $v \in V, \tilde{v} \in \tilde{V}$ et $s \in \mathbb{C}$, il existe $\phi \in \mathcal{S}_0$ tel que
\begin{equation}
\zeta(\pi,\phi,s)w = <w,\tilde{v}>v, \forall w \in V.
\end{equation}
En particulier, il existe $\phi \in \mathcal{S}_0$ tel que $\zeta(\pi, \phi, s) = Id_V.$
\end{lemme}

\begin{proof}
Soit $\phi \in \mathcal{S}_0$ tel que $\phi(g) = |\det g|_p^s<v, \tilde{\pi}\tilde{v}>$ si $|\det g|_p \in \left\lbrace 1, p, ..., p^{n-1} \right\rbrace$ et $\phi(g) = 0$ sinon. Alors
\begin{align}
<\zeta(\pi,\phi,s)w,\tilde{w}> &= \int_{G/\mathbb{Q}_p^\times} <v,\tilde{\pi}(g)\tilde{v}><\pi(g)w,\tilde{w}>dg \\
&= c <v, \tilde{w}><w,\tilde{v}>,
\end{align}
pour tout $w \in V, \tilde{w} \in \tilde{V}$. La dernière égalité est une conséquence du lemme de Schur. Ce qui montre que $\zeta(\pi, \phi,s)$ est proportionnel à $w \mapsto <w,\tilde{v}>v$.
\end{proof}

\begin{lemme}
Soit $v \in V$ non nul, alors
$$W=\left\lbrace u \in V, \exists \phi \in \mathcal{S}_0, c \neq 0, l \in \mathbb{Z}, \zeta(\pi, \phi, s) = cp^{-ls}u \forall s \in \mathbb{C} \right\rbrace$$
engendre $V$.
\end{lemme}

\begin{proof}
Si $u \in W$, alors $\pi(h)u$ l'est aussi pour $h \in G$. En effet,
$\zeta(\pi,\phi(.h^{-1}),s) = cp^{-ls}|\det h|_p^{-s}u$. Comme $V$ est irréductible, il suffit de montrer que $W \neq 0$.

Soit $\phi \in \mathcal{S}_0$ tel que $\phi(g) = <v, \pi(g)v>$ si $g \in G^0$ et $\phi(g) = 0$ sinon. Alors
\begin{equation}
u=\zeta(\pi, \phi, s)v = \int_{G^0} <v,\pi(g)v>\pi(g)v dg
\end{equation}
est indépendant de $s$ et non nul puisque
\begin{equation}
<\zeta(\pi,\phi,s)v,v> = \int_{G^0} |<v,\pi(g)v>|^2 dg > 0.
\end{equation}
Ce qui montre que $u \in W$ et $u$ non nul.
\end{proof}

Montrons que $\gamma(s)$ est non seulement une fraction rationnelle en $p^{-s}$, mais en fait une puissance de $p^s$. En effet, on a
\begin{equation}
\zeta(\check{f}, \hat{\phi}, n-s)=\gamma(s)\zeta(f,\phi,s), \forall \phi \in \mathcal{S}_0.
\end{equation}
D'après le lemme, on peut choisir $\phi \in \mathcal{S}_0$ et $f$ coefficient de $\pi$ tel que $\zeta(f,\phi,s)=p^{-ls}$. Alors $\gamma(s) = \zeta(\check{f}, \hat{\phi}, n-s)p^{ls}$ est un polynôme en $p^{-s}$ et $p^s$. En appliquant le lemme à $\check{\pi}$, on en déduit que $\gamma$ n'admet pas de zéros, c'est donc une puissance de $p^{s}$.

\begin{proposition}
Pour $\pi$ supercuspidale irréductible, on a $L(s,\pi)=1$.
\end{proposition}

\begin{proof}
Si $\omega$ est ramifié, alors $L(s,\omega)=1$. On en déduit que
$L(s,\pi)=\frac{L(s,\pi)}{L(ns,\omega)}$ est un polynôme en $p^{-s}$, donc $L(s, \pi)=1$.

Si $\omega$ est non ramifié, on peut supposer sans perte de généralité que $\omega=1$, alors
\begin{equation}
L(s,\omega)=\frac{1}{1-p^{-s}}, \quad L(ns,\omega)=\frac{1}{\prod_{\mu^n=1}(1-\mu p^{-s})}=\frac{1}{1-p^{-ns}}.
\end{equation}
Ce qui nous permet d'en déduire que
\begin{equation}
L(s,\pi) = \frac{1}{\prod_{\mu \in T}(1-\mu p^{-s})}, \quad L(s,\tilde{\pi}) = \frac{1}{\prod_{\mu \in T'}(1-\mu p^{-s})},
\end{equation}
où $T$ et $T'$ sont des sous-ensembles des racines $n$-ième de l'unité.

On vient de montrer précédemment que $\gamma$ est une puissance de $p^s$, il en est alors de même pour $\epsilon(s,\pi,\psi)$ et $\frac{L(s,\pi)}{L(1-s,\tilde{\pi})}$ d'après la relation (\ref{gammaepsilon}). Ce qui montre que la fraction
\begin{equation}
\frac{\prod_{\mu \in T'}(1-\mu p^{s-1})}{\prod_{\mu \in T}(1-\mu p^{-s})}
\end{equation}
est une puissance de $p^s$, d'où $L(s,\pi)=L(s,\tilde{\pi})=1$.
\end{proof}
\subsection{Représentation sphérique}

\subsection{Représentation de carré intégrable}
\section{Seconde preuve du théorème \ref{thm_padique}}

On reprend les notations de la section \ref{gln}. Dans cette partie, on propose de présenter une autre preuve qui se base sur un dévissage de l'espace de Schwartz.

\subsection{Équation fonctionnelle}
On veut montrer l'équation fonctionnelle suivante
\begin{equation}
\zeta(f, \phi, s) = \gamma(s) \zeta(\check{f}, \hat{\phi}, n-s),
\end{equation}
où $\gamma$ est une fonction rationnelle en $p^s$ et $\check{f}(g) = f(g^{-1})$.

Pour montrer cette équation fonctionnelle, on va utiliser la
\begin{propriete}
Les opérateurs $\zeta(., ., s)$ et $\zeta(\check{.}, \hat{.}, n-s)$ sont des opérateurs d'entrelacements, éléments de $\Hom_{G \times G} ( (\tilde{\pi} \boxtimes \pi) \otimes \mathcal{S}, |\det|_p^s \boxtimes |\det|_p^{-s})$.
\end{propriete}

On précise que l'action de $G \times G$ sur $\mathcal{S}$ est
$(g_1,g_2).\phi(x) = \phi(g_1^{-1} x g_2)$. De plus, on identifie l'ensemble des coefficients de $\pi$ avec l'espace $\tilde{V}\otimes V$; l'action de $G \times G$ sur $\tilde{\pi} \boxtimes \pi$ est $(g_1,g_2).f(g) = f(g_1^{-1} g g_2)$.

\begin{proof}
L'action de $G \times G$ sur $\zeta(f,\phi,s)$ donne
\begin{equation}
\int_G \phi(g_1^{-1}gg_2)f(g_1^{-1}gg_2)|\det g|_p^s dg.
\end{equation}
On effectue le changement de variable $g \mapsto g_1gg_2^{-1},$ le groupe $G$ étant unimodulaire l'intégrale devient
\begin{equation}
|\det g_1g_2^{-1}|^s\int_G \phi(g)f(g)|\det g|_p^s dg.
\end{equation}

D'autre part, l'action de $G \times G$ sur $\zeta(\check{f}, \hat{\phi}, n-s)$ donne
\begin{equation}
\label{intzetafourier}
\int_G \hat{\phi}_{g_1,g_2}(g)\check{f}_{g_1,g_2}(g)|\det g|_p^{n-s} dg,
\end{equation}
où l'on a noté $\phi_{g_1,g_2}(x) = \phi(g_1^{-1}xg_2)$ et $f_{g_1,g_2}(g) = f(g_1^{-1}gg_2).$

Un calcul immédiat, montre que $\check{f}_{g_1,g_2}(g) = \check{f}(g_2^{-1}gg_1)$. De plus,
\begin{equation}
\hat{\phi}_{g_1,g_2}(g) = \int_{M_n} \phi(g_1^{-1}xg_2) \psi(Tr(xg)) dx.
\end{equation}
Après le changement de variable $x \mapsto g_1xg_2^{-1}$ l'intégrale devient
\begin{equation}
|\det g_1^{-1}g_2|_p^n\int_{M_n} \phi(x) \psi(Tr(xg_2^{-1}gg_1)) dx,
\end{equation}
qui n'est autre que $|\det g_1g_2^{-1}|_p^n\hat{\phi}(g_2^{-1}gg_1)$.
L'intégrale (\ref{intzetafourier}) devient donc, après le changement de variable $g \mapsto g_2gg_1^{-1}$,
\begin{equation}|\det g_1^{-1}g_2|_p^n|\det g_2g_1^{-1}|_p^{n-s}\int_G \hat{\phi}(g)\check{f}(g)|\det g|_p^{n-s} dg.
\end{equation}
\end{proof}

Dans le but de comprendre l'espace $\Hom_{G \times G} ( (\tilde{\pi} \boxtimes \pi) \otimes \mathcal{S}, |\det|_p^s \boxtimes |\det|_p^{-s})$, on va décomposer $\mathcal{S}$ selon le rang des matrices. Soit $r$ un entier compris entier $1$ et $n$, on note $S_r$ l'espace des matrices $n \times n$ de rang $r$ et $S^{(r)}$ l'espace des matrices $n \times n$ de rang $< r$.

Si $X$ est un espace localement compact totalement discontinu, on note $C^\infty_c(X)$ l'espace des fonctions $f : X \rightarrow \mathbb{C}$ localement constantes à support compact. L'espace $\mathcal{S}$ est donc égal à $C^\infty_c(M_n)$.

Le groupe $G$ est un ouvert de $M_n$ et $M_n \setminus G = S^{(n)}$. Cette décomposition donne la suite exacte
\begin{equation}
\label{suiteexacte}
0 \rightarrow C^\infty_c(G) \rightarrow C^\infty_c(M_n) \rightarrow C^\infty_c(S^{(n)}) \rightarrow 0,
\end{equation}
où l'inclusion de $C^\infty_c(G)$ dans $C^\infty_c(M_n)$ se fait par extension par $0$ et l'application $C^\infty_c(M_n) \rightarrow C^\infty_c(S^{(n)})$ est l'application de restriction.

Cette suite exacte commute avec l'action de $G \times G$, on la voit donc comme une suite exacte de représentations de $G \times G$. On applique le foncteur $\Hom_{G \times G} (., (\pi \boxtimes \tilde{\pi}) \otimes (|\det|_p^s \boxtimes |\det|_p^{-s}))$, qui est exact à gauche, on en déduit alors l'inégalité suivante :
\begin{multline}
\dim \Hom_{G \times G} ((\tilde{\pi} \boxtimes \pi) \otimes \mathcal{S}, |.|_p^s) \leq \dim \Hom_{G \times G} ((\tilde{\pi} \boxtimes \pi) \otimes C^\infty_c(G), |.|_p^s) \\
+ \dim \Hom_{G \times G} ((\tilde{\pi} \boxtimes \pi) \otimes C^\infty_c(S^{(n)}), |.|_p^s),
\end{multline}
où l'on a abrégé $|.|_p^s = |\det|_p^s \boxtimes |\det|_p^{-s}$.

On décompose ensuite $S^{(n)}$ selon le rang $r$, ce qui donne, en utilisant le même raisonnement, que
\begin{equation}
\dim \Hom_{G \times G} ((\tilde{\pi} \boxtimes \pi) \otimes \mathcal{S}, |.|_p^s) \leq \sum_{r=0}^{n} \dim \Hom_{G \times G} ((\tilde{\pi} \boxtimes \pi) \otimes C^\infty_c(S_r), |.|_p^s).
\end{equation}

Il ne nous reste plus qu'à calculer la dimension de ces différents espaces, pour cela on dispose de la
\begin{proposition}
Pour $r=n$ $(S_r = G)$, on a
\begin{equation}
\dim \Hom_{G \times G} ( (\tilde{\pi} \boxtimes \pi) \otimes C_{c}^\infty(G), |.|_p^s) = 1;
\end{equation}
et pour $r < n$, on a
\begin{equation}
\Hom_{G \times G} ( (\tilde{\pi} \boxtimes \pi) \otimes C_{c}^\infty(S_r), |.|_p^s) = 0
\end{equation}
sauf pour un nombre fini de valeurs de $s$ modulo $\frac{2i\pi}{(n-r)\log p}\mathbb{Z}$.
\end{proposition}

\begin{proof}

Commençons par le cas $r=n$,
\begin{align}
\Hom_{G \times G}( (\tilde{\pi} \boxtimes \pi) \otimes C_c^\infty(G), |.|_p^s) &\simeq \Hom_{G \times G}( (\tilde{\pi} \boxtimes \pi) \otimes |.|_p^{-s}, C^\infty(G)) \\
&\simeq \Hom_H( (\tilde{\pi} \boxtimes \pi) \otimes |.|_p^{-s}, \mathbb{C}) \\
&\simeq \Hom_G(\tilde{\pi}, \tilde{\pi});
\end{align}
où le groupe $H$ désigne la diagonale de $G \times G$. Ce dernier espace est bien de dimension $1$ d'après le lemme de Schur.

Le premier isomorphisme provient de la dualité entre $C_c^\infty(G)$ et $C^\infty(G)$. Le deuxième isomorphisme est une application de la réciprocité de Frobenius avec l'identification $C^\infty(G) = Ind_H^{G \times G}(1)$. Pour finir, le dernier isomorphisme provient du fait que l'action diagonale de $H$ sur $\tilde{\pi} \boxtimes \pi$ correspond à l'action de $G$ sur $\tilde{\pi} \otimes \pi$ et que $|.|_p^{-s}$ est trivial sur $H$.

Passons au cas $r < n$, $S_r$ est l'orbite de $\begin{pmatrix} 
1_r & 0 \\
0 & 0 
\end{pmatrix}$ sous l'action de $G \times G$ par translation à gauche du premier facteur et translation à droite de l'inverse sur le second facteur. On calcule le stabilisateur,
\begin{equation}
H = Stab_{G \times G} \begin{pmatrix} 
1_r & 0 \\
0 & 0 
\end{pmatrix} = \left\lbrace \left(\begin{pmatrix} 
a & b \\
0 & c 
\end{pmatrix}, \begin{pmatrix} 
a & 0 \\
d & e 
\end{pmatrix} \right) \right\rbrace \subset G \times G,
\end{equation}
où $a$ décrit $GL_r(\mathbb{Q}_p)$; $c, e$ décrivent $GL_{n-r}(\mathbb{Q}_p)$; $b$ décrit $M_{r,n-r}(\mathbb{Q}_p)$ et $d$ décrit $M_{n-r,r}(\mathbb{Q}_p)$.
 
On note $P = MN$ le sous-groupe parabolique de $G$ des matrices de la forme $\begin{pmatrix} 
a & b \\
0 & c 
\end{pmatrix}$ et $\bar{P} = M\bar{N}$ le groupe parabolique opposé, alors $H \subset P \times \bar{P}$.

\begin{equation}
\label{isom}
\begin{split}
\Hom( (\tilde{\pi} \boxtimes \pi) \otimes C_c^\infty(S_r), |.|_p^s) &\simeq \Hom_{G \times G}( (\tilde{\pi} \boxtimes \pi) \otimes |.|_p^{-s}, Ind_H^{G \times G}(\delta_H)) \\
&\simeq \Hom_{M \times M}( (\tilde{\pi} \boxtimes \pi)_{N \times \bar{N}} \otimes |.|_p^{-s}, Ind_{(M \times M)\cap H}^{M \times M}(\delta_H)) \\
&\simeq \Hom_{(M \times M) \cap H}( (\tilde{\pi} \boxtimes \pi)_{N \times \bar{N}}, \delta_H \otimes |.|_p^s),
\end{split}
\end{equation}
où $\delta_H$ est le caractère modulaire de $H$.

Le premier isomorphisme provient de l'identification de $C_c^\infty(S_r)=c-Ind_H^{G \times G}(1)$ et de la dualité entre $c-Ind_H^{G \times G}(1)$ et $Ind_H^{G \times G}(\delta_H)$. Pour le deuxième isomorphisme, on utilise la transitivité de l'induction, $H \subset P \times \bar{P} \subset G \times G$, et l'adjonction entre $Ind_{P \times \bar{P}}^{G \times G}$ et le foncteur de Jacquet; en remarquant, que $N \times \bar{N}$ agit trivialement sur $|.|_p^{-s}$. Le dernier isomorphisme n'est autre que la réciprocité de Frobenius.

On utilise le fait que $(\tilde{\pi} \boxtimes \pi)_{N \times \bar{N}}$ est de longueur finie; en effet le foncteur de Jacquet préserve la longueur finie. Il existe donc des représentations admissibles $V_i$ de $M \times M$ telles que
\begin{equation}
0=V_0 \subset V_1 \subset ... \subset V_l = (\tilde{\pi} \boxtimes \pi)_{N \times \bar{N}},
\end{equation}
avec $V_i/V_{i-1}$ irréductibles.

En reprenant un raisonnement que l'on a déjà fait, la suite exacte de représentations de $M \times M$
\begin{equation}
0 \rightarrow V_{i-1} \rightarrow V_i \rightarrow V_i/V_{i-1} \rightarrow 0
\end{equation}
permet d'obtenir l'inégalité suivante :
\begin{equation}
\dim \Hom_{(M \times M)\cap H} ((\tilde{\pi} \boxtimes \pi)_{N \times \bar{N}}, |.|_p^s\delta_H) \leq \sum_{i=1}^{l} \dim \Hom_{(M \times M)\cap H} (V_i/V_{i-1}, |.|_p^s\delta_H).
\end{equation}

Il nous suffit donc de montrer que ces derniers espaces sont nuls sauf pour au plus une valeur de $s$ modulo $\frac{2i\pi}{(n-r)\log p}\mathbb{Z}$.

En tant que représentation irréductible de $M \times M \simeq GL_r^2(\mathbb{Q}_p) \times GL_{n-r}^2(\mathbb{Q}_p)$, on peut décomposer $V_i/V_{i-1} \otimes \delta_H^{-1}$ sous la forme
$\sigma^{(i)} \boxtimes (\tau_1^{(i)} \boxtimes \tau_2^{(i)})$, où $\sigma^{(i)}$ est une représentation irréductible de $GL_r^2(\mathbb{Q}_p)$ et $\tau_1^{(i)}, \tau_2^{(i)}$ sont des représentations irréductibles de $GL_{n-r}(\mathbb{Q}_p)$.

D'après le lemme de Schur, la représentation $\tau_2^{(i)}$ admet un caractère central $\omega^{(i)}$. On en déduit que
\begin{equation}
Hom_{(M \times M) \cap H} (V_i/V_{i-1}, |.|_p^s\delta_H) = 0,
\end{equation}
sauf si $\omega^{(i)} = |.|_p^{-(n-r)s}$ sur $\mathbb{Q}_p^\times$. Cette dernière équation ne peut être vérifiée que pour au plus une valeur de $s$ modulo $\frac{2i\pi}{(n-r)\log p}\mathbb{Z}$.
\end{proof}

Terminons la preuve de l'équation fonctionnelle. Rappelons que les opérateurs $\zeta(., ., s)$ et $\zeta(\check{.}, \hat{.}, n-s)$ sont des éléments de $\Hom_{G \times G} ( (\tilde{\pi} \boxtimes \pi) \otimes \mathcal{S}, |\det|_p^s \boxtimes |\det|_p^{-s})$, qui est de dimension $1$ sauf pour un nombre fini de valeurs de $s$ modulo $\sum_{r=0}^{n-1}\frac{2i\pi}{(n-r)\log p}\mathbb{Z}$.

Autrement dit, pour $s$ en dehors de cet ensemble de valeurs exceptionnelles, il existe $\gamma(s) \in \mathbb{C}$ tel que
\begin{equation}
\label{eqfun}
\zeta(., ., s) = \gamma(s) \zeta(\check{.}, \hat{.}, n-s).
\end{equation}

Les fonctions zêta étant des fonctions rationnelles en $p^s$ et l'ensemble des valeurs de $s$ pour lesquelles $\gamma$ est ainsi défini est dense pour la topologie de Zariski, on en déduit que l'on peut étendre $\gamma$ en une fonction rationnelle en $p^s$ pour laquelle l'équation (\ref{eqfun}) est vérifiée en tant qu'égalité de fonctions rationnelles en $p^s$.

\subsection{Familles de représentations}

On veut maintenant montrer que la fonction zêta est une fraction rationnelle où l'on peut choisir le dénominateur de façon indépendante de $f$ et de $\phi$.

Si l'on note $\pi_s = \pi \otimes |\det|^s$, alors $\zeta(f, \phi, s)$ est élément de $\Hom_{G \times G}((\tilde{\pi_s} \boxtimes \pi_s) \otimes \mathcal{S}(M), \mathbb{C})$. On va donc considérer $\pi_s$ comme une famille de représentation paramétrée par $s$.

\begin{definition}
Soit $B$ une $\mathbb{C}$-algèbre commutative noethérienne réduite. Un $(G, B)$-module (ou $B$-famille de représentation de $G$) est une représentation lisse $(\pi_B, V_B)$ de $G$ muni d'une structure de $B$-module plat, qui commute à l'action de $G$.
\end{definition}

On pose $B=\mathbb{C}[G/G^0]$ l'algèbre des polynômes sur la variété des caractères non ramifiés $X^*(G)$ de $G$. On définit alors un $(G,B)$-module $(\pi_B, V_B)$ par $V_B = V \otimes B$ et $\pi_B(g)(v \times b) = \pi(g)v \otimes gb$ pour $v \in V, b \in B$ et $g \in G$.

La variété des caractères non ramifiés est $X^*(G) = \left\lbrace |\det|^s, s \in \mathbb{C} \right\rbrace$, on retrouve la famille de représentation $\pi_s$.

Lorsque l'on restreint l'espace de Schwartz à $\mathcal{S}(G)$, la fonction zêta $\zeta(f, \phi, .)$ est un polynôme en $p^s$. On en déduit que $\zeta_{|\mathcal{S}(G)}$ (que l'on étend par linéarité à $B$) est un élément de $\Hom_{B, G \times G}((\tilde{\pi_B} \boxtimes \pi_B) \otimes \mathcal{S}(G), B)$.

On reprend la suite exacte (\ref{suiteexacte}) et on applique le foncteur $\Hom_{B, G \times G} (., \pi_B \boxtimes \tilde{\pi_B})$, ce qui nous donne
\begin{equation}
\begin{split}
0 \rightarrow \Hom_{B, G \times G}((\tilde{\pi_B} \boxtimes \pi_B) \otimes C^\infty_c(S^{(n)}), B) &\rightarrow \Hom_{B, G \times G}((\tilde{\pi_B} \boxtimes \pi_B) \otimes C^\infty_c(M_n), B) \\
&\rightarrow \Hom_{B, G \times G}((\tilde{\pi_B} \boxtimes \pi_B) \otimes C^\infty_c(G), B)
\end{split}
\end{equation}

Si la dernière flèche était surjective, cela montrerait immédiatement que $\zeta \in \Hom_{B, G \times G}((\tilde{\pi_B} \boxtimes \pi_B) \otimes C^\infty_c(M_n), B)$. Cependant, en général, elle n'est pas surjective. On doit donc utiliser le terme suivant de cette suite exacte, qui est $\Ext^1_{B, G \times G}((\tilde{\pi_B} \boxtimes \pi_B) \otimes C^\infty_c(S^{(n)}), B)$.

Le but est maintenant de trouver un élément de $\Hom_{B, G \times G}((\tilde{\pi_B} \boxtimes \pi_B) \otimes C^\infty_c(G), B)$ dont l'image dans $\Ext^1_{B, G \times G}((\tilde{\pi_B} \boxtimes \pi_B) \otimes C^\infty_c(S^{(n)}), B)$ est nulle, ce qui nous permettra de le relever dans $\Hom_{B, G \times G}((\tilde{\pi_B} \boxtimes \pi_B) \otimes C^\infty_c(M_n), B)$. Plus exactement, on va montrer qu'il existe un élément de $B$ qui agit par $0$ dans $\Ext^1_{B, G \times G}((\tilde{\pi_B} \boxtimes \pi_B) \otimes C^\infty_c(S^{(n)}), B)$.

Pour $r \in \left\lbrace 1, ..., n \right\rbrace$, on décompose l'espace $S^{(r)}$ des matrices de rang $< r$ en $S^{(r)}=S^{(r-1)} \cup S_{r-1}$ où $S_{r-1}$ est l'ouvert de $S^{(r)}$ des matrices de rang $r-1$, pour obtenir la suite exacte
\begin{equation}
0 \rightarrow C^\infty_c(S_{r-1}) \rightarrow C^\infty_c(S^{(r)}) \rightarrow C^\infty_c(S^{(r-1)}) \rightarrow 0.
\end{equation}
En prolongeant la suite exacte du foncteur $\Hom_{B, G \times G} (., \pi_B \boxtimes \tilde{\pi_B})$, on en déduit la suite exacte
\begin{equation}
\begin{split}
\Ext^1_{B, G \times G}((\tilde{\pi_B} \boxtimes \pi_B) \otimes C^\infty_c(S^{(r-1)}), B) &\rightarrow \Ext^1_{B, G \times G}((\tilde{\pi_B} \boxtimes \pi_B) \otimes C^\infty_c(S^{(r)}), B) \\ 
&\rightarrow \Ext^1_{B, G \times G}((\tilde{\pi_B} \boxtimes \pi_B) \otimes C^\infty_c(S_{r-1}), B)
\end{split}
\end{equation}

\begin{lemme}
Soit $M,N,P$ des $B$-module qui satisfont la suite exacte de $B$-module
\begin{equation}
M \xrightarrow{\alpha} N \xrightarrow{\beta} P.
\end{equation}
Supposons qu'il existe $b_1, b_2 \in B$ tel que $b_1M = 0$ et $b_2P=0$. Alors $b_1b_2N=0$.
\end{lemme}

\begin{proof}
Soit $n \in N$. Alors $\beta(n) \in P$, donc $b_2\beta(n)=0$. On en déduit qu'il existe $m \in M$ tel que $\alpha(m)=b_2n$. Or $b_1m=0$, d'où $b_1b_2n=\alpha(b_1m)=0$.
\end{proof}

Ce lemme nous permet d'en déduire qu'il suffit de montrer que les espaces $\Ext^1_{B, G \times G}((\tilde{\pi_B} \boxtimes \pi_B) \otimes C^\infty_c(S_r), B)$ sont de torsion, pour $r \in \left\lbrace 0, ..., n-1 \right\rbrace$.

De plus, on dispose de l'isomorphisme suivant
\begin{equation}
\Ext^1_{B, G \times G}((\tilde{\pi_B} \boxtimes \pi_B) \otimes C^\infty_c(S_r), B) \simeq \Ext^1_{B, (M \times M) \cap H}((\tilde{\pi_B} \boxtimes \pi_B)_{N \times \bar{N}}, \delta_H).
\end{equation}
Pour justifier cet isomorphisme, on reprend la suite d'isomorphisme (\ref{isom}) et on montre que les isomorphismes passe au $Ext^1$ en prenant une résolution projective.

Considérons l'action de $(z_1,z_2) \in Z(GL_{n-r}) \times Z(GL_{n-r})$ sur ce dernier. Il agit par $z_1^{-1}z_2 \in G/G_0$ sur $(\tilde{\pi}_B \boxtimes \pi_B)_{N \times \bar{N}}$ et par $1$ sur $\delta_H$. On en déduit que $z_1^{-1}z_2-1 \in B$ agit par $0$. Si l'on choisi $z_1 \neq z_2 \mod G_0$, on en déduit que $\Ext^1_{B, G \times G}((\tilde{\pi_B} \boxtimes \pi_B) \otimes C^\infty_c(S_r), B)$ est de torsion.

Terminons le raisonnement. Notons $Q$ le polynôme qui correspond à l'élément de $B$ qui agit par $0$ dans $\Ext^1_{B, G \times G}((\tilde{\pi_B} \boxtimes \pi_B) \otimes C^\infty_c(S^{(n)}), B)$. Alors par $B$-linéarité, on en déduit que l'image de $Q\zeta_{|\mathcal{S}(G)}$ dans $\Ext^1_{B, G \times G}((\tilde{\pi_B} \boxtimes \pi_B) \otimes C^\infty_c(S^{(n)}), B)$ est nulle. Ce qui permet d'en déduire que $Q\zeta_{|\mathcal{S}(G)}$ se relève en un élément de $\Hom_{B, G \times G}((\tilde{\pi_B} \boxtimes \pi_B) \otimes \mathcal{S}(M_n), B)$, que l'on note $\zeta_0$. Alors $\zeta_0(s)$ est un élément de $\Hom_{G \times G} ( (\tilde{\pi} \boxtimes \pi) \otimes \mathcal{S}, |\det|_p^s \boxtimes |\det|_p^{-s})$, qui est de dimension $1$ pour presque tout $s \in \mathbb{C}$, donc il est proportionnel à $\zeta(., ., s)$. De plus, la restriction de $\zeta_0$ à $\mathcal{S}(G)$ est $Q\zeta_{|\mathcal{S}(G)}$. On en déduit que $\zeta_0(s) = Q(p^s, p^{-s})\zeta(.,.,s)$ pour presque tout $s$; c'est un polynôme en $p^s$ et $p^{-s}$.
\section{Fonctions zêta sur $GL_n(\mathbb{R})$}

Cette section est consacré à l'étude du cas réel. Dans la théorie globale, la composante locale archimédienne n'est pas une représentation de $GL_n(\mathbb{R})$, on va introduire une notion plus faible. On explique les modifications nécessaires.

On note $G=GL_n(\mathbb{R})$ et $\mathfrak{g}$ son algèbre de Lie. On pose $K=O_n(\mathbb{R})$, c'est un sous-groupe compact maximal de $G$.

\begin{definition}
Un $(\mathfrak{g}, K)$-module $(\pi, V)$ est un espace vectoriel $V$ muni d'actions $\pi_\mathfrak{g} : U(\mathfrak{g}) \rightarrow End(V)$ et $\pi_K : K \rightarrow GL(V)$ tels que :
\begin{itemize}
\item Pour tout $v \in V$, $\left\lbrace \pi_K(k)v, k \in K \right\rbrace$ est de dimension finie,
\item Pour tout $k \in K$ et $X \in \mathfrak{g}$, on a
$\pi_K(k)\pi_\mathfrak{g}(X) = \pi_\mathfrak{g}(Ad(k)X)\pi_K(k)$,
\item Pour tout $Y \in Lie(K)$ et $v \in V$, on a
$\pi_\mathfrak{g}(Y)v = \left.\left(\frac{d}{dt}\pi_K(\exp(tY))v\right)\right|_{t=0}.$
\end{itemize}
\end{definition}

On va restreindre l'étude aux $(\mathfrak{g}, K)$-modules unitaires (ce sont les seuls qui sont utiles pour la théorie globale, proposition \ref{unitaire}), ils sont en correspondance avec les représentations unitaires de $G$.

\begin{definition}
Un $(\mathfrak{g}, K)$-module $(\pi, V)$ est dit unitaire s'il existe une forme hermitienne définie positive $(,) : V \times V \rightarrow \mathbb{C}$ invariante au sens suivant :
\begin{itemize}
\item $(\pi_K(k)v, v') = (v, \pi_K(k^{-1})v')$,
\item $(\pi_\mathfrak{g}(X)v, v') = -(v, \pi_\mathfrak{g}(X)v')$,
\end{itemize}
pour tous $v, v' \in V, k \in K$ et $X \in \mathfrak{g}$.
\end{definition}

Soit $(\pi, V)$ une représentation unitaire irréductible de $GL_n(\mathbb{R})$.
\begin{definition}
Soient $f$ un coefficient de $\pi$ et $\phi : M_n(\mathbb{R}) \rightarrow \mathbb{C}$ une fonction de la forme
$g \mapsto e^{-\pi Tr(gg^t)}P(g)$, où $P$ est un polynôme en les coefficients de $g$; on pose
\begin{equation}
\zeta(f, \phi, s) = \int_{GL_n(\mathbb{R})} f(g) \phi(g) |\det g|^s dg,
\end{equation}
où $dg$ est une mesure de Haar sur $GL_n(\mathbb{R})$.
\end{definition}

On est maintenant prêt pour écrire l'analogue du théorème \ref{thm_padique} dans le cas réel.
\begin{theoreme}[Godement-Jacquet \cite{godement-jacquet}]
\label{thm_reel}
\begin{enumerate}
\item Il existe $s_0 \in \mathbb{R}$ tel que l'intégrale définissant la fonction zêta convergence absolument pour tout $s \in \mathbb{C}$ vérifiant $Re(s) > s_0$.
\item Il existe $\nu_1, ..., \nu_n \in \mathbb{C}$ et des polynômes $\Xi(f,\phi,s)$ tels que
\begin{equation}
\zeta(f, \phi, s+ \frac{n-1}{2}) = \Xi(f, \phi, s)\prod_{i=1}^n \pi^{-\frac{s+\nu_i}{2}}\Gamma(\frac{s+\nu_i}{2}).
\end{equation}
\item On peut choisir un nombre fini de coefficients $f_i$ de $\pi$ et de fonctions $\phi_i$ de la forme $g \mapsto e^{-\pi Tr(gg^t)}P(g)$, telles que $\sum_i \Xi(f_i, \phi_i, s)$ est une constante non nulle.
\item Il existe une fonction méromorphe $\gamma(s, \pi)$ telle que
\begin{equation}
\zeta(\check{f}, \hat{\phi}, 1-s+\frac{n-1}{2}) = \gamma(s, \pi)\zeta(f, \phi, s),
\end{equation}
où la transformée de Fourier est définie par
\begin{equation}
\hat{\phi}(x) = \int_{M_n(\mathbb{R})}\phi(y)e^{-2i\pi Tr(xy)}dx.
\end{equation}
\end{enumerate}
\end{theoreme}

\subsection{$GL_1(\mathbb{R})$}

On doit traiter séparément le cas où $n=1$, comme précédemment pour $GL_n(\mathbb{Q}_p)$. Les caractères de $\mathbb{R}$ se répartissent en deux classes, les caractères de la forme $|.|^t$ et les caractères de la forme $sgn(.)|.|^t$. On en déduit immédiatement la convergence absolue des intégrables pour $Re(s)$ assez grand. De plus, on dispose de l'analogue du lemme \ref{lemme_fun}.
\begin{lemme}
Soient $\phi_1, \phi_2 : \mathbb{R} \rightarrow \mathbb{C}$ de la forme $x \mapsto e^{-\pi x^2}P(x)$ et $\omega$ de la forme $|.|^t$ ou $sgn(.)|.|^t$. Alors
\begin{equation}
\zeta(\omega, \phi_1, s)\zeta(\omega^{-1}, \hat{\phi}_2, 1-s)=\zeta(\omega^{-1}, \hat{\phi}_1, 1-s)\zeta(\omega, \phi_2, s)
\end{equation}
pour tout $s \in \mathbb{C}$ dans le domaine de convergence absolue.
\end{lemme}
La preuve est la même que dans le cas de $GL_1(\mathbb{Q}_p)$. Il nous suffit donc de faire le calcul explicite des membres de l'équation fonctionnelle pour des fonctions $\phi$ bien choisis.

On pose $\phi_0(x) = e^{-\pi x^2}$ et $\phi_{sgn}(x)=xe^{-\pi x^2}$. Le calcul des transformées de Fourier est un résultat classique, $\hat{\phi}_0 = \phi_0$ et $\hat{\phi}_{sgn} = i\phi_{sgn}$. On vérifie alors que
\begin{align}
\zeta(|.|^t, \phi_0, s) &= \pi^{-\frac{s+t}{2}}\Gamma(\frac{s+t}{2}), \\
\zeta(sgn(.)|.|^t, \phi_{sgn}, s) &= \pi^{-\frac{s+t+1}{2}}\Gamma(\frac{s+t+1}{2}), \\
\zeta(|.|^{-t}, \hat{\phi}_0, 1-s) &= \pi^{-\frac{1-s-t}{2}}\Gamma(\frac{1-s-t}{2}), \\
\zeta(sgn(.)|.|^{-t}, \hat{\phi}_{sgn}, 1-s) &= i\pi^{-\frac{(1-s+t)+1}{2}}\Gamma(\frac{(1-s-t)+1}{2}).
\end{align}
Ce qui nous permet de calculer explicitement le facteur $\gamma(s,\pi)$ et de vérifier que ce sont bien des fonctions méromorphes :
\begin{align}
\gamma(s, |.|^t) = \frac{\pi^{-\frac{s+t}{2}}\Gamma(\frac{s+t}{2})}{\pi^{-\frac{1-s-t}{2}}\Gamma(\frac{1-s-t}{2})}, \\
\gamma(s, sgn(.)|.|^t) = -i\frac{\pi^{-\frac{s+t+1}{2}}\Gamma(\frac{s+t+1}{2})}{\pi^{-\frac{(1-s+t)+1}{2}}\Gamma(\frac{(1-s-t)+1}{2})}.
\end{align}

\subsection{$GL_n(\mathbb{R}), n > 1$}

Les propositions \ref{comp_ind1} et \ref{comp_ind2} admettent un analogue dans le cas réel. Elles traduisent la compatibilité du théorème vis à vis des sous-représentations et des représentations induites. Le théorème \ref{thm_reel} est alors une conséquence du
\begin{theoreme}[de la sous-représentation]
Soit $\pi$ une représentation unitaire irréductible de $GL_n(\mathbb{R})$. Alors il existe des caractères $\chi_1, ..., \chi_n$ de $GL_1(\mathbb{R})$ tels que $\pi$ soit une sous-représentation de $Ind_{P_0}^{GL_n(\mathbb{R})}(\chi_1 \boxtimes ... \boxtimes \chi_n)$, où $P_0$ est le sous-groupe des matrices triangulaires supérieures de $GL_n(\mathbb{R})$.
\end{theoreme}
\section{Fonctions zêta sur $GL_n(\mathbb{A})$}

Dans cette partie, on note $G = GL_n(\mathbb{Q})$, $G_\mathbb{A}=GL_n(\mathbb{A})$. On pose $K = O_n(\mathbb{R}) \times \prod_p GL_n(\mathbb{Z}_p)$, c'est un sous-groupe compact maximal de $G_\mathbb{A}$.

\subsection{Formes cuspidales}
On commence par donner la définition des formes automorphes (et cuspidales), on renvoie à \cite{bump} et \cite{goldfeld-hundley} pour plus de détails.

On fixe un caractère unitaire $\omega : \mathbb{A}^\times/\mathbb{Q}^\times \rightarrow \mathbb{S}^1$.

\begin{definition}
Une forme automorphe de caractère central $\omega$ est une fonction $\varphi : G_\mathbb{A} \rightarrow \mathbb{C}$ lisse et $G$-invariante qui vérifie de plus :
\begin{itemize}
\item $\varphi$ est $K$-finie à droite,
\item $\varphi$ est $Z(U(\mathfrak{g}))$-finie,
\item \begin{equation}
\varphi(zg) = \omega(z)\varphi(g) \quad \forall g \in G_\mathbb{A}, z \in \mathbb{A}^\times,
\end{equation}
\item $\varphi$ est à croissance modérée.
\end{itemize}

On note $\mathcal{A}(G_\mathbb{A}, \omega)$ l'espace des formes automorphes de caractère central $\omega$.
\end{definition}

On rajoute aussi une condition d'annulation dont on aura besoin pour la preuve de l'équation fonctionnelle. Ce qui donne la
\begin{definition}
Une forme cuspidale $\varphi$ de caractère central $\omega$ est une forme automorphe de caractère central $\omega$ qui vérifie de plus les conditions :
\begin{equation}
\int_{U \backslash U_\mathbb{A}} \varphi(ug) du = 0
\end{equation}
pour tout radical unipotent $U$ d'un sous-groupe parabolique propre de $G_\mathbb{A}$ et tout $g \in G_\mathbb{A}$.

On note $\mathcal{A}_0(G_\mathbb{A}, \omega)$ l'espace des formes cuspidales de caractère central $\omega$.
\end{definition}

L'espace de Schwartz de $M_n(\mathbb{A})$ est, par définition, $\mathcal{S}(M_n(\mathbb{A}))=\otimes_v^{'} \mathcal{S}(M_n(\mathbb{Q}_v)=\left\lbrace \phi = \otimes \phi_v, \phi_v \in \mathcal{S}(M_n(\mathbb{Q}_v)), \phi_v = \mathbbm{1}_{\mathbb{Z}_v} \text{sauf pour un nombre fini de v} \right\rbrace$.

Pour $\varphi \in \mathcal{A}_0(G_\mathbb{A}, \omega)$, $\phi \in \mathcal{S}(M_\mathbb{A})$ et $s \in \mathbb{C}$, on pose
\begin{equation}
\zeta(\varphi, \phi, s) = \int_{G_\mathbb{A}} \phi(g) \varphi(g) |\det g|_\mathbb{A}^s dg,
\end{equation}
où $dg = \otimes_v dg_v$ est une mesure de Haar sur $GL_n(\mathbb{A})$ et $|.|_\mathbb{A} = \prod_v |.|_v$ est la valeur absolue adélique.

Notons $G^0_\mathbb{A}=\left\lbrace g \in G_\mathbb{A}, |\det g|_\mathbb{A} = 1 \right\rbrace$. Comme $\mathbb{R}_{> 0} \subset \mathbb{A}^\times=Z(G_\mathbb{A})$, l'application $|\det|_\mathbb{A} : G_\mathbb{A} \rightarrow \mathbb{R}_{> 0}$ est surjective de noyau $G^0_\mathbb{A}$.

La factorisation $G_\mathbb{A} = \mathbb{R}_{> 0}G^0_\mathbb{A}$ permet d'obtenir que
\begin{align}
\zeta(\varphi, \phi, s) &= \int_0^\infty \int_{G^0_\mathbb{A}} \phi(tg) \omega(t) \varphi(g) t^{ns} dg \frac{dt}{t} \\
&= \int_0^\infty \int_{G \backslash G^0_\mathbb{A}} \sum_{x \in G}{\phi(txg)} \varphi(g) \omega(t) t^{ns} dg \frac{dt}{t}.
\end{align}

Comme dans la preuve de l'équation fonctionnelle de la fonction zêta de Riemann, on scinde l'intégrale en $1$ dans le but de faire apparaître une symétrie. Autrement dit,
\begin{equation}
\begin{split}
\zeta(\varphi, \phi, s) &= \int_0^1 \int_{G \backslash G^0_\mathbb{A}} \sum_{x \in G}{\phi(txg)} \varphi(g) \omega(t) t^{ns} dg \frac{dt}{t} \\
&+ \int_1^\infty \int_{G \backslash G^0_\mathbb{A}} \sum_{x \in G}{\phi(txg)} \varphi(g) \omega(t) t^{ns} dg \frac{dt}{t}.
\end{split}
\end{equation}

La seconde intégrale converge absolument pour tout $s \in \mathbb{C}$, c'est une fonction entière. Pour la première intégrale, on fait le changement de variable $t \mapsto t^{-1}$, ce qui donne
\begin{equation}
\int_1^\infty \int_{G \backslash G^0_\mathbb{A}} \sum_{x \in G}{\phi(t^{-1}xg)} \varphi(g) \omega^{-1}(t) t^{-ns} dg \frac{dt}{t}.
\end{equation}

On va maintenant utiliser la formule de Poisson sur $M_n(\mathbb{A})$, ce qui donne pour la fonction $x \mapsto \phi(t^{-1}xg)$ :
\begin{equation}
\sum_{x \in M_n(\mathbb{Q})} \phi(t^{-1}xg) = t^{n^2}\sum_{x \in M_n(\mathbb{Q})} \hat{\phi}(txg^{-1}),
\end{equation}
on se rappelle que $g \in G^0_\mathbb{A}$, donc $|\det g|_\mathbb{A}=1$. On scinde la somme selon le rang de la matrice et on obtient :
\begin{equation}
\begin{split}
\sum_{x \in G} \phi(t^{-1}xg) &= t^{n^2}\sum_{x \in G} \hat{\phi}(txg^{-1}) \\
&+ \sum_{r < n, rg(x)=r} \left( t^{n^2}\hat{\phi}(txg^{-1}) - \phi(t^{-1}xg)\right).
\end{split}
\end{equation}

La contribution de la dernière somme s'avèrera nulle. Ce qui nous permet d'en déduire la
\begin{proposition}
Si $\varphi \in \mathcal{A}_0(G_\mathbb{A}, \omega)$ et $\phi \in \mathcal{S}(M_n(\mathbb{A})$, la fonction $\zeta(\varphi, \phi, .)$ peut être prolongée en une fonction entière et vérifie l'équation fonctionnelle
\begin{equation}
\label{eqcusp}
\zeta(\varphi, \phi, s) = \zeta(\check{\varphi}, \hat{\phi}, n-s),
\end{equation}
où $\check{\varphi}(g)=\varphi(g^{-1})$.
\end{proposition}

\begin{proof}
Il suffit de prouver que la contribution dans la formule de Poisson des matrices de rang $r < n$ est effectivement nulle. On considère l'action de $G$ par translation à droite sur l'ensemble des matrices de rang $r$. Chaque orbite contient un représentant de la forme $\begin{pmatrix} 
* & 0 \\
* & 0 
\end{pmatrix}$, on note $X$ l'ensemble des matrices de cette forme. On pose $P$ le sous-groupe parabolique de $G$ des matrices de la forme $\begin{pmatrix} 
* & 0 \\
* & * 
\end{pmatrix}$ et $U$ son radical unipotent.

On réécrit la somme sur les matrices de rang $r$ grâce au système de représentant $X$,
\begin{equation}
\sum_{rg(x)=r}\phi(xg) = \sum_{\gamma \in P \backslash G} \sum_{x \in X} \phi(x \gamma g).
\end{equation}
On en déduit que la contribution des matrices de rang $r$ dans la seconde intégrale est
\begin{equation}
\int_{P \backslash G^0_\mathbb{A}} \sum_{x \in X}{\phi(t^{-1}x g)} \varphi(g) dg.
\end{equation}
De plus, on remarque que, $xu=x$, pour tout $x \in X$ et $u \in U_\mathbb{A}$. Ce qui nous permet de réécrire cette intégrale sous la forme
\begin{equation}
\int_{PU_\mathbb{A} \backslash G^0_\mathbb{A}} \sum_{x \in X}{\phi(t^{-1}xg)} \int_{U \backslash U_\mathbb{A}} \varphi(ug) du dg.
\end{equation}
Cette dernière intégrale s'annule, car $f$ est cuspidale. On montre de même de l'intégrale correspondant au terme en $\hat{\phi}$ sur les matrices de rang $r < n$ s'annule aussi. Ce qui nous donne, grâce à la formule de Poisson et le raisonnement précédent, la formule
\begin{equation}
\begin{split}
\zeta(\varphi, \phi, s) &= \int_1^\infty \int_{G \backslash G^0_\mathbb{A}} \sum_{x \in G}{\hat{\phi}(txg^{-1})} \varphi(g) \omega^{-1}(t) t^{n(n-s)} dg \frac{dt}{t} \\
&+ \int_1^\infty \int_{G \backslash G^0_\mathbb{A}} \sum_{x \in G}{\phi(txg)} \varphi(g) \omega(t) t^{ns} dg \frac{dt}{t},
\end{split}
\end{equation}
ce qui démontre l'équation fonctionnelle en effectuant le changement de variable $g \mapsto g^{-1}$ dans la première intégrale.
\end{proof}

\subsection{Représentations automorphes}

L'espace des formes cuspidales $\mathcal{A}_0(G_\mathbb{A}, \omega)$ est stable par l'action de $U(\mathfrak{g})$ par opérateurs différentiels et par translation à droite de $O_n(\mathbb{R})$ et $GL_n(\mathbb{A}_f)$, c'est un $(\mathfrak{g}, O_n(\mathbb{R})) \times GL_n(\mathbb{A}_f)$-module.

Un coefficient $f$ de $\mathcal{A}_0(G_\mathbb{A}, \omega)$ est de la forme
\begin{equation}
f(g) = <\pi(g)\varphi, \tilde{\varphi}> = \int_{\mathbb{A}^\times G \backslash G_\mathbb{A}} \varphi(hg) \tilde{\varphi}(h) dh,
\end{equation}
où $\varphi \in \mathcal{A}_0(G_\mathbb{A}, \omega)$ et $\tilde{\varphi} \in \mathcal{A}_0(G_\mathbb{A}, \omega^{-1})$.

Pour un coefficient $f$ de $\mathcal{A}_0(G_\mathbb{A}, \omega)$, $\phi \in \mathcal{S}(M_\mathbb{A})$ et $s \in \mathbb{C}$, on pose
\begin{equation}
\zeta(f, \phi, s) = \int_{G_\mathbb{A}} \phi(g) f(g) |\det g|_\mathbb{A}^s dg.
\end{equation}

On peut déduire les propriétés de cette fonction zêta grâce à ce que l'on vient de faire pour les formes cuspidales. Plus précisément, on a
\begin{align}
\zeta(f, \phi, s) &= \int_{G_\mathbb{A}}\phi(g)\int_{\mathbb{A}^\times G \backslash G_\mathbb{A}} \varphi(hg) \tilde{\varphi}(h) dh |\det g|_\mathbb{A}^s dg \\
&= \int_{\mathbb{A}^\times G \backslash G_\mathbb{A}} \tilde{\varphi}(h) \int_{G_\mathbb{A}}\phi(h^{-1}g)\varphi(g)|\det g|_\mathbb{A}^s dg |\det h|_\mathbb{A}^{-s} dh \\
&= \int_{\mathbb{A}^\times G \backslash G_\mathbb{A}} \tilde{\varphi}(h) \zeta(\varphi, \phi(h^{-1}.), s)|\det h|_\mathbb{A}^{-s} dh,
\end{align}
où la deuxième égalité s'obtient grâce au changement de variable $g \mapsto h^{-1}g$. Ceci nous permet de démontrer la
\begin{proposition}
Si $f$ est un coefficient de $\mathcal{A}_0(G_\mathbb{A}, \omega)$ et $\phi \in \mathcal{S}(M_\mathbb{A})$, la fonction $\zeta(f, \phi, .)$ peut être prolongée en une fonction entière et vérifie l'équation fonctionnelle
\begin{equation}
\zeta(f, \phi, s) = \zeta(\check{f}, \hat{\phi}, n-s),
\end{equation}
où $\check{f}(g) = f(g^{-1})$.
\end{proposition}

\begin{proof}
On utilise l'équation fonctionnelle (\ref{eqcusp}) et le fait que la transformée de Fourier de $\phi(h^{-1}.)$ est $|\det h|_\mathbb{A}^n\hat{\phi}(.h)$,
\begin{align}
\zeta(f, \phi, s) &= \int_{\mathbb{A}^\times G \backslash G_\mathbb{A}} \tilde{\varphi}(h) \zeta(\check{\varphi}, \hat{\phi}(.h), n-s)|\det h|_\mathbb{A}^{n-s} dh \\
&= \int_{\mathbb{A}^\times G \backslash G_\mathbb{A}} \tilde{\varphi}(h) \int_{G_\mathbb{A}}\hat{\phi}(gh)\varphi(g^{-1})|\det g|_\mathbb{A}^{n-s} dg |\det h|_\mathbb{A}^{n-s} dh.
\end{align}
On effectue maintenant le changement de variable $g \mapsto gh^{-1}$, ce qui donne
\begin{equation}
\int_{\mathbb{A}^\times G \backslash G_\mathbb{A}} \tilde{\varphi}(h) \int_{G_\mathbb{A}}\hat{\phi}(g)\varphi(hg^{-1})|\det g|_\mathbb{A}^{n-s} dg dh,
\end{equation}
qui est bien $\zeta(\check{f}, \hat{\phi}, n-s)$.
\end{proof}

Si l'on combine cette proposition avec les résultats locaux, on peut construire la fonction $L$ attachée à une représentation cuspidale irréductible.
\begin{definition}
Une représentation cuspidale est un $(\mathfrak{g}, O_n(\mathbb{R})) \times GL_n(\mathbb{A}_f)$-module qui est isomorphe à un sous-quotient de $\mathcal{A}_0(G_\mathbb{A}, \omega)$.
\end{definition}

Plus précisément, on montre le
\begin{theoreme}
Soit $\pi$ une représentation cuspidale irréductible.

Le produit $L(s, \pi) = \prod_v L(s, \pi_v)$, qui est défini pour $Re(s) > n$, se prolonge en une fonction entière. De plus, $L(s, \pi)$ vérifie l'équation fonctionnelle
\begin{equation}
L(s,\pi) = \epsilon(s,\pi)L(1-s,\tilde{\pi}),
\end{equation}
où $\epsilon(s,\pi) = \prod_v \epsilon(s, \pi_v)$.
\end{theoreme}

\begin{proof}
La représentation $\pi$ se décompose en facteurs locaux,
$\pi \simeq \otimes_v^{'} \pi_v$, où $\pi_v$ est une représentation admissible irréductible de $GL_n(\mathbb{Q}_v)$ (un $(\mathfrak{g}, O_n(\mathbb{R}))$-module irréductible pour la place archimédienne) et pour presque toutes les places $\pi_v$ est sphérique (contient la représentation unité de $GL_n(\mathbb{Z}_v)$).

D'après les résultats locaux, pour chaque place $v$, il existe un nombre fini $(\phi_{\alpha_v})_{\alpha_v \in I_v}$ d'éléments de $\mathcal{S}(M_v)$ et de coefficient $(f_{\alpha_v})_{\alpha_v \in I_v}$ de $\pi_v$ tel que
\begin{equation}
\sum_{\alpha_v \in I_v} \zeta(f_{\alpha_v}, \phi_{\alpha_v}, s + \frac{1}{2}(n-1)) = L(s, \pi_v).
\end{equation}
De plus, d'après l'équation fonctionnelle locale
\begin{equation}
\sum_{\alpha_v \in I_v} \zeta(\check{f}_{\alpha_v}, \hat{\phi}_{\alpha_v}, 1-s + \frac{1}{2}(n-1)) = \epsilon(s,\pi_v)L(1-s, \tilde{\pi}_v).
\end{equation}

Notons $I = \prod_v I_v$. Pour presque toutes les places $v$, $\pi_v$ est sphérique, $I_v$ est un singleton; donc $I$ est fini.

Pour $\alpha = (\alpha_v) \in I$, on pose
\begin{equation}
\phi_\alpha = \prod_v \phi_{\alpha_v}, \quad f_\alpha = \prod_v f_{\alpha_v}.
\end{equation}
Alors $\phi_\alpha \in \mathcal{S}(M_\mathbb{A})$ et $f_\alpha$ est un coefficient de $\pi$ qui est un sous-quotient de $\mathcal{A}_0(G_\mathbb{A}, \omega)$. De plus,
\begin{equation}
\zeta(f_\alpha, \phi_\alpha, s) = \prod_v \zeta(f_{\alpha_v}, \phi_{\alpha_v}, s).
\end{equation}

On en déduit que
\begin{align}
L(s, \pi) &= \prod_v L(s, \pi_v) = \prod_v \sum_{\alpha_v \in I_v} \zeta(f_{\alpha_v}, \phi_{\alpha_v}, s + \frac{1}{2}(n-1)) \\
&= \sum_{\alpha \in I} \zeta(f_\alpha, \phi_\alpha, s + \frac{1}{2}(n-1))
\end{align}
est une somme finie de fonction zêta, qui chacune se prolonge en une fonction entière. De plus,
\begin{align}
L(s, \pi) &= \sum_{\alpha \in I} \zeta(f_\alpha, \phi_\alpha, s + \frac{1}{2}(n-1)) \\
&= \sum_{\alpha \in I} \zeta(\check{f}_\alpha, \hat{\phi}_\alpha, 1 - s + \frac{1}{2}(n-1)) \\
&= \prod_v \sum_{\alpha_v \in I_v} \zeta(\check{f}_{\alpha_v}, \hat{\phi}_{\alpha_v}, 1-s + \frac{1}{2}(n-1)) \\
&= \prod_v \epsilon(s, \pi_v) L(1-s, \tilde{\pi}_v) \\
&= \epsilon(s, \pi)L(1-s, \tilde{\pi}).
\end{align}
\end{proof}

\bibliographystyle{siam}
\bibliography{zeta}
 
\end{document}