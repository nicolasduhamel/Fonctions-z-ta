\documentclass{amsart}

\usepackage[utf8]{inputenc}
\usepackage[T1]{fontenc}
\usepackage[francais]{babel}
\usepackage{bbm}
\usepackage{amssymb}
\usepackage{amsmath}
\usepackage{amsthm}
\usepackage{mathtools}
\usepackage{hyperref}
\usepackage{graphics}
\usepackage{enumerate}

\usepackage{eulervm}

\newtheorem{proposition}{Proposition}
\newtheorem{propriete}{Propriété}
\newtheorem{definition}{Définition}
\newtheorem{theoreme}{Théorème}
\newtheorem{lemme}{Lemme}

\DeclareMathOperator{\Hom}{\mathnormal{Hom}}
\DeclareMathOperator{\Ext}{\mathnormal{Ext}}

\begin{document}

\title{Fonctions zêta sur $GL_n$}
\maketitle

Ce mémoire est consacré à la théorie de Godement-Jacquet \cite{godement-jacquet} des fonctions zêtas, qui est une généralisation des résultats de Tate \cite{tate} sur $GL_1$.

Dans la section 1, on rappelle l'essentiel des résultats sur $GL_1$, les méthodes de cette section seront ensuite généralisées à $GL_n$. Dans la section 2, on présente la théorie locale de Godement-Jacquet; on se ramène au cas d'une représentation supercuspidale et on utilise le fait que les coefficients sont à support compact modulo le centre. Dans la section 3, on présente une seconde preuve du théorème \ref{thm_padique}, la preuve consiste en un dévissage de l'espace de Schwartz suivant une suggestion de Raphaël Beuzart-Plessis. Dans la section 4, on explique les modifications nécessaires dans le cas archimédien. Pour finir, la section 5 est consacré à la théorie globale de Godement-Jacquet avec la définition de la fonction $L$ attachée à une représentation cuspidale et les premières propriétés de cette fonction $L$ (holomorphie, équation fonctionnelle).

Ce mémoire a été effectué sous la direction de Raphaël Beuzart-Plessis.

\tableofcontents

\section{Fonctions zêta sur $GL_1(\mathbb{Q}_p)$}
\section{Fonctions zêta sur $GL_n(\mathbb{Q}_p), n > 1$}

\label{gln}
Cette section présente la théorie locale de Godement-Jacquet \cite{godement-jacquet}. On montre que le théorème principal est compatible à l'induction et aux sous-représentations, ce qui nous permet de nous ramener au cas des représentations supercuspidales. Pour finir, on calcule les fonctions $L$ attachées aux représentations sphériques (qui sont utiles pour la théorie globale) et aux représentations de carré intégrable.

Dans la suite, on notera $G = GL_n(\mathbb{Q}_p)$, $dg$ une mesure de Haar sur $G$ et $(\pi, V)$ une représentation admissible irréductible de $G$. On pose $K=GL_n(\mathbb{Z}_p)$, c'est un sous-groupe compact maximal de $G$.

\begin{definition}
Une représentation $\pi : G \rightarrow GL(V)$ sur un $\mathbb{C}$-espace vectoriel $V$ est dite admissible si elle vérifie :
\begin{itemize}
\item Pour tout $v \in V$, le stabilisateur de $v$ dans $G$, $\left\lbrace g \in G, \pi(g)v = v \right\rbrace$, est un sous-groupe ouvert de $G$,
\item Pour tout sous-groupe ouvert $H$ de $G$, le sous-espace
\begin{equation*}
V^H=\left\lbrace v \in V, \pi(h)v = v, \forall h \in H \right\rbrace
\end{equation*}
des vecteurs stable par $H$ est de dimension fini.
\end{itemize}
\end{definition}

\begin{definition}
On note $\tilde{V}$ l'espace des formes linéaires $l : V \rightarrow \mathbb{C}$ telles qu'il existe un sous-groupe ouvert $H$ de $G$ vérifiant $l(\pi(h^{-1})v)=l(v)$ pour tout $h \in H$. On définit alors la contragrédiente $(\tilde{\pi}, \tilde{V})$ de $(\pi, V)$ par
\begin{equation}
(\tilde{\pi}(g)l)(v) = l(\pi(g^{-1}v)),
\end{equation}
pour tout $g \in G, l \in \tilde{V}$ et $v \in V$.
\end{definition}

Les coefficients de $\pi$ sont les fonctions de la forme $g \in G \mapsto <\pi(g)v, \tilde{v}>$, où $v \in V$ et $\tilde{v} \in \tilde{V}$. Alors $\check{f}(g)=f(g^{-1})=<v, \tilde{\pi}(g)\tilde{v}>$ est un coefficient de $\tilde{\pi}$.

On note $M_n$ l'ensemble des matrices $n \times n$ à coefficients dans $\mathbb{Q}_p$ et $\mathcal{S}$ l'ensemble des fonctions $\phi : M_n \rightarrow \mathbb{C}$ localement constantes à support compact.

Si $f$ est un coefficient de $\pi$, $\phi \in \mathcal{S}$ et $s \in \mathbb{C}$, on pose
\begin{equation}
\zeta(f, \phi, s) = \int_G \phi(g)f(g)|\det g|_p^s dg.
\end{equation}

On fixe un caractère non trivial $\psi$ de $\mathbb{Q}_p$ et on pose
\begin{equation}
\hat{\phi}(y) = \int_{M_n} \phi(x) \psi(Tr(xy)) dx,
\end{equation}
où $dx$ est une mesure de Haar sur $M_n$, normalisée telle que $\hat{\hat{\phi}}(x)=\phi(-x)$.

L'objectif de cette section est de montrer le
\begin{theoreme}
\label{thm_padique}
\begin{enumerate}
\item Il existe $s_0 \in \mathbb{R}$ tel que pour tout $s \in \mathbb{C}$ vérifiant $Re (s) > s_0$, $\phi \in \mathcal{S}$ et $f$ un coefficient de $\pi$, les intégrales
\begin{align}
\zeta(f, \phi, s) &= \int_G \phi(g)f(g)|\det g|_p^s dg \\
\zeta(\check{f}, \phi, s) &= \int_G \phi(g)\check{f}(g)|\det g|_p^s dg
\end{align}
convergent absolument.
\item Ces intégrales sont des fonctions rationnelles en $p^{-s}$. Plus précisément, il existe des polynômes $Q$ et $\tilde{Q}$ indépendants de $f$ et $\phi$ avec $Q(0)\neq 0$ (respectivement $\tilde{Q}(0)\neq 0$) et des polynômes $\Xi(f, \phi, s)$, $\tilde{\Xi}(\check{f}, \phi, s)$ en $p^{s}$ et $p^{-s}$ tels que
\begin{align}
\zeta(f, \phi, s+\frac{1}{2}(n-1)) &= \frac{\Xi(f, \phi, s)}{Q(p^{-s})}, \\
\zeta(\check{f}, \phi, s+\frac{1}{2}(n-1)) &= \frac{\tilde{\Xi}(\check{f}, \phi, s)}{\tilde{Q}(p^{-s})},
\end{align}
pour tous $s \in \mathbb{C}$, $\phi \in \mathcal{S}$ et $f$ coefficient de $\pi$. Les polynômes $Q$ et $\tilde{Q}$ sont choisis de degré minimal.
\item On peut choisir un nombre fini, de coefficients $f_i$ de $\pi$ (respectivement $\tilde{\pi}$) et de fonctions $\phi_i \in \mathcal{S}$, telles que $\sum_i \Xi(f_i, \phi_i, s)$ (respectivement $\sum_i \tilde{\Xi}(f_i, \phi_i, s)$ soit une constante non nulle.
\item Il existe une fonction $\epsilon(s, \pi, \psi)$, qui est à une constante près une puissance de $p^{-s}$, telle que
\begin{equation}
\label{epsilon}
\tilde{\Xi}(\check{f}, \hat{\phi}, 1-s) = \epsilon(s, \pi, \psi)\Xi(f, \phi, s),
\end{equation}
pour tous $s\in \mathbb{C}$, $\phi \in \mathcal{S}$ et $f$ coefficient de $\pi$.
\end{enumerate}
\end{theoreme}

Les conditions (2) et (3) caractérisent (à constante près) les polynômes $Q$ et $\tilde{Q}$. On normalise $Q$ et $\tilde{Q}$ tel que $Q(0)=\tilde{Q}(0)=1$, on pose alors
\begin{equation}
L(s, \pi) = \frac{1}{Q(p^{-s})}, \quad L(s, \tilde{\pi}) = \frac{1}{\tilde{Q}(p^{-s})}.
\end{equation}

L'existence de la fonction $\epsilon(s, \pi, \psi)$ est équivalente à l'existence d'une fonction méromorphe $\gamma(s,\pi,\psi)$ telle que
\begin{equation}
\zeta(\check{f}, \hat{\phi}, 1-s+\frac{1}{2}(n-1))=\gamma(s, \pi, \psi)\zeta(f, \phi, s),
\end{equation}
pour tout $\phi \in \mathcal{S}$ et $f$ coefficient de $\pi$. Ces deux fonctions étant reliées par la relation
\begin{equation}
\label{gammaepsilon}
\epsilon(s,\pi,\psi)=\gamma(s,\pi,\psi)\frac{L(s,\pi)}{L(1-s,\tilde{\pi})}.
\end{equation}
En effet, supposons l'existence de $\gamma(s,\pi,\psi)$ alors $\epsilon(s,\pi,\psi)$ vérifie 
\begin{equation}
\tilde{\Xi}(\check{f}, \hat{\phi}, 1-s) = \epsilon(s, \pi, \psi)\Xi(f, \phi, s).
\end{equation}
On a de plus une égalité similaire avec $\epsilon(s,\tilde{\pi},\psi)$,
\begin{equation}
\Xi(f, s, \hat{\hat{\phi}}, s)=\epsilon(1-s, \tilde{\pi}, \psi)\tilde{\Xi}(\check{f}, \hat{\phi}, 1-s).
\end{equation}
Il ne nous reste plus qu'à utiliser la formule $\hat{\hat{\phi}}(x)=\phi(-x)$ pour obtenir la relation
\begin{equation}
\epsilon(s, \pi, \psi)\epsilon(1-s, \tilde{\pi}, \psi)=\omega(-1),
\end{equation}
où $\omega$ est le caractère de $\mathbb{Q}_p^\times$ tel que $\pi(z)=\omega(z)1$ pour $z\in \mathbb{Q}_p^\times$. D'après (2) et (3) du théorème, $\epsilon(s, \pi, \psi)$ est alors un polynôme en $p^s$ et $p^{-s}$, on en déduit que $\epsilon(s, \pi, \psi)$ est une puissance de $p^{-s}$ à constante près.

\subsection{Réduction au cas supercuspidal}

Si $\pi$ est une représentation admissible (non nécessairement irréductible) de $G$, les assertions du théorème font sens pour $\pi$ et $\tilde{\pi}$, mais peuvent être fausse si $\pi$ n'est pas irréductible.

Supposons le théorème vrai pour $\pi$ et $\tilde{\pi}$. Soit $\sigma$ une sous-représentation irréductible de $\pi$. Alors les coefficients de $\sigma$ sont de la forme $<\pi(g)v,\tilde{v}>$ avec $v\in V$ et $\tilde{v} \in \tilde{V}$. Cependant, toutes ces fonctions ne sont pas des coefficients de $\sigma$. On en déduit la
\begin{proposition}
\label{comp_ind1}
Le théorème est vrai pour $\sigma$ et il existe des polynômes $R$ et $\tilde{R}$ en $p^{-s}$ tels que
\begin{align}
L(s,\sigma)&=R(p^{-s})L(s,\pi), \\
L(s,\tilde{\sigma})&=\tilde{R}(p^{-s})L(s,\tilde{\pi}).
\end{align}
De plus,
\begin{equation}
\gamma(s,\sigma,\psi)=\gamma(s,\pi,\psi).
\end{equation}
\end{proposition}

Soit $P$ un sous-groupe parabolique propre maximal de $G$ et $U$ son radical unipotent alors $P/U \simeq G' \times G''$, où l'on note $G'=GL_{n'}(\mathbb{Q}_p)$ et $G''=GL_{n''}(\mathbb{Q}_p)$.

Soit $\sigma'$ (respectivement $\sigma''$) une représentation admissible de $G'$ (respectivement $G''$). On ne les suppose pas irréductible, on suppose cependant qu'ils admettent des caractères centraux $\omega'$ et $\omega''$. Alors $\sigma' \boxtimes \sigma''$ est naturellement une représentation de $P/U$, donc une représentation de $P$ triviale sur $U$.
\begin{proposition}
\label{comp_ind2}
Notons $\pi = Ind_P^G(\sigma' \boxtimes \sigma'')$. Supposons le théorème vrai pour $\sigma'$ et $\sigma''$. Alors le théorème est vrai pour $\pi$. De plus, on a
\begin{align}
L(s,\pi)&=L(s,\sigma')L(s,\sigma''), \\
L(s,\tilde{\pi})&=L(s,\tilde{\sigma}')L(s,\tilde{\sigma}''), \\
\epsilon(s,\pi,\psi)&=\epsilon(s,\sigma',\psi)\epsilon(s,\sigma'',\psi).
\end{align}
\end{proposition}

\begin{proof}
On notera $M'=M_{n'}(\mathbb{Q}_p)$ et $M''=M_{n''}(\mathbb{Q}_p)$. Soit $f$ un coefficient de $\pi$, $\phi \in \mathcal{S}$ et $s \in \mathbb{C}$.

L'espace vectoriel $V$ sur lequel $\pi$ agit est l'espace des fonctions $v : G \rightarrow W$ localement constante qui vérifient
\begin{equation}
v(pg)=\delta_P^{\frac{1}{2}}(p)(\sigma' \boxtimes \sigma'')(p)v(g),
\end{equation}
où $\delta_P$ est le caractère modulaire de $P$ et $W$ est l'espace vectoriel sur lequel $\sigma' \boxtimes \sigma''$ agit.

Le coefficient $f$ est alors de la forme
\begin{align}
f(g)&=<\pi(g)v,\tilde{v}> \\
&= \int_K <v(kg),\tilde{v}(k)>_W dk.
\end{align}

Posons $t=s+\frac{1}{2}(n-1)$, $t'=s+\frac{1}{2}(n'-1)$ et $t''=s+\frac{1}{2}(n''-1)$. L'intégrale zêta est donc
\begin{equation}
\zeta(f,\phi,t)=\int_G \phi(g)|\det g|_p^t \int_K <v(kg),\tilde{v}(k)>dk dg.
\end{equation}
On échange l'ordre d'intégration et on fait le changement de variables $g \mapsto k^{-1}g$, on obtient
\begin{equation}
\label{integrale1}
\int_K \int_G \phi(k^{-1}g)|\det g|^t<v(g),\tilde{v}(k)>dg dk.
\end{equation}
On utilise la décomposition de Cartan pour écrire $g \in G$ sous la forme $g = \begin{pmatrix} 
g' & u \\
0 & g'' 
\end{pmatrix} k'$, où $g' \in G'$, $g'' \in G''$, $u \in U$ et $k' \in K$. On peut alors décomposer la mesure de Haar de $G$ en fonction des mesures de Haar de $G'$, $G''$, $U$ et $K$. En effet,
\begin{equation}
dg = |\det g'|^{-n''}dg'dg''dudk'.
\end{equation}
L'expression (\ref{integrale1}) devient
\begin{equation}
\label{integrale2}
\begin{split}
\int_K \int_{G' \times G'' \times U \times K} &\phi(k^{-1}\begin{pmatrix} 
g' & u \\
0 & g'' 
\end{pmatrix} k') |\det g'|^{t'}|\det g''|^{t''} \\
&<(\sigma'(g') \boxtimes \sigma''(g''))v(k'), \tilde{v}(k)> dg' dg'' du dk' dk.
\end{split}
\end{equation}

Le facteur $<(\sigma'(g') \boxtimes \sigma''(g''))v(k'), \tilde{v}(k)>$ est un coefficient de $\sigma' \boxtimes \sigma''$, donc est une combinaison linéaire de produits de coefficients de $\sigma'$ et de coefficients de $\sigma''$ :
\begin{equation}
<(\sigma'(g') \boxtimes \sigma''(g''))v(k'), \tilde{v}(k)> = \sum_{i=1}^l \lambda_i(k,k')f_i'(g')f_i''(g''),
\end{equation}
où les fonctions $\lambda_i : K \times K \rightarrow \mathbb{C}$ sont localement constante et les $f_i'$ (respectivement $f_i''$) sont des coefficients de $\sigma'$ (respectivement $\sigma''$).

D'autre part, la fonction
\begin{equation}
(x' \in M', x'' \in M'') \mapsto \int_U \phi(k^{-1}\begin{pmatrix} 
x' & u \\
0 & x'' 
\end{pmatrix} k') du
\end{equation}
est un élément de l'espace de Schwartz $\mathcal{S}(M' \times M'')$. On peut donc l'écrire sous la forme
\begin{equation}
\label{u_integrale}
\int_U \phi(k^{-1}\begin{pmatrix} 
x' & u \\
0 & x'' 
\end{pmatrix} k') du = \sum_{j=1}^{l'} \mu_j(k, k')\phi_j'(x')\phi_j''(x''),
\end{equation}
où les $\mu_j$ sont localement constantes et $\phi_j' \in \mathcal{S}(M')$ (respectivement $\phi_j'' \in \mathcal{S}(M'')$).

En remplaçant ces expressions dans l'intégrale (\ref{integrale2}), on trouve
\begin{equation}
\label{zetainduite}
\zeta(f, \phi, t) = \sum_{i,j=1}^{l,l'} \int_{K \times K} \lambda_i(k,k')\mu_j(k,k') dk dk' \zeta(f_i',\phi_j',t') \zeta(f_i'',\phi_j'',t'').
\end{equation}

D'après les hypothèses faites sur $\sigma'$ et $\sigma''$, les intégrales définissant les $\zeta(f_i',\phi_j',t')$ (respectivement $\zeta(f_i'',\phi_j'',t'')$) sont absolument convergentes pour $Re(s)$ assez grande. Ce qui justifie à posteriori les calculs que l'on vient de faire et prouve la partie (1) du théorème pour $\pi$.

D'après (\ref{zetainduite}) et les hypothèses faites sur $\sigma'$ et $\sigma''$, on obtient la relation
\begin{equation}
\zeta(f,\phi,s+\frac{1}{2}(n-1))=\sum_{i,j=1}^{l,l'}c_{i,j}\Xi(f_i',\phi_j',s)L(s,\sigma')\Xi(f_i'',\phi_j'',s)L(s,\sigma'').
\end{equation}
Ce qui prouve la partie (2) du théorème pour $\pi$.

Passons à la partie (4) du théorème. La valeur $\zeta(\check{f}, \hat{\phi}, t)$ s'obtient en remplaçant $f$ par $\check{f}$, ce qui remplace les $f_i'$ et $f''_i$ en $\check{f}_i'$ et $\check{f}_i''$, et $\phi$ en $\hat{\phi}$. Voyons maintenant comment ce dernier changement affecte l'intégrale. Montrons que l'équation (\ref{u_integrale}) se transforme en
\begin{equation}
\int_U \hat{\phi}(k'^{-1}\begin{pmatrix} 
x' & u \\
0 & x'' 
\end{pmatrix} k) du = \sum_{j=1}^{l'} \mu_j(k, k')\hat{\phi}_j'(x')\hat{\phi}_j''(x'').
\end{equation}
En effet, 
\begin{align}
\int_U \hat{\phi}(k'^{-1}\begin{pmatrix} 
x' & u \\
0 & x'' 
\end{pmatrix} k) du &= \int_U \int_{M_n} \phi(k^{-1}xk')\psi(Tr(\begin{pmatrix} 
x_1 & x_2 \\
x_3 & x_4 
\end{pmatrix}\begin{pmatrix} 
x' & u \\
0 & x'' 
\end{pmatrix})dxdu \\
&= \int \phi(k^{-1}\begin{pmatrix} 
x_1 & x_2 \\
0 & x_4 
\end{pmatrix}k')\psi(x_1x'+x_4x'')dx_1dx_2dx_4 \\
&= \sum_{j=1}^{l'} \mu_j(k, k')\hat{\phi}_j'(x')\hat{\phi}_j''(x'').
\end{align}
La première égalité s'obtient en considérant la transformée de Fourier en les variables $(x_3, u)$. La dernière s'obtient en appliquant la transformée de Fourier sur $M'\times M''$ à l'équation (\ref{u_integrale}).

Ces considérations nous donnent une égalité similaire à (\ref{zetainduite}),
\begin{equation}
\zeta(\check{f},\hat{\phi},1-s+\frac{1}{2}(n-1))=\sum_{i,j=1}^{l,l'}c_{i,j}\Xi(\check{f}_i',\hat{\phi}_j',1-s)L(1-s,\tilde{\sigma}')\Xi(\check{f}_i'',\hat{\phi}_j'',1-s)L(1-s,\tilde{\sigma}'').
\end{equation}
On obtient ainsi l'équation fonctionnelle
\begin{equation}
\tilde{\Xi}(\check{f}, \hat{\phi}, 1-s)=\epsilon(s, \sigma', \psi)\epsilon(s,\sigma'',\psi)\Xi(f,\phi,s),
\end{equation}
on en déduit que $\epsilon(s,\pi,\psi)=\epsilon(s, \sigma', \psi)\epsilon(s,\sigma'',\psi)$ et la partie (4) du théorème pour $\pi$.

Il ne reste plus qu'à prouver la partie (3). Il suffit de montrer que si l'on fixe $\phi' \in \mathcal{S}(M')$, $\phi'' \in \mathcal{S}(M'')$ et $f'$ (respectivement $f''$) coefficient de $\sigma'$ (respectivement $\sigma''$) alors il existe $\phi \in \mathcal{S}(M)$ et $f$ coefficient de $\pi$ tel que
\begin{equation}
\zeta(f,\phi,t)=\zeta(f',\phi',t')\zeta(f'',\phi'',t'').
\end{equation}
En effet, le calcul du produit des fonctions zêta $\zeta(f',\phi',t')\zeta(f'',\phi'',t'')$ donne
\begin{equation}
\label{zetaprod}
\int_{G' \times G''} \phi'(g')\phi''(g'')f'(g')f''(g'')|\det g'|_p^{t'}|\det g''|_p^{t''}dg'dg''.
\end{equation}
On choisit alors $\phi \in \mathcal{S}(M)$ de la forme $\begin{pmatrix} 
x' & u \\
v & x'' 
\end{pmatrix} \mapsto \phi'(x')\phi''(x'')\phi_0(u)\phi_1(v)$, où $\phi_1 \in \mathcal{S}(M_{n'',n'})$ vérifie $\phi_1(0)=1$ et $\phi_0 \in \mathcal{S}(M_{n',n''})$ est d'intégrale $1$. Avec ce choix, on a
\begin{equation}
\int_U \phi(\begin{pmatrix} 
g' & u \\
0 & g'' 
\end{pmatrix})=\phi'(g')\phi''(g'').
\end{equation}
De plus, il existe une fonction localement constante $\eta : K \rightarrow \mathbb{C}$ telle que
\begin{equation}
\int_{U \times K} \phi(\begin{pmatrix} 
g' & u \\
0 & g'' 
\end{pmatrix}k)\eta(k)dudk = \phi(g')\phi(g'').
\end{equation}
On pose aussi $f(g)=\delta_P^{\frac{1}{2}}(\begin{pmatrix} 
g' & u \\
0 & g'' 
\end{pmatrix})\eta(k)f'(g')f''(g'')$, alors $f$ est bien un coefficient de $\pi$. De plus, en intégrant sur $U \times K$ l'expression (\ref{zetaprod}) devient
\begin{equation}
\int_G \phi(\begin{pmatrix} 
g' & u \\
0 & g'' 
\end{pmatrix}k)f(g)|\deg g|_p^t|\det g'|_p^{-n''}dg'dg''dudk,
\end{equation}
qui est bien $\zeta(f,\phi,t)$. Ce qui termine la preuve de la proposition.
\end{proof}
\subsection{Représentation supercuspidale}

Dans cette partie, on suppose que $\pi$ est une représentation supercuspidale irréductible de $G$. Avant d'aller plus loin, commençons par rappeler un résultat fondamental sur les représentations supercuspidales.

\begin{proposition}
Les coefficients de $\pi$ sont à support compact modulo $\mathbb{Q}_p^\times$.
\end{proposition}

Soit $f$ un coefficient de $\pi$ et $\phi \in \mathcal{S}$, alors il existe un sous-groupe compact $K'$ de $G$ tel que $f$ et $\phi$ sont invariant à gauche par $K'$. De plus, le support de $f$ est, d'après la proposition, à support compact modulo $\mathbb{Q}_p^\times$. Il existe donc un nombre fini d'éléments $(g_i)_{1 \leq i \leq N}$ de $G$ tel que
\begin{equation}
supp(f) \subset \cup_{i=1}^N K'\mathbb{Q}_p^\times g_i.
\end{equation}

On en déduit que
\begin{equation}
\zeta(f,\phi,s) = vol(K') \sum_{i=1}^N f(g_i)|\det g_i|_p^s \int_{\mathbb{Q}_p^\times} \phi(xg_i)|x|_p^{ns}\omega(x)dx,
\end{equation}
où $\omega$ est le caractère central de $\pi$. Cette dernière intégrale est absolument convergente pour $Re(s) > 0$. De plus, le quotient $\frac{\zeta(f,\phi,s)}{L(ns,\omega)}$ est un polynôme en $p^s$ et $p^{-s}$. Ce qui prouve les parties (1) et (2) du théorème pour $\pi$.

Posons $G^0=\left\lbrace g \in G, |\det g|_p = 1 \right\rbrace$, alors $G^0 \cap \mathbb{Q}_p^\times = \mathbb{Z}_p^\times$ est compact. On choisit $\phi \in \mathcal{S}$ tel que $\phi(g) = \overline{f(g)}$ si $g \in G^0$ et $\phi(g)=0$ sinon. Alors
\begin{equation}
\zeta(f, \phi, s) = \int_{G^0} f(g)\overline{f(g)} dg > 0
\end{equation}
est une constante non nulle, ce qui prouve la partie (3) du théorème pour $\pi$. Il ne nous reste plus qu'à montrer l'équation fonctionnelle.

Commençons par définir l'opérateur zêta,
\begin{equation}
\zeta(\pi, \phi, s) = \int_G \phi(g)|\det g|_p^s\pi(g) dg.
\end{equation}
C'est l'opérateur dont les coefficients sont exactement les $\zeta(f, \phi,s)$ pour $f$ coefficient de $\pi$.

Posons $\mathcal{S}_0=\left\lbrace \phi \in \mathcal{S}| supp(\phi), supp(\hat{\phi}) \subset G \right\rbrace$. Le résultat qui va nous permettre de prouver l'équation fonctionnelle est la
\begin{proposition}
\label{pre-eq-func}
Pour $\phi \in \mathcal{S}, \phi' \in \mathcal{S}_0$, on a
\begin{equation}
\zeta(\check{\pi}, \hat{\phi}', n-s)\zeta(\pi,\phi,s) = \zeta(\pi, \phi',s)\zeta(\check{\pi}, \hat{\phi}, n-s),
\end{equation}
où $\check{\pi}(g) = \pi(g^{-1})$.
\end{proposition}

\begin{proof}
La proposition est une conséquence immédiate du
\begin{lemme}
Soit $\phi \in \mathcal{S}, \phi' \in \mathcal{S}_0, v \in V$ et $\tilde{v} \in \tilde{V}$, pour $0 < Re(s) < n$, les intégrales
\begin{align}
&\int_G \int_G \phi(g)\hat{\phi}'(h)<\pi(g)v,\tilde{\pi}(h)\tilde{v}>|\det g|_p^s|\det h|_p^{n-s}dg dh,
&\int_G \int_G \hat{\phi}(g)\phi'(h)<\pi(g^{-1}v,\tilde{\pi}(h^{-1})\tilde{v}>|\det g|_p^{n-s}|\det h|_p^s dg dh,
\end{align}
sont absolument convergentes et coïncident. De plus, ces intégrales sont les coefficients des opérateurs $\zeta(\check{\pi}, \hat{\phi}', n-s)\zeta(\pi,\phi,s)$ et $\zeta(\pi, \phi',s)\zeta(\check{\pi}, \hat{\phi}, n-s)$.
\end{lemme}

\end{proof}

\begin{proposition}
Pour $s \in \mathbb{C}$, il existe un opérateur $\gamma(s) : V \rightarrow V$ tel que
\begin{equation}
\zeta(\check{\pi}, \hat{\phi}, n-s) = \gamma(s)\zeta(\pi,\phi,s), \forall \phi \in \mathcal{S}_0.
\end{equation}
De plus, l'opérateur $\gamma(s)$ est un scalaire.
\end{proposition}

\begin{proof}
Unicité : On choisit $\phi \in \mathcal{S}_0$ tel que $\zeta(\pi,\phi,s)=Id_V$, alors $\gamma(s)=\zeta(\check{\pi}, \hat{\phi}, n-s)$.

Existence : Il faut démontrer que les différents $\phi \in \mathcal{S}_0$ tel que $\zeta(\pi,\phi,s)=Id_V$ donnent un même opérateur $\zeta(\check{\pi},\hat{\phi}, n-s)$. Soit $\phi_1,\phi_2 \in \mathcal{S}_0$ tel que $\zeta(\pi,\phi_1,s)=\zeta(\pi,\phi_2,s)=Id_v$. D'après la proposition (\ref{pre-eq-func}), on en déduit que $\zeta(\check{\pi},\hat{\phi_1}, n-s)=\zeta(\check{\pi},\hat{\phi_2}, n-s)$.

On pose $\gamma(s)=\zeta(\check{\pi},\hat{\phi_0}, n-s)$ pour $\phi_0 \in \mathcal{S}_0$ tel que $\zeta(\pi,\phi,s)=Id_V$. Alors, d'après la proposition (\ref{pre-eq-func}),
\begin{align}
\gamma(s)\zeta(\pi,\phi,s) &= \zeta(\check{\pi},\hat{\phi_0}, n-s)\zeta(\pi,\phi,s) \\
&= \zeta(\pi,\phi_0,s)\zeta(\check{\pi}, \hat{\phi}, n-s) \\
&= \zeta(\check{\pi}, \hat{\phi},s),
\end{align}
pour tout $\phi \in \mathcal{S}$.

Montrons maintenant que $\gamma(s) \in \Hom_G(\pi,\pi)$, le lemme de Schur nous permet de conclure que $\gamma(s)$ est un scalaire.

Pour $\phi \in \mathcal{S}$, on pose $\phi_h = \phi(h.)$. Alors $\hat{\phi}_h = |\det h|_p^{-n}\hat{\phi}(.h^{-1})$. Ce qui nous permet d'obtenir
\begin{align}
\zeta(\check{\pi}, \hat{\phi}_h, n-s)&=|\det h|_p^{-n}\zeta(\check{\pi}, \hat{\phi}(.h^{-1}),n-s) \\
&= |\det h|_p^{-s}\pi(h^{-1})\zeta(\check{pi}, \hat{\phi}, n-s) \\
&= |\det h|_p^{-s}\pi(h^{-1})\gamma(s)\zeta(\pi,\phi,s).
\end{align}
D'autre part, on a
\begin{equation}
\gamma(s)\zeta(\pi,\phi_h,s)=|\det h|_p^{-s}\gamma(s)\pi(h^{-1})\zeta(\pi,\phi,s).
\end{equation}
Par unicité de l'opérateur $\gamma(s)$, on en déduit que $\pi(h^{-1})\gamma(s)=\gamma(s)\pi(h^{-1})$. Autrement dit, $\gamma(s) \in \Hom_G(\pi,\pi)$.
\end{proof}

\begin{lemme}
Soit $v \in V, \tilde{v} \in \tilde{V}$ et $s \in \mathbb{C}$, il existe $\phi \in \mathcal{S}_0$ tel que
\begin{equation}
\zeta(\pi,\phi,s)w = <w,\tilde{v}>v, \forall w \in V.
\end{equation}
En particulier, il existe $\phi \in \mathcal{S}_0$ tel que $\zeta(\pi, \phi, s) = Id_V.$
\end{lemme}

\begin{proof}
Soit $\phi \in \mathcal{S}_0$ tel que $\phi(g) = |\det g|_p^s<v, \tilde{\pi}\tilde{v}>$ si $|\det g|_p \in \left\lbrace 1, p, ..., p^{n-1} \right\rbrace$ et $\phi(g) = 0$ sinon. Alors
\begin{align}
<\zeta(\pi,\phi,s)w,\tilde{w}> &= \int_{G/\mathbb{Q}_p^\times} <v,\tilde{\pi}(g)\tilde{v}><\pi(g)w,\tilde{w}>dg \\
&= c <v, \tilde{w}><w,\tilde{v}>,
\end{align}
pour tout $w \in V, \tilde{w} \in \tilde{V}$. La dernière égalité est une conséquence du lemme de Schur. Ce qui montre que $\zeta(\pi, \phi,s)$ est proportionnel à $w \mapsto <w,\tilde{v}>v$.
\end{proof}

\begin{lemme}
Soit $v \in V$ non nul, alors
$$W=\left\lbrace u \in V, \exists \phi \in \mathcal{S}_0, c \neq 0, l \in \mathbb{Z}, \zeta(\pi, \phi, s) = cp^{-ls}u \forall s \in \mathbb{C} \right\rbrace$$
engendre $V$.
\end{lemme}

\begin{proof}
Si $u \in W$, alors $\pi(h)u$ l'est aussi pour $h \in G$. En effet,
$\zeta(\pi,\phi(.h^{-1}),s) = cp^{-ls}|\det h|_p^{-s}u$. Comme $V$ est irréductible, il suffit de montrer que $W \neq 0$.

Soit $\phi \in \mathcal{S}_0$ tel que $\phi(g) = <v, \pi(g)v>$ si $g \in G^0$ et $\phi(g) = 0$ sinon. Alors
\begin{equation}
u=\zeta(\pi, \phi, s)v = \int_{G^0} <v,\pi(g)v>\pi(g)v dg
\end{equation}
est indépendant de $s$ et non nul puisque
\begin{equation}
<\zeta(\pi,\phi,s)v,v> = \int_{G^0} |<v,\pi(g)v>|^2 dg > 0.
\end{equation}
Ce qui montre que $u \in W$ et $u$ non nul.
\end{proof}

Montrons que $\gamma(s)$ est non seulement une fraction rationnelle en $p^{-s}$, mais en fait une puissance de $p^s$. En effet, on a
\begin{equation}
\zeta(\check{f}, \hat{\phi}, n-s)=\gamma(s)\zeta(f,\phi,s), \forall \phi \in \mathcal{S}_0.
\end{equation}
D'après le lemme, on peut choisir $\phi \in \mathcal{S}_0$ et $f$ coefficient de $\pi$ tel que $\zeta(f,\phi,s)=p^{-ls}$. Alors $\gamma(s) = \zeta(\check{f}, \hat{\phi}, n-s)p^{ls}$ est un polynôme en $p^{-s}$ et $p^s$. En appliquant le lemme à $\check{\pi}$, on en déduit que $\gamma$ n'admet pas de $zéros$, c'est donc une puissance de $p^{s}$.

\begin{proposition}
Pour $\pi$ supercuspidale irréductible, on a $L(s,\pi)=1$.
\end{proposition}

\begin{proof}
Si $\omega$ est ramifié, alors $L(s,\omega)=1$. On en déduit que
$L(s,\pi)=\frac{L(s,\pi)}{L(ns,\omega)}$ est un polynôme en $p^{-s}$, donc $L(s, \pi)=1$.

Si $\omega$ est non ramifié, on peut supposer sans perte de généralité que $\omega=1$, alors
\begin{equation}
L(s,\omega)=\frac{1}{1-p^{-s}}, \quad L(ns,\omega)=\frac{1}{\prod_{\mu^n=1}(1-\mu p^{-s})}=\frac{1}{1-p^{-ns}}.
\end{equation}
Ce qui nous permet d'en déduire que
\begin{equation}
L(s,\pi) = \frac{1}{\prod_{\mu \in T}(1-\mu p^{-s})}, \quad L(s,\tilde{\pi}) = \frac{1}{\prod_{\mu \in T'}(1-\mu p^{-s})},
\end{equation}
où $T$ et $T'$ sont des sous-ensembles des racines $n$-ième de l'unité.

On vient de montrer précédemment que $\gamma$ est une puissance de $p^s$, il en est alors de même pour $\epsilon(s,\pi,\psi)$ et $\frac{L(s,\pi)}{L(1-s,\tilde{\pi})}$ d'après la relation (\ref{gammaepsilon}). Ce qui montre que la fraction
\begin{equation}
\frac{\prod_{\mu \in T'}(1-\mu p^{s-1})}{\prod_{\mu \in T}(1-\mu p^{-s})}
\end{equation}
est une puissance de $p^s$, d'où $L(s,\pi)=L(s,\tilde{\pi})=1$.
\end{proof}
\subsection{Représentation sphérique}

L'importance des représentations sphériques vient du fait que dans la théorie globale presque toutes les composantes locales sont des représentations sphériques.

\begin{definition}
Une représentation sphérique de $G$ est une représentation $(\pi, V)$ admissible irréductible de $G$ telle qu'il existe un vecteur $v \in V$ non nul invariant par $K$.
\end{definition}

Notons $P_0 \subset G$ l'ensemble des matrices triangulaires supérieures. Soient $\chi_1, ..., \chi_n$ des caractères non ramifiés de $\mathbb{Q}_p^\times$, on note $\chi = \chi_1 \boxtimes ... \boxtimes \chi_n$ le caractère de $P_0$ trivial sur son radical unipotent. On pose $\pi = Ind_{P_0}^G(\chi)$.

Alors $\pi$ contient un vecteur invariant par $K$, $\varphi_0$ défini par $\varphi_0(pk)=\delta_P(p)^{\frac{1}{2}}\chi(p)$. On considère le coefficient $f_0$ défini par $f_0(g)=<\pi(g)\varphi_0, \tilde{\varphi}_0>$, où $\tilde{\varphi}_0$ est le vecteur de $\tilde{\pi}$ invariant par $K$ défini par $\tilde{\varphi}_0(pk)=\delta_P(p)^{\frac{1}{2}}\chi(p)^{-1}$. C'est un coefficient d'une représentation admissible irréductible que l'on note $\pi_0$, cette représentation est sphérique.

De plus, toutes les représentations sphériques sont de cette forme (à isomorphisme près).

\begin{lemme}
\label{lemmespherique}
On note $\phi_0$ l'indicatrice de $M_n(\mathbb{Z}_p)$. Alors
\begin{equation}
\zeta(f_0, \phi_0, s+\frac{1}{2}(n-1)) = \prod_{i=1}^n L(s, \chi_i).
\end{equation}
\end{lemme}

\begin{proof}
Comme $\tilde{\varphi}_0$ est $K$-invariant, on en déduit que
\begin{equation}
f_0(g)=\int_K \varphi_0(kg)\tilde{\varphi}_0(k)dk=\int_K \varphi_0(kg) dk.
\end{equation}
De plus,
\begin{align}
\zeta(f_0, \phi_0, s) &= \int_G \int_K \phi_0(g)\varphi_0(kg)|\det g|^s dkdg \\
&= \int_G \int_K \phi_0(k^{-1}g)\varphi_0(g)|\det k^{-1}g|^s dkdg \\
&= \int_G \phi_0(g)\varphi_0(g)|\det g|^s dg \\
&= \int_{P_0} \int_K \phi_0(pk) \varphi_0(pk) |\det p|^s dp dk \\
&= \prod_{i=1}^n \int_{\mathbb{Z}_p}|a_i|^s\chi_i(a_i) |a_i|^{s+\frac{1}{2}(n-1)}da_i \\
&= \prod_{i=1}^n L(s-\frac{1}{2}(n-1), \chi_i).
\end{align}
\end{proof}

On en déduit le calcul de la fonction $L$ d'une représentation sphérique.
\begin{proposition}
En reprenant les notations précédentes,
\begin{equation}
L(s, \pi_0) = \prod_{i=1}^n L(s, \chi_i).
\end{equation}
\end{proposition}

\begin{proof}
En effet, $\pi_0$ est une sous-représentation de $\pi=Ind_{P_0}^G(\chi)$. D'après la proposition \ref{comp_ind1}, $L(s, \pi_0)$ est de la forme $R(p^{-s})L(s, \pi)$. De plus, d'après la proposition \ref{comp_ind2},
\begin{equation}
L(s, \pi) = \prod_{i=1}^n L(s, \chi_i).
\end{equation}
Il nous suffit donc de montrer que $R=1$. C'est le lemme \ref{lemmespherique}.
\end{proof}

\subsection{Représentation de carré intégrable}

Cette partie est consacrée extraite au calcul de la fonction $L$ d'une représentation de carré intégrable, elle est extraite de \cite{goldfeld-hundley}.

\begin{definition}
Une représentation admissible irréductible $(\pi, V)$ de $G$ est dite de carré intégrable si son caractère central est unitaire et $\int_{G/\mathbb{Q}_p^\times} |f(g)|^2 dg < \infty$ pour tout coefficient $f$ de $\pi$.
\end{definition}

On décrit maintenant la classification de Bernstein-Zelevinsky des représentations admissibles irréductibles de carré intégrable. Soit $r,d$ des entiers $> 0$, soit $\tau$ une représentation supercuspidale de $GL(r, \mathbb{Q}_p)$ de caractère central unitaire. On note $P_{r,d}$ le sous-groupe parabolique de $GL(rd, \mathbb{Q}_p)$ des matrices triangulaires supérieures par bloc où les blocs diagonaux sont de tailles $r \times r$. Alors
\begin{equation}
Ind_{P_{r,d}}^{GL(rd, \mathbb{Q}_p)}(|.|_p^{\frac{1-d}{2}}\tau \boxtimes ... \boxtimes |.|_p^{\frac{d-1}{2}}\tau)
\end{equation}
admet un unique quotient irréductible que l'on note $\tau'$. La représentation $\tau'$ est admissible irréductible de carré intégrable. De plus, toutes les représentations admissibles irréductibles de carré intégrable sont de cette forme (à isomorphisme près) avec $n=rd$.

\begin{proposition}
En reprenant les notations précédentes,
\begin{equation}
L(s, \tau') = \begin{cases}
    L(s,|.|_p^{\frac{n-1}{2}}\tau),& \text{si } r=1\\
    1,              & \text{si } r > 1.
\end{cases}
\end{equation}
\end{proposition}

\begin{proof}
Si $r > 1$, par la compatibilité de l'induction,
\begin{equation}
L(s, Ind_{P_{r,d}}^{GL(rd, \mathbb{Q}_p)}(|.|_p^{\frac{1-d}{2}}\tau \boxtimes ... \boxtimes |.|_p^{\frac{d-1}{2}}\tau)) = \prod_{i=0}^{d-1} L(s, |.|_p^{\frac{1-d}{2}+i}\tau)=1,
\end{equation}
la dernière égalité vient du calcul de la fonction $L$ d'une représentation supercuspidale. On en déduit immédiatement que $L(s,\tau')=1$, d'après la proposition \ref{lfunsupercusp}.

Supposons maintenant que $r=1$. Soit $\phi \in \mathcal{S}$, $f$ un coefficient de $\tau'$ et $s \in \mathbb{C}$. Alors, d'après l'équation fonctionnelle,
\begin{equation}
\zeta(\check{f}, \hat{\phi}, s) = \gamma(s, \tau', \lambda)\zeta(f, \phi, s),
\end{equation}
où la transformée de Fourier $\hat{\phi}$ est calculée par rapport au caractère $\lambda$ définit dans la section \ref{caracqp}.
Les propriétés du facteur $\gamma$ et le calcul pour $GL_1(\mathbb{Q}_p)$ donnent
\begin{equation}
\gamma(s, \tau', \lambda) = \prod_{i=0}^{n-1} \frac{1-\tau(p)p^{-s-\frac{n-1}{2}+i}}{1-\tau(p)^{-1}p^{s-1+\frac{n-1}{2}-i}}.
\end{equation}
Ce qui nous permet d'en déduire immédiatement que
\begin{equation}
\label{carreintegrable1}
\prod_{i=0}^{n-1}(1-\tau(p)p^{-s-\frac{n-1}{2}+i})\zeta(f, \phi, s) =
\prod_{i=1}^{n}(1-\tau(p)^{-1}p^{s+\frac{n-1}{2}-i})\zeta(\check{f}, \hat{\phi}, 1-s).
\end{equation}
On utilise maintenant le fait que
\begin{equation}
(1-\tau(p)^{-1}p^{s+\frac{n-1}{2}-i}) = -\tau(p)^{-1}p^{s+\frac{n-1}{2}-i}(1-\tau(p)p^{-s-\frac{n-1}{2}+i}),
\end{equation}
pour simplifier l'équation (\ref{carreintegrable1}), ce qui nous donne
\begin{equation}
\label{carreintegrable2}
(1-\tau(p)p^{-s-\frac{n-1}{2}})\zeta(f, \phi, s) = (-\tau(p))^{-n+1}
p^{(s-\frac{1}{2})(n-1)}(1-\tau(p)^{-1}p^{s-\frac{n-1}{2}})\zeta(\check{f}, \hat{\phi}, s).
\end{equation}

Pour les représentations de carré intégrable, on dispose d'une information plus précise sur de domaine de convergence de l'intégrale zêta.
\begin{lemme}[\cite{goldfeld-hundley}]
L'intégrale définissant la fonction zêta pour $\tau'$,
\begin{equation}
\zeta(f, \phi, s) = \int_{G} \phi(g)f(g)|\det g|_p^s dg
\end{equation}
est absolument convergente pour $Re(s) > 0$.
\end{lemme}

Ce lemme nous permet d'en déduire que les deux membres de l'équation (\ref{carreintegrable2}) sont holomorphes dans les régions $Re(s) > 0$ et $Re(s) < 1$. Ceci nous permet voir immédiatement que $L(s, \tau') = (1-\tau(p)p^{-s-\frac{n-1}{2}})$ ou $1$. Il nous suffit donc de montrer que $L(s, \tau') \neq 1$ pour avoir le résultat voulu.

On choisit $\phi \in \mathcal{S}$ et $f$ un coefficient de $\pi$ tels que $\zeta(\check{f}, \hat{\phi}, 1-s)$ est une constante non nulle. En effet, on choisit un compact $C$ de $G^0$ suffisamment grand tel qu'il existe un coefficient $f$ de $\pi$ non identiquement nul sur $C$ et on pose alors $\phi(g) = \overline{f(g)}$ si $g \in C$ et $\phi(g)=0$ sinon. Alors le membre de droite de l'équation (\ref{carreintegrable2}) ne s'annule pas en $s=-\frac{n-1}{2}$. On en déduit que $\zeta(f, \phi, s)$ doit avoir un pôle en $s=-\frac{n-1}{2}$, d'où $L(s,\tau')\neq 1$.
\end{proof}
\section{Seconde preuve du théorème \ref{thm_padique}}

On reprend les notations de la section \ref{gln}. Dans cette partie, on propose de présenter une autre preuve qui se base sur un dévissage de l'espace de Schwartz.

\subsection{Convergence de l'intégrale zêta}

Commençons par donner une seconde preuve de la convergence absolue. On va utiliser la
\begin{proposition}
Soit $f$ un coefficient de $\pi$. Alors il existe $c > 0$ et $d \in \mathbb{N}$ tel que pour tout $g \in G$, on ait $|f(g)| \leq c ||g||^d$, où $||g|| = max(|g_{ij}|, |\det g|)$.
\end{proposition}

\begin{proof}
Voir \cite[Corollaire I.4.4]{waldspurger}.
\end{proof}

On sépare l'intégrale
\begin{equation}
\int_G \phi(g)f(g)|\det g|^s dg
\end{equation}
en deux intégrales selon que $max(|g_{ij}|) < 1$ et $max(|g_{ij}|) \geq 1$. Comme $\phi$ est à support compact la deuxième intégrale converge absolument pour tout $s \in \mathbb{C}$. En ce qui concerne la première intégrale, on utilise la proposition pour montrer que
\begin{align}
\int_{max(|g_{ij}|) < 1}|\phi(g)f(g)|\det g|^s dg &\leq C \int_{GL_n(\mathbb{Z}_p)} |\det g|^{s-d} dg \\
&\leq C' \sum_{m_1 \leq ... \leq m_n} p^{-(m_1+...+m_n)(s-d)},
\end{align}
cette série géométrique converge pour $Re(s)$ assez grand.

\subsection{Équation fonctionnelle}
On veut montrer l'équation fonctionnelle suivante
\begin{equation}
\zeta(f, \phi, s) = \gamma(s) \zeta(\check{f}, \hat{\phi}, n-s),
\end{equation}
où $\gamma$ est une fonction rationnelle en $p^s$ et $\check{f}(g) = f(g^{-1})$.

Pour montrer cette équation fonctionnelle, on va utiliser la
\begin{propriete}
Les opérateurs $\zeta(., ., s)$ et $\zeta(\check{.}, \hat{.}, n-s)$ sont des opérateurs d'entrelacements, éléments de $\Hom_{G \times G} ( (\tilde{\pi} \boxtimes \pi) \otimes \mathcal{S}, |\det|_p^s \boxtimes |\det|_p^{-s})$.
\end{propriete}

On précise que l'action de $G \times G$ sur $\mathcal{S}$ est
$(g_1,g_2).\phi(x) = \phi(g_1^{-1} x g_2)$. De plus, on identifie l'ensemble des coefficients de $\pi$ avec l'espace $\tilde{V}\otimes V$; l'action de $G \times G$ sur $\tilde{\pi} \boxtimes \pi$ est $(g_1,g_2).f(g) = f(g_1^{-1} g g_2)$.

\begin{proof}
L'action de $G \times G$ sur $\zeta(f,\phi,s)$ donne
\begin{equation}
\int_G \phi(g_1^{-1}gg_2)f(g_1^{-1}gg_2)|\det g|_p^s dg.
\end{equation}
On effectue le changement de variable $g \mapsto g_1gg_2^{-1},$ le groupe $G$ étant unimodulaire l'intégrale devient
\begin{equation}
|\det g_1g_2^{-1}|^s\int_G \phi(g)f(g)|\det g|_p^s dg.
\end{equation}

D'autre part, l'action de $G \times G$ sur $\zeta(\check{f}, \hat{\phi}, n-s)$ donne
\begin{equation}
\label{intzetafourier}
\int_G \hat{\phi}_{g_1,g_2}(g)\check{f}_{g_1,g_2}(g)|\det g|_p^{n-s} dg,
\end{equation}
où l'on a noté $\phi_{g_1,g_2}(x) = \phi(g_1^{-1}xg_2)$ et $f_{g_1,g_2}(g) = f(g_1^{-1}gg_2).$

Un calcul immédiat, montre que $\check{f}_{g_1,g_2}(g) = \check{f}(g_2^{-1}gg_1)$. De plus,
\begin{equation}
\hat{\phi}_{g_1,g_2}(g) = \int_{M_n} \phi(g_1^{-1}xg_2) \psi(Tr(xg)) dx.
\end{equation}
Après le changement de variable $x \mapsto g_1xg_2^{-1}$ l'intégrale devient
\begin{equation}
|\det g_1^{-1}g_2|_p^n\int_{M_n} \phi(x) \psi(Tr(xg_2^{-1}gg_1)) dx,
\end{equation}
qui n'est autre que $|\det g_1g_2^{-1}|_p^n\hat{\phi}(g_2^{-1}gg_1)$.
L'intégrale (\ref{intzetafourier}) devient donc, après le changement de variable $g \mapsto g_2gg_1^{-1}$,
\begin{equation}|\det g_1^{-1}g_2|_p^n|\det g_2g_1^{-1}|_p^{n-s}\int_G \hat{\phi}(g)\check{f}(g)|\det g|_p^{n-s} dg.
\end{equation}
\end{proof}

Dans le but de comprendre l'espace $\Hom_{G \times G} ( (\tilde{\pi} \boxtimes \pi) \otimes \mathcal{S}, |\det|_p^s \boxtimes |\det|_p^{-s})$, on va décomposer $\mathcal{S}$ selon le rang des matrices. Soit $r$ un entier compris entier $1$ et $n$, on note $S_r$ l'espace des matrices $n \times n$ de rang $r$ et $S^{(r)}$ l'espace des matrices $n \times n$ de rang $< r$.

Si $X$ est un espace localement compact totalement discontinu, on note $C^\infty_c(X)$ l'espace des fonctions $f : X \rightarrow \mathbb{C}$ localement constantes à support compact. L'espace $\mathcal{S}$ est donc égal à $C^\infty_c(M_n)$.

Le groupe $G$ est un ouvert de $M_n$ et $M_n \setminus G = S^{(n)}$. Cette décomposition donne la suite exacte
\begin{equation}
\label{suiteexacte}
0 \rightarrow C^\infty_c(G) \rightarrow C^\infty_c(M_n) \rightarrow C^\infty_c(S^{(n)}) \rightarrow 0,
\end{equation}
où l'inclusion de $C^\infty_c(G)$ dans $C^\infty_c(M_n)$ se fait par extension par $0$ et l'application $C^\infty_c(M_n) \rightarrow C^\infty_c(S^{(n)})$ est l'application de restriction.

Cette suite exacte commute avec l'action de $G \times G$, on la voit donc comme une suite exacte de représentations de $G \times G$. On applique le foncteur $\Hom_{G \times G} (., (\pi \boxtimes \tilde{\pi}) \otimes (|\det|_p^s \boxtimes |\det|_p^{-s}))$, qui est exact à gauche, on en déduit alors l'inégalité suivante :
\begin{multline}
\dim \Hom_{G \times G} ((\tilde{\pi} \boxtimes \pi) \otimes \mathcal{S}, |.|_p^s) \leq \dim \Hom_{G \times G} ((\tilde{\pi} \boxtimes \pi) \otimes C^\infty_c(G), |.|_p^s) \\
+ \dim \Hom_{G \times G} ((\tilde{\pi} \boxtimes \pi) \otimes C^\infty_c(S^{(n)}), |.|_p^s),
\end{multline}
où l'on a abrégé $|.|_p^s = |\det|_p^s \boxtimes |\det|_p^{-s}$.

On décompose ensuite $S^{(n)}$ selon le rang $r$, ce qui donne, en utilisant le même raisonnement, que
\begin{equation}
\dim \Hom_{G \times G} ((\tilde{\pi} \boxtimes \pi) \otimes \mathcal{S}, |.|_p^s) \leq \sum_{r=0}^{n} \dim \Hom_{G \times G} ((\tilde{\pi} \boxtimes \pi) \otimes C^\infty_c(S_r), |.|_p^s).
\end{equation}

Il ne nous reste plus qu'à calculer la dimension de ces différents espaces, pour cela on dispose de la
\begin{proposition}
Pour $r=n$ $(S_r = G)$, on a
\begin{equation}
\dim \Hom_{G \times G} ( (\tilde{\pi} \boxtimes \pi) \otimes C_{c}^\infty(G), |.|_p^s) = 1;
\end{equation}
et pour $r < n$, on a
\begin{equation}
\Hom_{G \times G} ( (\tilde{\pi} \boxtimes \pi) \otimes C_{c}^\infty(S_r), |.|_p^s) = 0
\end{equation}
sauf pour un nombre fini de valeurs de $s$ modulo $\frac{2i\pi}{(n-r)\log p}\mathbb{Z}$.
\end{proposition}

\begin{proof}

Commençons par le cas $r=n$,
\begin{align}
\Hom_{G \times G}( (\tilde{\pi} \boxtimes \pi) \otimes C_c^\infty(G), |.|_p^s) &\simeq \Hom_{G \times G}( (\tilde{\pi} \boxtimes \pi) \otimes |.|_p^{-s}, C^\infty(G)) \\
&\simeq \Hom_H( (\tilde{\pi} \boxtimes \pi) \otimes |.|_p^{-s}, \mathbb{C}) \\
&\simeq \Hom_G(\tilde{\pi}, \tilde{\pi});
\end{align}
où le groupe $H$ désigne la diagonale de $G \times G$. Ce dernier espace est bien de dimension $1$ d'après le lemme de Schur.

Le premier isomorphisme provient de la dualité entre $C_c^\infty(G)$ et $C^\infty(G)$. Le deuxième isomorphisme est une application de la réciprocité de Frobenius avec l'identification $C^\infty(G) = Ind_H^{G \times G}(1)$. Pour finir, le dernier isomorphisme provient du fait que l'action diagonale de $H$ sur $\tilde{\pi} \boxtimes \pi$ correspond à l'action de $G$ sur $\tilde{\pi} \otimes \pi$ et que $|.|_p^{-s}$ est trivial sur $H$.

Passons au cas $r < n$, $S_r$ est l'orbite de $\begin{pmatrix} 
1_r & 0 \\
0 & 0 
\end{pmatrix}$ sous l'action de $G \times G$ par translation à gauche du premier facteur et translation à droite de l'inverse sur le second facteur. On calcule le stabilisateur,
\begin{equation}
H = Stab_{G \times G} \begin{pmatrix} 
1_r & 0 \\
0 & 0 
\end{pmatrix} = \left\lbrace \left(\begin{pmatrix} 
a & b \\
0 & c 
\end{pmatrix}, \begin{pmatrix} 
a & 0 \\
d & e 
\end{pmatrix} \right) \right\rbrace \subset G \times G,
\end{equation}
où $a$ décrit $GL_r(\mathbb{Q}_p)$; $c, e$ décrivent $GL_{n-r}(\mathbb{Q}_p)$; $b$ décrit $M_{r,n-r}(\mathbb{Q}_p)$ et $d$ décrit $M_{n-r,r}(\mathbb{Q}_p)$.
 
On note $P = MN$ le sous-groupe parabolique de $G$ des matrices de la forme $\begin{pmatrix} 
a & b \\
0 & c 
\end{pmatrix}$ et $\bar{P} = M\bar{N}$ le groupe parabolique opposé, alors $H \subset P \times \bar{P}$.

\begin{equation}
\label{isom}
\begin{split}
\Hom( (\tilde{\pi} \boxtimes \pi) \otimes C_c^\infty(S_r), |.|_p^s) &\simeq \Hom_{G \times G}( (\tilde{\pi} \boxtimes \pi) \otimes |.|_p^{-s}, Ind_H^{G \times G}(\delta_H)) \\
&\simeq \Hom_{M \times M}( (\tilde{\pi} \boxtimes \pi)_{N \times \bar{N}} \otimes |.|_p^{-s}, Ind_{(M \times M)\cap H}^{M \times M}(\delta_H)) \\
&\simeq \Hom_{(M \times M) \cap H}( (\tilde{\pi} \boxtimes \pi)_{N \times \bar{N}}, \delta_H \otimes |.|_p^s),
\end{split}
\end{equation}
où $\delta_H$ est le caractère modulaire de $H$.

Le premier isomorphisme provient de l'identification de $C_c^\infty(S_r)=c-Ind_H^{G \times G}(1)$ et de la dualité entre $c-Ind_H^{G \times G}(1)$ et $Ind_H^{G \times G}(\delta_H)$. Pour le deuxième isomorphisme, on utilise la transitivité de l'induction, $H \subset P \times \bar{P} \subset G \times G$, et l'adjonction entre $Ind_{P \times \bar{P}}^{G \times G}$ et le foncteur de Jacquet; en remarquant, que $N \times \bar{N}$ agit trivialement sur $|.|_p^{-s}$. Le dernier isomorphisme n'est autre que la réciprocité de Frobenius.

On utilise le fait que $(\tilde{\pi} \boxtimes \pi)_{N \times \bar{N}}$ est de longueur finie; en effet le foncteur de Jacquet préserve la longueur finie. Il existe donc des représentations admissibles $V_i$ de $M \times M$ telles que
\begin{equation}
0=V_0 \subset V_1 \subset ... \subset V_l = (\tilde{\pi} \boxtimes \pi)_{N \times \bar{N}},
\end{equation}
avec $V_i/V_{i-1}$ irréductibles.

En reprenant un raisonnement que l'on a déjà fait, la suite exacte de représentations de $M \times M$
\begin{equation}
0 \rightarrow V_{i-1} \rightarrow V_i \rightarrow V_i/V_{i-1} \rightarrow 0
\end{equation}
permet d'obtenir l'inégalité suivante :
\begin{equation}
\dim \Hom_{(M \times M)\cap H} ((\tilde{\pi} \boxtimes \pi)_{N \times \bar{N}}, |.|_p^s\delta_H) \leq \sum_{i=1}^{l} \dim \Hom_{(M \times M)\cap H} (V_i/V_{i-1}, |.|_p^s\delta_H).
\end{equation}

Il nous suffit donc de montrer que ces derniers espaces sont nuls sauf pour au plus une valeur de $s$ modulo $\frac{2i\pi}{(n-r)\log p}\mathbb{Z}$.

En tant que représentation irréductible de $M \times M \simeq GL_r^2(\mathbb{Q}_p) \times GL_{n-r}^2(\mathbb{Q}_p)$, on peut décomposer $V_i/V_{i-1} \otimes \delta_H^{-1}$ sous la forme
$\sigma^{(i)} \boxtimes (\tau_1^{(i)} \boxtimes \tau_2^{(i)})$, où $\sigma^{(i)}$ est une représentation irréductible de $GL_r^2(\mathbb{Q}_p)$ et $\tau_1^{(i)}, \tau_2^{(i)}$ sont des représentations irréductibles de $GL_{n-r}(\mathbb{Q}_p)$.

D'après le lemme de Schur, la représentation $\tau_2^{(i)}$ admet un caractère central $\omega^{(i)}$. On en déduit que
\begin{equation}
Hom_{(M \times M) \cap H} (V_i/V_{i-1}, |.|_p^s\delta_H) = 0,
\end{equation}
sauf si $\omega^{(i)} = |.|_p^{-(n-r)s}$ sur $\mathbb{Q}_p^\times$. Cette équation n'est en fait vérifiée que pour au plus une valeur de $s$ modulo $\frac{2i\pi}{(n-r)\log p}\mathbb{Z}$.
\end{proof}

Terminons la preuve de l'équation fonctionnelle. Rappelons que les opérateurs $\zeta(., ., s)$ et $\zeta(\check{.}, \hat{.}, n-s)$ sont des éléments de $\Hom_{G \times G} ( (\tilde{\pi} \boxtimes \pi) \otimes \mathcal{S}, |\det|_p^s \boxtimes |\det|_p^{-s})$, qui est de dimension $1$ sauf pour un nombre fini de valeurs de $s$ modulo $\sum_{r=0}^{n-1}\frac{2i\pi}{(n-r)\log p}\mathbb{Z}$.

Autrement dit, pour $s$ en dehors de cet ensemble de valeurs exceptionnelles, il existe $\gamma(s) \in \mathbb{C}$ tel que
\begin{equation}
\label{eqfun}
\zeta(., ., s) = \gamma(s) \zeta(\check{.}, \hat{.}, n-s).
\end{equation}

Les fonctions zêta étant des fonctions rationnelles en $p^s$ et l'ensemble des valeurs de $s$ pour lesquelles $\gamma$ est ainsi défini est dense pour la topologie de Zariski, on en déduit que l'on peut étendre $\gamma$ en une fonction rationnelle en $p^s$ pour laquelle l'équation (\ref{eqfun}) est vérifiée en tant qu'égalité de fonctions rationnelles en $p^s$.

\subsection{Familles de représentations}

On veut maintenant montrer que la fonction zêta est une fraction rationnelle où l'on peut choisir le dénominateur de façon indépendante de $f$ et de $\phi$.

Si l'on note $\pi_s = \pi \otimes |\det|^s$, alors $\zeta(f, \phi, s)$ est élément de $\Hom_{G \times G}((\tilde{\pi_s} \boxtimes \pi_s) \otimes \mathcal{S}(M), \mathbb{C})$. On va donc considérer $\pi_s$ comme une famille de représentations paramétrée par $s$.

\begin{definition}
Soit $B$ une $\mathbb{C}$-algèbre commutative noethérienne réduite. Un $(G, B)$-module (ou $B$-famille de représentation de $G$) est une représentation lisse $(\pi_B, V_B)$ de $G$ munie d'une structure de $B$-module plat, qui commute à l'action de $G$.
\end{definition}

On pose $B=\mathbb{C}[G/G^0]$ l'algèbre des polynômes sur la variété des caractères non ramifiés $X^*(G)$ de $G$. On définit alors un $(G,B)$-module $(\pi_B, V_B)$ par $V_B = V \otimes B$ et $\pi_B(g)(v \otimes b) = \pi(g)v \otimes gb$ pour $v \in V, b \in B$ et $g \in G$.

La variété des caractères non ramifiés est $X^*(G) = \left\lbrace |\det|^s, s \in \mathbb{C} \right\rbrace$, on retrouve ainsi la famille de représentation $\pi_s$.

Lorsque l'on restreint l'espace de Schwartz à $\mathcal{S}(G)$, la fonction zêta $\zeta(f, \phi, .)$ est un polynôme en $p^s$. On en déduit que $\zeta_{|\mathcal{S}(G)}$ (que l'on étend par linéarité à $B$) est un élément de $\Hom_{B, G \times G}((\tilde{\pi_B} \boxtimes \pi_B) \otimes \mathcal{S}(G), B)$.

On reprend la suite exacte (\ref{suiteexacte}) et on applique le foncteur $\Hom_{B, G \times G} (., \pi_B \boxtimes \tilde{\pi_B})$, ce qui nous donne
\begin{equation}
\begin{split}
0 \rightarrow \Hom_{B, G \times G}((\tilde{\pi_B} \boxtimes \pi_B) \otimes C^\infty_c(S^{(n)}), B) &\rightarrow \Hom_{B, G \times G}((\tilde{\pi_B} \boxtimes \pi_B) \otimes C^\infty_c(M_n), B) \\
&\rightarrow \Hom_{B, G \times G}((\tilde{\pi_B} \boxtimes \pi_B) \otimes C^\infty_c(G), B)
\end{split}
\end{equation}

Si la dernière flèche était surjective, cela montrerait immédiatement que $\zeta \in \Hom_{B, G \times G}((\tilde{\pi_B} \boxtimes \pi_B) \otimes C^\infty_c(M_n), B)$. Cependant, en général, elle n'est pas surjective. On doit donc utiliser le terme suivant de cette suite exacte, qui est $\Ext^1_{B, G \times G}((\tilde{\pi_B} \boxtimes \pi_B) \otimes C^\infty_c(S^{(n)}), B)$.

Le but est maintenant de trouver un élément de $\Hom_{B, G \times G}((\tilde{\pi_B} \boxtimes \pi_B) \otimes C^\infty_c(G), B)$ dont l'image dans $\Ext^1_{B, G \times G}((\tilde{\pi_B} \boxtimes \pi_B) \otimes C^\infty_c(S^{(n)}), B)$ est nulle, ce qui nous permettra de le relever en un élément de $\Hom_{B, G \times G}((\tilde{\pi_B} \boxtimes \pi_B) \otimes C^\infty_c(M_n), B)$. Plus exactement, on va montrer qu'il existe un élément non nul de $B$ qui agit par $0$ dans $\Ext^1_{B, G \times G}((\tilde{\pi_B} \boxtimes \pi_B) \otimes C^\infty_c(S^{(n)}), B)$.

Pour $r \in \left\lbrace 1, ..., n \right\rbrace$, on décompose l'espace $S^{(r)}$ des matrices de rang $< r$ en $S^{(r)}=S^{(r-1)} \cup S_{r-1}$ où $S_{r-1}$ est l'ouvert de $S^{(r)}$ des matrices de rang $r-1$, pour obtenir la suite exacte
\begin{equation}
0 \rightarrow C^\infty_c(S_{r-1}) \rightarrow C^\infty_c(S^{(r)}) \rightarrow C^\infty_c(S^{(r-1)}) \rightarrow 0.
\end{equation}
En prolongeant la suite exacte du foncteur $\Hom_{B, G \times G} (., \pi_B \boxtimes \tilde{\pi_B})$, on en déduit la suite exacte
\begin{equation}
\begin{split}
\Ext^1_{B, G \times G}((\tilde{\pi_B} \boxtimes \pi_B) \otimes C^\infty_c(S^{(r-1)}), B) &\rightarrow \Ext^1_{B, G \times G}((\tilde{\pi_B} \boxtimes \pi_B) \otimes C^\infty_c(S^{(r)}), B) \\ 
&\rightarrow \Ext^1_{B, G \times G}((\tilde{\pi_B} \boxtimes \pi_B) \otimes C^\infty_c(S_{r-1}), B)
\end{split}
\end{equation}

\begin{lemme}
Soit $M,N,P$ des $B$-module qui satisfont la suite exacte de $B$-modules
\begin{equation}
M \xrightarrow{\alpha} N \xrightarrow{\beta} P.
\end{equation}
Supposons qu'il existe $b_1, b_2 \in B$ tel que $b_1M = 0$ et $b_2P=0$. Alors $b_1b_2N=0$.
\end{lemme}

\begin{proof}
Soit $n \in N$. Alors $\beta(n) \in P$, donc $b_2\beta(n)=0$. On en déduit qu'il existe $m \in M$ tel que $\alpha(m)=b_2n$. Or $b_1m=0$, d'où $b_1b_2n=\alpha(b_1m)=0$.
\end{proof}

Ce lemme nous permet d'en déduire qu'il suffit de montrer que les espaces $\Ext^1_{B, G \times G}((\tilde{\pi_B} \boxtimes \pi_B) \otimes C^\infty_c(S_r), B)$ sont de torsion, pour $r \in \left\lbrace 0, ..., n-1 \right\rbrace$.

De plus, on dispose de l'isomorphisme suivant
\begin{equation}
\Ext^1_{B, G \times G}((\tilde{\pi_B} \boxtimes \pi_B) \otimes C^\infty_c(S_r), B) \simeq \Ext^1_{B, (M \times M) \cap H}((\tilde{\pi_B} \boxtimes \pi_B)_{N \times \bar{N}}, \delta_H).
\end{equation}
Pour justifier cet isomorphisme, on reprend la suite d'isomorphismes (\ref{isom}) et on montre que les isomorphismes successifs passent à $Ext^1$ en prenant une résolution projective.

Considérons l'action de $(z_1,z_2) \in Z(GL_{n-r}) \times Z(GL_{n-r})$ sur ce dernier. Il agit par $z_1^{-1}z_2 \in G/G_0$ sur $(\tilde{\pi}_B \boxtimes \pi_B)_{N \times \bar{N}}$ et par $1$ sur $\delta_H$. On en déduit que $z_1^{-1}z_2-1 \in B$ agit par $0$ dans $\Ext^1_{B, (M \times M) \cap H}((\tilde{\pi_B} \boxtimes \pi_B)_{N \times \bar{N}}, \delta_H)$. Si l'on choisi $z_1 \neq z_2 \mod G_0$, on en déduit que $\Ext^1_{B, G \times G}((\tilde{\pi_B} \boxtimes \pi_B) \otimes C^\infty_c(S_r), B)$ est de torsion.

Terminons le raisonnement. Notons $Q$ le polynôme qui correspond à l'élément de $B$ qui agit par $0$ dans $\Ext^1_{B, G \times G}((\tilde{\pi_B} \boxtimes \pi_B) \otimes C^\infty_c(S^{(n)}), B)$. Alors par $B$-linéarité, on en déduit que l'image de $Q\zeta_{|\mathcal{S}(G)}$ dans $\Ext^1_{B, G \times G}((\tilde{\pi_B} \boxtimes \pi_B) \otimes C^\infty_c(S^{(n)}), B)$ est nulle. Ce qui permet d'en déduire que $Q\zeta_{|\mathcal{S}(G)}$ se relève en un élément de $\Hom_{B, G \times G}((\tilde{\pi_B} \boxtimes \pi_B) \otimes \mathcal{S}(M_n), B)$, que l'on note $\zeta_0$. Alors $\zeta_0(s)$ est un élément de $\Hom_{G \times G} ( (\tilde{\pi} \boxtimes \pi) \otimes \mathcal{S}, |\det|_p^s \boxtimes |\det|_p^{-s})$, qui est de dimension $1$ pour presque tout $s \in \mathbb{C}$, donc il est proportionnel à $\zeta(., ., s)$. De plus, la restriction de $\zeta_0$ à $\mathcal{S}(G)$ est $Q\zeta_{|\mathcal{S}(G)}$. On en déduit que $\zeta_0(s) = Q(p^s, p^{-s})\zeta(.,.,s)$ pour presque tout $s$; c'est un polynôme en $p^s$ et $p^{-s}$. Ce qui montre que l'on peut bien choisir le dénominateur indépendant de $f$ et $\phi$; $Q$ convient.
\section{Fonctions zêta sur $GL_n(\mathbb{R})$}
\section{Fonctions zêta sur $GL_n(\mathbb{A})$}

On note $G = GL_n(\mathbb{Q})$ et $G_\mathbb{A}=GL_n(\mathbb{A})$. On pose $K = O_n(\mathbb{R}) \times \prod_p GL_n(\mathbb{Z}_p)$, c'est un sous-groupe compact maximal de $G_\mathbb{A}$.

\subsection{Formes cuspidales}
On considère une fonction $f : G \backslash G_\mathbb{A} \rightarrow \mathbb{C}$ qui vérifie :
\begin{itemize}
\item $f$ est $K$-finie à droite,
\item il existe un caractère $\omega : \mathbb{A}^\times/\mathbb{Q}^\times \rightarrow \mathbb{S}^1$ tel que
\begin{equation}
f(zg) = \omega(z)f(g) \quad \forall g \in G_\mathbb{A}, z \in \mathbb{A}^\times,
\end{equation}
\item $f$ est à décroissance rapide,
\item $f$ est cuspidale :
\begin{equation}
\int_{U \backslash U_\mathbb{A}} f(ug) du = 0
\end{equation}
pour tout radical unipotent $U$ d'un sous-groupe parabolique de $G$ et tout $g \in G_\mathbb{A}$.
\end{itemize}
On dira qu'une telle fonction est une forme cuspidale.

Pour $\phi \in \mathcal{S}(M_\mathbb{A})$ et $s \in \mathbb{C}$, on pose
\begin{equation}
\zeta(f, \phi, s) = \int_{G_\mathbb{A}} \phi(g) f(g) |\det g|_\mathbb{A}^s dg,
\end{equation}
où $dg = \otimes_v dg_v$ est une mesure de Haar sur $GL_n(\mathbb{A})$ et $|.|_\mathbb{A} = \prod_v |.|_v$ est la valeur absolue adélique.

Notons $G^0_\mathbb{A}=\left\lbrace g \in G_\mathbb{A}, \det g = 1 \right\rbrace$. Comme $\mathbb{R}_{> 0} \subset \mathbb{A}^\times=Z(G_\mathbb{A})$, l'application $\det : G_\mathbb{A} \rightarrow \mathbb{R}_{> 0}$ est surjective de noyau $G^0_\mathbb{A}$.

La factorisation $G_\mathbb{A} = \mathbb{R}_{> 0}G^0_\mathbb{A}$ permet d'obtenir que
\begin{align}
\zeta(f, \phi, s) &= \int_0^\infty \int_{G^0_\mathbb{A}} \phi(tg) \omega(t) f(g) t^{ns} dg dt \\
&= \int_0^\infty \int_{G \backslash G^0_\mathbb{A}} \sum_{x \in G}{\phi(txg)} f(g) \omega(t) t^{ns} dg dt
\end{align}

Comme dans la preuve de l'équation fonctionnelle de la fonction zêta de Riemann, on scinde l'intégrale en $1$ pour faire apparaître une symétrie. Autrement dit,
\begin{equation}
\begin{split}
\zeta(f, \phi, s) &= \int_0^1 \int_{G \backslash G^0_\mathbb{A}} \sum_{x \in G}{\phi(txg)} f(g) \omega(t) t^{ns} dg dt \\
&+ \int_1^\infty \int_{G \backslash G^0_\mathbb{A}} \sum_{x \in G}{\phi(txg)} f(g) \omega(t) t^{ns} dg dt.
\end{split}
\end{equation}

La seconde intégrale converge absolument pour tout $s \in \mathbb{C}$, c'est une fonction entière. Pour la première intégrale, on fait le changement de variable $t \mapsto t^{-1}$, ce qui donne
\begin{equation}
\int_1^\infty \int_{G \backslash G^0_\mathbb{A}} \sum_{x \in G}{\phi(t^{-1}xg)} f(g) \omega^{-1}(t) t^{-ns} dg dt.
\end{equation}

On va maintenant utiliser la formule de Poisson sur $M_\mathbb{A}$, ce qui donne pour la fonction $x \mapsto \phi(t^{-1}xg)$ :
\begin{equation}
\sum_{x \in M} \phi(t^{-1}xg) = t^{n^2}|\det g|_\mathbb{A}^{-1}\sum_{x \in M} \hat{\phi}(txg^{-1}).
\end{equation}

On scinde la somme selon le rang de la matrice et on obtient :
\begin{equation}
\begin{split}
\sum_{x \in G} \phi(t^{-1}xg) &= t^{n^2}|\det g|_\mathbb{A}^{-1}\sum_{x \in G} \hat{\phi}(txg^{-1}) \\
&+ \sum_{r < n, rg(x)=r} \left( t^{n^2}|\det g|_\mathbb{A}^{-1}\hat{\phi}(txg^{-1}) - \phi(t^{-1}xg)\right).
\end{split}
\end{equation}

La contribution de la dernière somme est nulle dans l'intégrale définissant la fonction zêta. Ce qui nous permet d'en déduire la
\begin{proposition}
La fonction $\zeta(f, \phi, .)$ peut être prolonger en une fonction entière et vérifie l'équation fonctionnelle
\begin{equation}
\zeta(f, \phi, s) = \zeta(\check{f}, \hat{\phi}, n-s),
\end{equation}
où $\check{f}(g)=f(g^{-1})$.
\end{proposition}

\subsection{Représentations automorphes}

Soit $\pi$ une représentation admissible irréductible de $L_0^2(G \backslash G_\mathbb{A}, \omega)$. Un coefficient $f$ de $\pi$ est de la forme
\begin{equation}
f(g) = <\pi(g)\varphi, \tilde{\varphi}> = \int_{\mathbb{A}^\times G \backslash G_\mathbb{A}} \varphi(hg) \tilde{\varphi}(h) dh,
\end{equation}
où $\varphi \in \pi$ et $\tilde{\varphi} \in \tilde{\pi}$. On dira que $f$ est un coefficient admissible si $\varphi$ et $\tilde{\varphi}$ sont des formes cuspidales.

Pour un coefficient admissible $f$ de $\pi$, $\phi \in \mathcal{S}(M_\mathbb{A})$ et $s \in \mathbb{C}$, on pose
\begin{equation}
\zeta(f, \phi, s) = \int_{G_\mathbb{A}} \phi(g) f(g) |\det g|_\mathbb{A}^s dg.
\end{equation}

On peut déduire les propriétés de cette fonction zêta grâce à ce qui l'on a fait précédemment sur les formes cuspidales. Plus précisément, on a
\begin{align}
\zeta(f, \phi, s) &= \int_{G_\mathbb{A}}\phi(g)\int_{\mathbb{A}^\times G \backslash G_\mathbb{A}} \varphi(hg) \tilde{\varphi}(h) dh |\det g|_\mathbb{A}^s dg \\
&= \int_{\mathbb{A}^\times G \backslash G_\mathbb{A}} \tilde{\varphi}(h) \int_{G_\mathbb{A}}\phi(h^{-1}g)\varphi(g)|\det g|_\mathbb{A}^s dg |\det h|^{-s} dh \\
&= \int_{\mathbb{A}^\times G \backslash G_\mathbb{A}} \tilde{\varphi}(h) \zeta(\varphi, \phi(h^{-1}.), s)|\det h|^{-s} dh,
\end{align}
où la deuxième égalité s'obtient grâce au changement de variable $g \mapsto h^{-1}g$. Ceci nous permet de démontrer la
\begin{proposition}
Si $f$ est un coefficient admissible de $\pi$, la fonction $\zeta(f, \phi, .)$ peut se prolonger en une fonction entière et vérifie l'équation fonctionnelle
\begin{equation}
\zeta(f, \phi, s) = \zeta(\check{f}, \hat{\phi}, .),
\end{equation}
où $\check{f}(g) = f(g^{-1}$.
\end{proposition}

Si l'on combine ce résultat avec les résultats locaux, on peut construire la fonction $L$ attachée à une représentation automorphe irréductible. Plus précisément, on a le
\begin{theoreme}
Soit $\pi$ une représentation automorphe irréductible, telle que $\pi \subset \mathcal{A}_0$.

Le produit $L(s, \pi) = \prod_v L(s, \pi_v)$, qui est défini pour $Re(s) > n$, se prolonge en une fonction entière. De plus, $L(s, \pi)$ vérifie l'équation fonctionnelle
\begin{equation}
L(s,\pi) = \epsilon(s,\pi)L(1-s,\tilde{\pi}),
\end{equation}
où $\epsilon(s,\pi) = \prod_v \epsilon(s, \pi_v)$.
\end{theoreme}

\begin{proof}
Comme $\pi$ est admissible, elle se décompose en facteurs locaux,
$\pi = \hat{\otimes}_v \pi_v$, où pour presque toutes les places $\pi_v$ est sphérique (contient la représentation unité de $K_v$).

D'après les résultats locaux, pour chaque place $v$, il existe un nombre fini $(\phi_{\alpha_v})_{\alpha_v \in I_v}$ d'éléments de $\mathcal{S}(M_v)$ et de coefficient $(f_{\alpha_v})_{\alpha_v \in I_v}$ de $\pi_v$ tel que
\begin{equation}
\sum_{\alpha_v \in I_v} \zeta(f_{\alpha_v}, \phi_{\alpha_v}, s + \frac{1}{2}(n-1)) = L(s, \pi_v).
\end{equation}
De plus, d'après l'équation fonctionnelle locale
\begin{equation}
\sum_{\alpha_v \in I_v} \zeta(\check{f}_{\alpha_v}, \hat{\phi}_{\alpha_v}, n-s + \frac{1}{2}(n-1)) = \epsilon(s,\pi_v)L(1-s, \tilde{\pi}_v).
\end{equation}

Notons $I = \prod_v I_v$. Pour presque toutes les places $v$, $I_v$ est un singleton, puisque $\pi_v$ est sphérique; donc $I$ est fini.

Pour $\alpha = (\alpha_v) \in I$, on pose
\begin{equation}
\phi_\alpha = \prod_v \phi_{\alpha_v}, \quad f_\alpha = \prod_v f_{\alpha_v}.
\end{equation}
Alors $\phi_\alpha \in \mathcal{S}(M_\mathbb{A})$ et $f_\alpha$ est un coefficient (admissible) de $\pi$. De plus,
\begin{equation}
\zeta(f_\alpha, \phi_\alpha, s) = \prod_v \zeta(f_{\alpha_v}, \phi_{\alpha_v}, s).
\end{equation}

On en déduit que
\begin{align}
L(s, \pi) &= \prod_v L(s, \pi_v) = \prod_v \sum_{\alpha_v \in I_v} \zeta(f_{\alpha_v}, \phi_{\alpha_v}, s + \frac{1}{2}(n-1)) \\
&= \sum_{\alpha \in I} \zeta(f_\alpha, \phi_\alpha, s + \frac{1}{2}(n-1))
\end{align}
est une somme finie de fonction zêta, qui chacune se prolonge en une fonction entière. De plus,
\begin{align}
L(s, \pi) &= \sum_{\alpha \in I} \zeta(f_\alpha, \phi_\alpha, s + \frac{1}{2}(n-1)) \\
&= \sum_{\alpha \in I} \zeta(\check{f}_\alpha, \hat{\phi}_\alpha, n - s + \frac{1}{2}(n-1)) \\
&= \prod_v \sum_{\alpha_v \in I_v} \zeta(\check{f}_{\alpha_v}, \hat{\phi}_{\alpha_v}, n-s + \frac{1}{2}(n-1)) \\
&= \prod_v \epsilon(s, \pi_v) L(1-s, \tilde{\pi}_v) \\
&= \epsilon(s, \pi)L(1-s, \tilde{\pi}).
\end{align}
\end{proof}

\bibliographystyle{siam}
\bibliography{zeta}
 
\end{document}