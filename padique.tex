\section{Fonctions zêta sur $GL_n(\mathbb{Q}_p), n > 1$}

\label{gln}
Dans la suite, on notera $G = GL_n(\mathbb{Q}_p)$, $dg$ une mesure de Haar sur $G$ et $(\pi, V)$ une représentation admissible irréductible de $G$. On pose $K=GL_n(\mathbb{Z}_p)$, c'est un sous-groupe compact maximal de $G$.

\begin{definition}
Une représentation $\pi : G \rightarrow GL(V)$ sur un $\mathbb{C}$-espace vectoriel $V$ est dite admissible si elle vérifie :
\begin{itemize}
\item Pour tout $v \in V$, le stabilisateur de $v$ dans $G$, $\left\lbrace g \in G, \pi(g)v = v \right\rbrace$, est un sous-groupe ouvert de $G$,
\item Pour tout sous-groupe ouvert $H$ de $G$, le sous-espace
\begin{equation*}
V^H=\left\lbrace v \in V, \pi(h)v = v, \forall h \in H \right\rbrace
\end{equation*}
des vecteurs stable par $H$ est de dimension fini.
\end{itemize}
\end{definition}

Les coefficients de $\pi$ sont les fonctions de la forme $g \in G \mapsto <\pi(g)v, \tilde{v}>$, où $v \in V$ et $\tilde{v} \in \tilde{V}$. Alors $\check{f}(g)=f(g^{-1})=<v, \tilde{\pi}(g)\tilde{v}>$ est un coefficient de $\tilde{\pi}$.

On note $M_n$ l'ensemble des matrices $n \times n$ à coefficients dans $\mathbb{Q}_p$ et $\mathcal{S}$ l'ensemble des fonctions $\phi : M_n \rightarrow \mathbb{C}$ localement constantes à support compact.

Si $f$ est un coefficient de $\pi$, $\phi \in \mathcal{S}$ et $s \in \mathbb{C}$, on pose
\begin{equation}
\zeta(f, \phi, s) = \int_G \phi(g)f(g)|\det g|_p^s dg.
\end{equation}

On fixe un caractère non trivial $\psi$ de $\mathbb{Q}_p$ et on pose
\begin{equation}
\hat{\phi}(y) = \int_{M_n} \phi(x) \psi(Tr(xy)) dx,
\end{equation}
où $dx$ est une mesure de Haar sur $M_n$, normalisée telle que $\hat{\hat{\phi}}(x)=\phi(-x)$.

L'objectif de cette section est de montrer le
\begin{theoreme}
\label{thm_padique}
\begin{enumerate}
\item Il existe $s_0 \in \mathbb{R}$ tel que pour tout $s \in \mathbb{C}$ vérifiant $Re (s) > s_0$, $\phi \in \mathcal{S}$ et $f$ un coefficient de $\pi$, les intégrales
\begin{align}
\zeta(f, \phi, s) &= \int_G \phi(g)f(g)|\det g|_p^s dg \\
\zeta(\check{f}, \phi, s) &= \int_G \phi(g)\check{f}(g)|\det g|_p^s dg
\end{align}
convergent absolument.
\item Ces intégrales sont des fonctions rationnelles en $p^{-s}$. Plus précisément, il existe des polynômes $Q$ et $\tilde{Q}$ indépendant de $f$ et $\phi$ avec $Q(0)\neq 0$ (respectivement $\tilde{Q}(0)\neq 0$) et des polynômes $\Xi(f, \phi, s)$, $\tilde{\Xi}(\check{f}, \phi, s)$ en $p^{s}$ et $p^{-s}$ tel que
\begin{align}
\zeta(f, \phi, s+\frac{1}{2}(n-1)) &= \frac{\Xi(f, \phi, s)}{Q(p^{-s})}, \\
\zeta(\check{f}, \phi, s+\frac{1}{2}(n-1)) &= \frac{\tilde{\Xi}(\check{f}, \phi, s)}{\tilde{Q}(p^{-s})},
\end{align}
pour tout $s \in \mathbb{C}$, $\phi \in \mathcal{S}$ et $f$ coefficient de $\pi$.
\item On peut choisir un nombre fini, de coefficients $f_i$ de $\pi$ (respectivement $\tilde{\pi}$) et de fonctions $\phi_i \in \mathcal{S}$, telles que $\sum_i \Xi(f_i, \phi_i, s)$ (respectivement $\sum_i \tilde{\Xi}(f_i, \phi_i, s)$ soit une constante non nulle.
\item Il existe une fonction $\epsilon(s, \pi, \psi)$, qui est à une constante près une puissance de $p^{-s}$, telle que
\begin{equation}
\label{epsilon}
\tilde{\Xi}(\check{f}, \hat{\phi}, 1-s) = \epsilon(s, \pi, \psi)\Xi(f, \phi, s),
\end{equation}
pour tout $s\in \mathbb{C}$, $\phi \in \mathcal{S}$ et $f$ coefficient de $\pi$.
\end{enumerate}
\end{theoreme}

Les conditions (2) et (3) caractérisent (à constante près) les polynômes $Q$ et $\tilde{Q}$. On normalise $Q$ et $\tilde{Q}$ tel que $Q(0)=\tilde{Q}(0)=1$, on pose alors
\begin{equation}
L(s, \pi) = \frac{1}{Q(p^{-s})}, \quad L(s, \tilde{\pi}) = \frac{1}{\tilde{Q}(p^{-s})}.
\end{equation}

L'existence de la fonction $\epsilon(s, \pi, \psi)$ est équivalente à l'existence d'une fonction méromorphe $\gamma(s,\pi,\psi)$ telle que
\begin{equation}
\zeta(\check{f}, \hat{\phi}, 1-s+\frac{1}{2}(n-1))=\gamma(s, \pi, \psi)\zeta(f, \phi, s),
\end{equation}
pour tout $\phi \in \mathcal{S}$ et $f$ coefficient de $\pi$. Ces deux fonctions étant reliées par la relation
\begin{equation}
\label{gammaepsilon}
\epsilon(s,\pi,\psi)=\gamma(s,\pi,\psi)\frac{L(s,\pi)}{L(1-s,\tilde{\pi})}.
\end{equation}
En effet, supposons l'existence de $\gamma(s,\pi,\psi)$ alors $\epsilon(s,\pi,\psi)$ vérifie 
\begin{equation}
\tilde{\Xi}(\check{f}, \hat{\phi}, 1-s) = \epsilon(s, \pi, \psi)\Xi(f, \phi, s).
\end{equation}
On a de plus une égalité similaire avec $\epsilon(s,\tilde{\pi},\psi)$,
\begin{equation}
\Xi(f, s, \hat{\hat{\phi}}, s)=\epsilon(1-s, \tilde{\pi}, \psi)\tilde{\Xi}(\check{f}, \hat{\phi}, 1-s).
\end{equation}
Il ne nous reste plus qu'à utiliser la formule $\hat{\hat{\phi}}(x)=\phi(-x)$ pour obtenir la relation
\begin{equation}
\epsilon(s, \pi, \psi)\epsilon(1-s, \tilde{\pi}, \psi)=\omega(-1),
\end{equation}
où $\omega$ est le caractère de $\mathbb{Q}_p^\times$ tel que $\pi(z)=\omega(z)1$ pour $z\in \mathbb{Q}_p^\times$. D'après (2) et (3) du théorème, $\epsilon(s, \pi, \psi)$ est alors un polynôme en $p^s$ et $p^{-s}$, on en déduit que $\epsilon(s, \pi, \psi)$ est une puissance de $p^{-s}$ à constante près.

\subsection{Réduction au cas supercuspidal}

Si $\pi$ est une représentation admissible (non nécessairement irréductible) de $G$, les assertions du théorème font sens pour $\pi$ et $\tilde{\pi}$, mais peuvent être fausse si $\pi$ n'est pas irréductible.

Supposons le théorème vrai pour $\pi$ et $\tilde{\pi}$. Soit $\sigma$ une sous-représentation irréductible de $\pi$. Alors les coefficients de $\sigma$ sont de la forme $<\pi(g)v,\tilde{v}>$ avec $v\in V$ et $\tilde{v} \in \tilde{V}$. Cependant, toutes ces fonctions ne sont pas des coefficients de $\sigma$. On en déduit la
\begin{proposition}
\label{comp_ind1}
Il existe des polynômes $R$ et $\tilde{R}$ en $p^{-s}$ tel que
\begin{align}
L(s,\sigma)&=R(p^{-s})L(s,\pi), \\
L(s,\tilde{\sigma})&=\tilde{R}(p^{-s})L(s,\tilde{\pi}).
\end{align}
De plus,
\begin{equation}
\gamma(s,\sigma,\psi)=\gamma(s,\pi,\psi).
\end{equation}
\end{proposition}

Soit $P$ un sous-groupe parabolique propre maximal de $G$ et $U$ son radical unipotent alors $P/U \simeq G' \times G''$, où l'on note $G'=GL_{n'}(\mathbb{Q}_p)$ et $G''=GL_{n''}(\mathbb{Q}_p)$.

Soit $\sigma'$ (respectivement $\sigma''$) une représentation admissible de $G'$ (respectivement $G''$). On ne les suppose pas irréductible, on suppose cependant qu'ils admettent des caractères centraux $\omega'$ et $\omega''$. Alors $\sigma' \boxtimes \sigma''$ est naturellement une représentation de $P/U$, donc une représentation de $P$ triviale sur $U$.
\begin{proposition}
\label{comp_ind2}
Notons $\pi = Ind_P^G(\sigma' \boxtimes \sigma'')$. Supposons le théorème vrai pour $\sigma'$ et $\sigma''$. Alors le théorème est vrai pour $\pi$. De plus, on a
\begin{align}
L(s,\pi)&=L(s,\sigma')L(s,\sigma''), \\
L(s,\tilde{\pi})&=L(s,\tilde{\sigma}')L(s,\tilde{\sigma}''), \\
\epsilon(s,\pi,\psi)&=\epsilon(s,\sigma',\psi)\epsilon(s,\sigma'',\psi).
\end{align}
\end{proposition}

\begin{proof}
On notera $M'=M_{n'}(\mathbb{Q}_p)$ et $M''=M_{n''}(\mathbb{Q}_p)$. Soit $f$ un coefficient de $\pi$, $\phi \in \mathcal{S}$ et $s \in \mathbb{C}$.

L'espace vectoriel $V$ sur lequel $\pi$ agit est l'espace des fonctions $v : G \rightarrow W$ localement constante qui vérifient
\begin{equation}
v(pg)=\delta_P^{\frac{1}{2}}(p)(\sigma' \boxtimes \sigma'')(p)v(g),
\end{equation}
où $\delta_P$ est le caractère modulaire de $P$ et $W$ est l'espace vectoriel sur lequel $\sigma' \boxtimes \sigma''$ agit.

Le coefficient $f$ est alors de la forme
\begin{align}
f(g)&=<\pi(g)v,\tilde{v}> \\
&= \int_K <v(kg),\tilde{v}(k)>_W dk.
\end{align}

Posons $t=s+\frac{1}{2}(n-1)$, $t'=s+\frac{1}{2}(n'-1)$ et $t''=s+\frac{1}{2}(n''-1)$. L'intégrale zêta est donc
\begin{equation}
\zeta(f,\phi,s)=\int_G \phi(g)|\det g|_p^t \int_K <v(kg),\tilde{v}(k)>dk dg.
\end{equation}
On échange l'ordre d'intégration et on fait le changement de variables $g \mapsto k^{-1}g$, on obtient
\begin{equation}
\label{integrale1}
\int_K \int_G \phi(k^{-1}g)|\det g|^t<v(g),\tilde{v}(k)>dg dk.
\end{equation}
On utilise la décomposition de Cartan pour écrire $g \in G$ sous la forme $g = \begin{pmatrix} 
g' & u \\
0 & g'' 
\end{pmatrix} k'$, où $g' \in G'$, $g'' \in G''$, $u \in U$ et $k' \in K$. On peut alors décomposer la mesure de Haar de $G$ en fonction des mesures de Haar de $G'$, $G''$, $U$ et $K$. En effet,
\begin{equation}
dg = |\det g'|^{-n''}dg'dg''dudk'.
\end{equation}
L'expression (\ref{integrale1}) devient
\begin{equation}
\label{integrale2}
\begin{split}
\int_K \int_{G' \times G'' \times U \times K} &\phi(k^{-1}\begin{pmatrix} 
g' & u \\
0 & g'' 
\end{pmatrix} k') |\det g'|^{t'}|\det g''|^{t''} \\
&<(\sigma'(g') \boxtimes \sigma''(g''))v(k'), \tilde{v}(k)> dg' dg'' du dk' dk.
\end{split}
\end{equation}

Le facteur $<(\sigma'(g') \boxtimes \sigma''(g''))v(k'), \tilde{v}(k)>$ est un coefficient de $\sigma' \boxtimes \sigma''$, donc est une combinaison linéaire de produits de coefficients de $\sigma'$ et de coefficients de $\sigma''$ :
\begin{equation}
<(\sigma'(g') \boxtimes \sigma''(g''))v(k'), \tilde{v}(k)> = \sum_{i=1}^l \lambda_i(k,k')f_i'(g')f_i''(g''),
\end{equation}
où les fonctions $\lambda_i : K \times K \rightarrow \mathbb{C}$ sont localement constante et les $f_i'$ (respectivement $f_i''$) sont des coefficients de $\sigma'$ (respectivement $\sigma''$).

D'autre part, la fonction
\begin{equation}
(x' \in M', x'' \in M'') \mapsto \int_U \phi(k^{-1}\begin{pmatrix} 
x' & u \\
0 & x'' 
\end{pmatrix} k') du
\end{equation}
est un élément de l'espace de Schwartz $\mathcal{S}(M' \times M'')$. On peut donc l'écrire sous la forme
\begin{equation}
\label{u_integrale}
\int_U \phi(k^{-1}\begin{pmatrix} 
x' & u \\
0 & x'' 
\end{pmatrix} k') du = \sum_{j=1}^{l'} \mu_j(k, k')\phi_j'(x')\phi_j''(x''),
\end{equation}
où les $\mu_j$ sont localement constantes et $\phi_j' \in \mathcal{S}(M')$ (respectivement $\phi_j'' \in \mathcal{S}(M'')$).

En remplaçant ces expressions dans l'intégrale (\ref{integrale2}), on trouve
\begin{equation}
\label{zetainduite}
\zeta(f, \phi, t) = \sum_{i,j=1}^{l,l'} \int_{K \times K} \lambda_i(k,k')\mu_j(k,k') dk dk' \zeta(f_i',\phi_j',t') \zeta(f_i'',\phi_j'',t'').
\end{equation}

D'après les hypothèses faites sur $\sigma'$ et $\sigma''$, les intégrales définissant les $\zeta(f_i',\phi_j',t')$ (respectivement $\zeta(f_i'',\phi_j'',t'')$) sont absolument convergentes pour $Re(s)$ assez grande. Ce qui justifie à posteriori les calculs que l'on vient de faire et prouve la partie (1) du théorème pour $\pi$.

D'après (\ref{zetainduite}) et les hypothèses faites sur $\sigma'$ et $\sigma''$, on obtient la relation
\begin{equation}
\zeta(f,\phi,s)=\sum_{i,j=1}^{l,l'}c_{i,j}\Xi(f_i',\phi_j',s)L(s,\sigma')\Xi(f_i'',\phi_j'',s)L(s,\sigma'').
\end{equation}
Ce qui prouve la partie (2) du théorème pour $\pi$.

Passons à la partie (4) du théorème. La valeur $\zeta(\check{f}, \hat{\phi}, t)$ s'obtient en remplaçant $f$ par $\check{f}$, ce qui remplace les $f_i'$ et $f''_i$ en $\check{f}_i'$ et $\check{f}_i''$, et $\phi$ en $\hat{\phi}$. Voyons maintenant comment ce dernier changement affecte l'intégrale. Montrons que l'équation (\ref{u_integrale}) se transforme en
\begin{equation}
\int_U \hat{\phi}(k'^{-1}\begin{pmatrix} 
x' & u \\
0 & x'' 
\end{pmatrix} k) du = \sum_{j=1}^{l'} \mu_j(k, k')\hat{\phi}_j'(x')\hat{\phi}_j''(x'').
\end{equation}
En effet, 
\begin{align}
\int_U \hat{\phi}(k'^{-1}\begin{pmatrix} 
x' & u \\
0 & x'' 
\end{pmatrix} k) du &= \int_U \int_{M_n} \phi(k^{-1}xk')\psi(Tr(\begin{pmatrix} 
x_1 & x_2 \\
x_3 & x_4 
\end{pmatrix}\begin{pmatrix} 
x' & u \\
0 & x'' 
\end{pmatrix})dxdu \\
&= \int \phi(k^{-1}\begin{pmatrix} 
x_1 & x_2 \\
0 & x_4 
\end{pmatrix}k')\psi(x_1x'+x_4x'')dx_1dx_2dx_4 \\
&= \sum_{j=1}^{l'} \mu_j(k, k')\hat{\phi}_j'(x')\hat{\phi}_j''(x'').
\end{align}
La première égalité s'obtient en considérant la transformée de Fourier en les variables $(x_3, u)$. La dernière s'obtient en appliquant la transformée de Fourier sur $M'\times M''$ à l'équation (\ref{u_integrale}).

Ces considérations nous donnent une égalité similaire à (\ref{zetainduite}),
\begin{equation}
\zeta(\check{f},\phi,1-s+\frac{1}{2}(n-1))=\sum_{i,j=1}^{l,l'}c_{i,j}\Xi(\check{f}_i',\hat{\phi}_j',1-s)L(1-s,\tilde{\sigma}')\Xi(\check{f}_i'',\hat{\phi}_j'',1-s)L(1-s,\tilde{\sigma}'').
\end{equation}
On obtient ainsi l'équation fonctionnelle
\begin{equation}
\tilde{\Xi}(\check{f}, \hat{\phi}, 1-s)=\epsilon(s, \sigma', \psi)\epsilon(s,\sigma'',\psi)\Xi(f,\phi,s),
\end{equation}
on en déduit que $\epsilon(s,\pi,\psi)=\epsilon(s, \sigma', \psi)\epsilon(s,\sigma'',\psi)$ et la partie (4) du théorème pour $\pi$.

Il ne reste plus qu'à prouver la partie (3). Il suffit de montrer que si l'on fixe $\phi' \in \mathcal{S}(M')$, $\phi'' \in \mathcal{S}(M'')$ et $f$ (respectivement $f'$) coefficient de $\sigma'$ (respectivement $\sigma''$) alors il existe $\phi \in \mathcal{S}(M)$ et $f$ coefficient de $\pi$ tel que
\begin{equation}
\zeta(f,\phi,t)=\zeta(f',\phi',t')\zeta(f'',\phi'',t'').
\end{equation}
En effet, le calcul du produit des fonctions zêta $\zeta(f',\phi',t')\zeta(f'',\phi'',t'')$ donne
\begin{equation}
\label{zetaprod}
\int_{G' \times G''} \phi'(g')\phi''(g'')f'(g')f''(g'')|\det g'|_p^{t'}|\det g''|_p^{t''}dg'dg''.
\end{equation}
On choisit alors $\phi \in \mathcal{S}(M)$ de la forme $\begin{pmatrix} 
x' & u \\
v & x'' 
\end{pmatrix} \mapsto \phi'(x')\phi''(x'')\phi_0(u)\phi_1(v)$, où $\phi_1 \in \mathcal{S}(M_{n'',n'})$ vérifie $\phi_1(0)=1$ et $\phi_0 \in \mathcal{S}(M_{n',n''})$ est d'intégrale $1$. Avec ce choix, on a
\begin{equation}
\int_U \phi(\begin{pmatrix} 
g' & u \\
0 & g'' 
\end{pmatrix})=\phi'(g')\phi''(g'').
\end{equation}
De plus, il existe une fonction localement constante $\eta : K \rightarrow \mathbb{C}$ telle que
\begin{equation}
\int_{U \times K} \phi(\begin{pmatrix} 
g' & u \\
0 & g'' 
\end{pmatrix}k)\eta(k)dudk = \phi(g')\phi(g'').
\end{equation}
On pose aussi $f(g)=\delta_P^{\frac{1}{2}}(\begin{pmatrix} 
g' & u \\
0 & g'' 
\end{pmatrix})\eta(k)f(g')f(g'')$, alors $f$ est bien un coefficient de $\pi$. De plus, en intégrant sur $U \times K$ l'expression (\ref{zetaprod}) devient
\begin{equation}
\int_G \phi(\begin{pmatrix} 
g' & u \\
0 & g'' 
\end{pmatrix}k)f(g)|\deg g|_p^t\delta_P(\begin{pmatrix} 
g' & u \\
0 & g'' 
\end{pmatrix})dg'dg''dudk,
\end{equation}
qui est bien $\zeta(f,\phi,t)$. Ce qui termine la preuve de la proposition.
\end{proof}