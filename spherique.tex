\subsection{Représentation sphérique}

L'importance des représentations sphériques vient du fait que dans la théorie globale presque toutes les composantes locales sont des représentations sphériques.

\begin{definition}
Une représentation sphérique de $G$ est une représentation $(\pi, V)$ admissible irréductible de $G$ telle qu'il existe un vecteur $v \in V$ non nul invariant par $K$.
\end{definition}

Notons $P_0 \subset G$ l'ensemble des matrices triangulaires supérieures. Soit $\chi_1, ..., \chi_n$ des caractères non ramifiés de $\mathbb{Q}_p^\times$, on note $\chi = \chi_1 \boxtimes ... \boxtimes \chi_n$ le caractère de $P_0$ trivial sur son radical unipotent. On pose $\pi = Ind_{P_0}^G(\chi)$.

Alors $\pi$ contient un vecteur invariant par $K$, $\varphi_0$ définie par $\varphi_0(pk)=\delta_P(p)^{\frac{1}{2}}\chi(p)$. On considère le coefficient $f_0$ défini par $f_0(g)=<\pi(g)\varphi_0, \tilde{\varphi}_0>$, où $\tilde{\varphi}_0$ est un vecteur de $\tilde{\pi}$ invariant par $K$ défini par $\tilde{\varphi}_0(pk)=\delta_P(p)^{\frac{1}{2}}\chi(p)^{-1}$. C'est un coefficient d'une représentation admissible irréductible que l'on note $\pi_0$, cette représentation est sphérique.

De plus, toutes les représentations sphériques sont de cette forme (à isomorphisme près).

\begin{lemme}
On note $\phi_0$ l'indicatrice de $M_n(\mathbb{Z}_p)$. Alors
\begin{equation}
\zeta(f_0, \phi_0, s) = \prod_{i=1}^n L(s, \chi_i).
\end{equation}
\end{lemme}

\begin{proof}
Comme $\varphi_0$ et $\tilde{\varphi}$ sont $K$-invariant, on en déduit que
\begin{equation}
f_0(g)=\int_K \varphi_0(kg)\tilde{\varphi}_0(k)dk=\varphi_0(g).
\end{equation}
De plus,
\begin{align}
\zeta(f_0, \phi_0, s) &= \int_G \phi_0(g)\varphi_0(g)|\det g|^s dg \\
&= \int_P \int_K \phi_0(pk) \varphi_0(pk) |\det p|^s \delta_P(p)^{-\frac{1}{2}}dp dk \\
&= \prod_{i=1}^n \int_{\mathbb{Z}_p}|a_i|^s\chi_i(a_i)da_i \\
&= \prod_{i=1}^n L(s, \chi_i).
\end{align}
\end{proof}

On en déduit le calcul de la fonction $L$ d'une représentation sphérique.
\begin{proposition}
En reprenant les notations précédentes,
\begin{equation}
L(s, \pi_0) = \prod_{i=1}^n L(s+\frac{1}{2}(n-1), \chi_i).
\end{equation}
\end{proposition}

\subsection{Représentation de carré intégrable}

Cette partie provient de \cite{goldfeld-hundley}.
\begin{definition}
Une représentation admissible irréductible $(\pi, V)$ de $G$ est dite de carré intégrable si son caractère central est unitaire et $\int_{G/\mathbb{Q}_p^\times} |f(g)|^2 dg < \infty$ pour tout coefficient $f$ de $\pi$.
\end{definition}

On décrit maintenant la classification de Bernstein-Zelevinsky des représentations admissibles irréductibles de carré intégrable. Soit $r,d$ des entiers $> 0$, soit $\tau$ une représentation supercuspidale de $GL(r, \mathbb{Q}_p)$ de caractère central unitaire. On note $P_{r,d}$ le sous-groupe parabolique de $GL(rd, \mathbb{Q}_p)$ des matrices triangulaires supérieures par bloc où les blocs diagonaux sont de tailles $r \times r$. Alors
\begin{equation}
Ind_{P_{r,d}}^{GL(rd, \mathbb{Q}_p)}(|.|_p^{\frac{1-d}{2}}\tau \boxtimes ... \boxtimes |.|_p^{\frac{d-1}{2}}\tau)
\end{equation}
admet un unique quotient irréductible que l'on note $\tau'$. La représentation $\tau'$ est admissible irréductible de carré intégrable. De plus, toutes les représentations admissibles irréductibles de carré intégrable sont de cette forme (à isomorphisme près) avec $n=rd$.

\begin{proposition}
En reprenant les notations précédentes,
\begin{equation}
L(s, \tau') = \begin{cases}
    L(s,|.|_p^{\frac{1-n}{2}}\tau),& \text{si } r=1\\
    1,              & \text{si } r > 1.
\end{cases}
\end{equation}
\end{proposition}

\begin{proof}
Si $r > 1$, par la compatibilité de l'induction,
\begin{equation}
L(s, Ind_{P_{r,d}}^{GL(rd, \mathbb{Q}_p)}(|.|_p^{\frac{1-d}{2}}\tau \boxtimes ... \boxtimes |.|_p^{\frac{d-1}{2}}\tau)) = \prod_{i=0}^{d-1} L(s, |.|_p^{\frac{1-d}{2}+i}\tau)=1,
\end{equation}
la dernière égalité vient du calcul de la fonction $L$ d'une représentation supercuspidale. On en déduit immédiatement que $L(s,\tau')=1$, d'après la proposition \ref{lfunsupercusp}.

Supposons maintenant que $r=1$. Soit $\phi \in \mathcal{S}$, $f$ un coefficient de $\tau'$ et $s \in \mathbb{C}$. Alors, d'après l'équation fonctionnelle,
\begin{equation}
\zeta(\check{f}, \hat{\phi}, s) = \gamma(s, \tau', \lambda)\zeta(f, \phi, s).
\end{equation}
Les propriétés du facteur $\gamma$ et le calcul pour $GL_1(\mathbb{Q}_p)$ donnent
\begin{equation}
\gamma(s, \tau', \lambda) = \prod_{i=0}^{n-1} \frac{1-\tau(p)p^{-s-\frac{n-1}{2}+i}}{1-\tau(p)^{-1}p^{s-1+\frac{n-1}{2}-i}}.
\end{equation}
Ce qui nous permet d'en déduire immédiatement que
\begin{equation}
\label{carreintegrable1}
\prod_{i=0}^{n-1}(1-\tau(p)p^{-s-\frac{n-1}{2}+i})\zeta(f, \phi, s) =
\prod_{i=1}^{n}(1-\tau(p)^{-1}p^{s+\frac{n-1}{2}-i})\zeta(\check{f}, \hat{\phi}, 1-s).
\end{equation}
On utilise maintenant le fait que
\begin{equation}
(1-\tau(p)^{-1}p^{s+\frac{n-1}{2}-i}) = -\tau(p)^{-1}p^{s+\frac{n-1}{2}-i}(1-\tau(p)p^{-s-\frac{n-1}{2}+i}),
\end{equation}
pour simplifier l'équation (\ref{carreintegrable1}), ce qui nous donne
\begin{equation}
\label{carreintegrable2}
(1-\tau(p)p^{-s-\frac{n-1}{2}})\zeta(f, \phi, s) = (-\tau(p))^{-n+1}
p^{(s-\frac{1}{2})(n-1)}(1-\tau(p)^{-1}p^{s-\frac{n-1}{2}})\zeta(\check{f}, \hat{\phi}, s).
\end{equation}

Pour les représentations de carré intégrable, on dispose d'une information plus précise sur de domaine de convergence de l'intégrale zêta.
\begin{lemme}[\cite{goldfeld-hundley}]
L'intégrale définissant la fonction zêta pour $\tau'$,
\begin{equation}
\zeta(f, \phi, s) = \int_{G} \phi(g)f(g)|\det g|_p^s dg
\end{equation}
est absolument convergente pour $Re(s) > 0$.
\end{lemme}

Ce lemme nous permet d'en déduire que les deux membres de l'équation (\ref{carreintegrable2}) sont holomorphes dans les régions $Re(s) > 0$ et $Re(s) < 1$. Ceci nous permet voir immédiatement que $L(s, \tau') = (1-\tau(p)p^{-s-\frac{n-1}{2}})$ ou $1$. Il nous suffit donc de montrer que $L(s, \tau') \neq 1$ pour avoir le résultat voulu.

On choisit $\phi \in \mathbb{S}$ et $f$ un coefficient de $\pi$ tels que $\zeta(\check{f}, \hat{\phi}, 1-s)$ est une constante non nulle. En effet, on choisit un compact $C$ de $G^0$ suffisamment grand tel qu'il existe un coefficient $f$ de $\pi$ non identiquement nul sur $C$ et on pose alors $\phi(g) = \overline{f(g)}$ si $g \in C$ et $\phi(g)=0$ sinon. Alors le membre de droite de l'équation (\ref{carreintegrable2}) ne s'annule pas en $s=-\frac{n-1}{2}$. On en déduit que $\zeta(f, \phi, s)$ doit avoir un pôle en $s=-\frac{n-1}{2}$, d'où $L(s,\tau')\neq 1$.
\end{proof}