\section{Fonctions zêta sur $GL_n(\mathbb{A})$}

Dans cette dernière section, on présente la théorie globale de Godement-Jacquet \cite{godement-jacquet} et on termine par la définition de la fonction $L$ attachée à une représentation cuspidale. Dans la suite, on note $G = GL_n(\mathbb{Q})$, $G_\mathbb{A}=GL_n(\mathbb{A})$. On pose $K = O_n(\mathbb{R}) \times \prod_p GL_n(\mathbb{Z}_p)$, c'est un sous-groupe compact maximal de $G_\mathbb{A}$.

\subsection{Formes cuspidales}
On commence par donner la définition des formes automorphes (et cuspidales), on renvoie à \cite{bump} et \cite{goldfeld-hundley} pour plus de détails.

On fixe un caractère unitaire $\omega : \mathbb{A}^\times/\mathbb{Q}^\times \rightarrow \mathbb{S}^1$.

\begin{definition}
Une forme automorphe de caractère central $\omega$ est une fonction $\varphi : G_\mathbb{A} \rightarrow \mathbb{C}$ lisse et $G$-invariante qui vérifie de plus :
\begin{itemize}
\item $\varphi$ est $K$-finie à droite,
\item $\varphi$ est $Z(U(\mathfrak{g}))$-finie,
\item \begin{equation}
\varphi(zg) = \omega(z)\varphi(g) \quad \forall g \in G_\mathbb{A}, z \in \mathbb{A}^\times,
\end{equation}
\item $\varphi$ est à croissance modérée.
\end{itemize}

On note $\mathcal{A}(G_\mathbb{A}, \omega)$ l'espace des formes automorphes de caractère central $\omega$.
\end{definition}

On rajoute aussi une condition d'annulation dont on aura besoin pour la preuve de l'équation fonctionnelle. Ce qui donne la
\begin{definition}
Une forme cuspidale $\varphi$ de caractère central $\omega$ est une forme automorphe de caractère central $\omega$ qui vérifie de plus les conditions :
\begin{equation}
\int_{U \backslash U_\mathbb{A}} \varphi(ug) du = 0
\end{equation}
pour tout radical unipotent $U$ d'un sous-groupe parabolique propre de $G_\mathbb{A}$ et tout $g \in G_\mathbb{A}$.

On note $\mathcal{A}_0(G_\mathbb{A}, \omega)$ l'espace des formes cuspidales de caractère central $\omega$.
\end{definition}

L'espace de Schwartz de $M_n(\mathbb{A})$ est, par définition, $\mathcal{S}(M_n(\mathbb{A}))=\otimes_v^{'} \mathcal{S}(M_n(\mathbb{Q}_v)=\left\lbrace \phi = \otimes \phi_v, \phi_v \in \mathcal{S}(M_n(\mathbb{Q}_v)), \phi_v = \mathbbm{1}_{\mathbb{Z}_v} \text{ sauf pour un nombre fini de v} \right\rbrace$.

Pour $\varphi \in \mathcal{A}_0(G_\mathbb{A}, \omega)$, $\phi \in \mathcal{S}(M_\mathbb{A})$ et $s \in \mathbb{C}$, on pose
\begin{equation}
\zeta(\varphi, \phi, s) = \int_{G_\mathbb{A}} \phi(g) \varphi(g) |\det g|_\mathbb{A}^s dg,
\end{equation}
où $dg = \otimes_v dg_v$ est une mesure de Haar sur $GL_n(\mathbb{A})$ et $|.|_\mathbb{A} = \prod_v |.|_v$ est la valeur absolue adélique.

Notons $G^0_\mathbb{A}=\left\lbrace g \in G_\mathbb{A}, |\det g|_\mathbb{A} = 1 \right\rbrace$. Comme $\mathbb{R}_{> 0} \subset \mathbb{A}^\times=Z(G_\mathbb{A})$, l'application $|\det|_\mathbb{A} : G_\mathbb{A} \rightarrow \mathbb{R}_{> 0}$ est surjective de noyau $G^0_\mathbb{A}$.

La factorisation $G_\mathbb{A} = \mathbb{R}_{> 0}G^0_\mathbb{A}$ permet d'obtenir que
\begin{align}
\zeta(\varphi, \phi, s) &= \int_0^\infty \int_{G^0_\mathbb{A}} \phi(tg) \omega(t) \varphi(g) t^{ns} dg \frac{dt}{t} \\
&= \int_0^\infty \int_{G \backslash G^0_\mathbb{A}} \sum_{x \in G}{\phi(txg)} \varphi(g) \omega(t) t^{ns} dg \frac{dt}{t}.
\end{align}

Comme dans la preuve de l'équation fonctionnelle de la fonction zêta de Riemann, on scinde l'intégrale en $1$ dans le but de faire apparaître une symétrie. Autrement dit,
\begin{equation}
\begin{split}
\zeta(\varphi, \phi, s) &= \int_0^1 \int_{G \backslash G^0_\mathbb{A}} \sum_{x \in G}{\phi(txg)} \varphi(g) \omega(t) t^{ns} dg \frac{dt}{t} \\
&+ \int_1^\infty \int_{G \backslash G^0_\mathbb{A}} \sum_{x \in G}{\phi(txg)} \varphi(g) \omega(t) t^{ns} dg \frac{dt}{t}.
\end{split}
\end{equation}

La seconde intégrale converge absolument pour tout $s \in \mathbb{C}$, c'est une fonction entière. Pour la première intégrale, on fait le changement de variable $t \mapsto t^{-1}$, ce qui donne
\begin{equation}
\int_1^\infty \int_{G \backslash G^0_\mathbb{A}} \sum_{x \in G}{\phi(t^{-1}xg)} \varphi(g) \omega^{-1}(t) t^{-ns} dg \frac{dt}{t}.
\end{equation}

On va maintenant utiliser la formule de Poisson sur $M_n(\mathbb{A})$. Commençons par définir la transformée de Fourier adélique, on pose
\begin{equation}
\hat{\phi}(y) = \int_{M_n(\mathbb{A})} \phi(x)\psi(Tr(xy))dx, y \in M_n(\mathbb{A}),
\end{equation}
où $dx$ est une mesure de Haar normalisée pour avoir la formule $\hat{\hat{\phi}}(x)=\phi(-x)$ et $\psi : \mathbb{A}/\mathbb{Q} \rightarrow \mathbb{C}$ est un caractère non trivial.

La formule de Poisson donne, pour la fonction $x \mapsto \phi(t^{-1}xg)$ :
\begin{equation}
\sum_{x \in M_n(\mathbb{Q})} \phi(t^{-1}xg) = t^{n^2}\sum_{x \in M_n(\mathbb{Q})} \hat{\phi}(txg^{-1}),
\end{equation}
on se rappelle que $g \in G^0_\mathbb{A}$, donc $|\det g|_\mathbb{A}=1$. On scinde la somme selon le rang de la matrice et on obtient :
\begin{equation}
\begin{split}
\sum_{x \in G} \phi(t^{-1}xg) &= t^{n^2}\sum_{x \in G} \hat{\phi}(txg^{-1}) \\
&+ \sum_{r < n, rg(x)=r} \left( t^{n^2}\hat{\phi}(txg^{-1}) - \phi(t^{-1}xg)\right).
\end{split}
\end{equation}

La contribution de la dernière somme s'avèrera nulle. Ce qui nous permet d'en déduire la
\begin{proposition}
Si $\varphi \in \mathcal{A}_0(G_\mathbb{A}, \omega)$ et $\phi \in \mathcal{S}(M_n(\mathbb{A})$, la fonction $\zeta(\varphi, \phi, .)$ peut être prolongée en une fonction entière et vérifie l'équation fonctionnelle
\begin{equation}
\label{eqcusp}
\zeta(\varphi, \phi, s) = \zeta(\check{\varphi}, \hat{\phi}, n-s),
\end{equation}
où $\check{\varphi}(g)=\varphi(g^{-1})$.
\end{proposition}

\begin{proof}
Il suffit de prouver que la contribution dans la formule de Poisson des matrices de rang $r < n$ est effectivement nulle. On considère l'action de $G$ par translation à droite sur l'ensemble des matrices de rang $r$. Chaque orbite contient un représentant de la forme $\begin{pmatrix} 
* & 0 \\
* & 0 
\end{pmatrix}$, on note $X$ l'ensemble des matrices de cette forme. On pose $P$ le sous-groupe parabolique de $G$ des matrices de la forme $\begin{pmatrix} 
* & 0 \\
* & * 
\end{pmatrix}$ et $U$ son radical unipotent.

On réécrit la somme sur les matrices de rang $r$ grâce au système de représentant $X$,
\begin{equation}
\sum_{rg(x)=r}\phi(xg) = \sum_{\gamma \in P \backslash G} \sum_{x \in X} \phi(x \gamma g).
\end{equation}
On en déduit que la contribution des matrices de rang $r$ dans la seconde intégrale est
\begin{equation}
\int_{P \backslash G^0_\mathbb{A}} \sum_{x \in X}{\phi(t^{-1}x g)} \varphi(g) dg.
\end{equation}
De plus, on remarque que, $xu=x$, pour tout $x \in X$ et $u \in U_\mathbb{A}$. Ce qui nous permet de réécrire cette intégrale sous la forme
\begin{equation}
\int_{PU_\mathbb{A} \backslash G^0_\mathbb{A}} \sum_{x \in X}{\phi(t^{-1}xg)} \int_{U \backslash U_\mathbb{A}} \varphi(ug) du dg.
\end{equation}
Cette dernière intégrale s'annule, car $f$ est cuspidale. On montre de même de l'intégrale correspondant au terme en $\hat{\phi}$ sur les matrices de rang $r < n$ s'annule aussi. Ce qui nous donne, grâce à la formule de Poisson et le raisonnement précédent, la formule
\begin{equation}
\begin{split}
\zeta(\varphi, \phi, s) &= \int_1^\infty \int_{G \backslash G^0_\mathbb{A}} \sum_{x \in G}{\hat{\phi}(txg^{-1})} \varphi(g) \omega^{-1}(t) t^{n(n-s)} dg \frac{dt}{t} \\
&+ \int_1^\infty \int_{G \backslash G^0_\mathbb{A}} \sum_{x \in G}{\phi(txg)} \varphi(g) \omega(t) t^{ns} dg \frac{dt}{t},
\end{split}
\end{equation}
ce qui démontre l'équation fonctionnelle en effectuant le changement de variable $g \mapsto g^{-1}$ dans la première intégrale.
\end{proof}

\subsection{Représentations automorphes}

L'espace des formes cuspidales $\mathcal{A}_0(G_\mathbb{A}, \omega)$ est stable par l'action de $U(\mathfrak{g})$ par opérateurs différentiels et par translation à droite de $O_n(\mathbb{R})$ et $GL_n(\mathbb{A}_f)$, c'est un $(\mathfrak{g}, O_n(\mathbb{R})) \times GL_n(\mathbb{A}_f)$-module.

Un coefficient $f$ de $\mathcal{A}_0(G_\mathbb{A}, \omega)$ est de la forme
\begin{equation}
f(g) = <\pi(g)\varphi, \tilde{\varphi}> = \int_{\mathbb{A}^\times G \backslash G_\mathbb{A}} \varphi(hg) \tilde{\varphi}(h) dh,
\end{equation}
où $\varphi \in \mathcal{A}_0(G_\mathbb{A}, \omega)$ et $\tilde{\varphi} \in \mathcal{A}_0(G_\mathbb{A}, \omega^{-1})$.

Pour un coefficient $f$ de $\mathcal{A}_0(G_\mathbb{A}, \omega)$, $\phi \in \mathcal{S}(M_\mathbb{A})$ et $s \in \mathbb{C}$, on pose
\begin{equation}
\zeta(f, \phi, s) = \int_{G_\mathbb{A}} \phi(g) f(g) |\det g|_\mathbb{A}^s dg.
\end{equation}

On peut déduire les propriétés de cette fonction zêta grâce à ce que l'on vient de faire pour les formes cuspidales. Plus précisément, on a
\begin{align}
\zeta(f, \phi, s) &= \int_{G_\mathbb{A}}\phi(g)\int_{\mathbb{A}^\times G \backslash G_\mathbb{A}} \varphi(hg) \tilde{\varphi}(h) dh |\det g|_\mathbb{A}^s dg \\
&= \int_{\mathbb{A}^\times G \backslash G_\mathbb{A}} \tilde{\varphi}(h) \int_{G_\mathbb{A}}\phi(h^{-1}g)\varphi(g)|\det g|_\mathbb{A}^s dg |\det h|_\mathbb{A}^{-s} dh \\
&= \int_{\mathbb{A}^\times G \backslash G_\mathbb{A}} \tilde{\varphi}(h) \zeta(\varphi, \phi(h^{-1}.), s)|\det h|_\mathbb{A}^{-s} dh,
\end{align}
où la deuxième égalité s'obtient grâce au changement de variable $g \mapsto h^{-1}g$. Ceci nous permet de démontrer la
\begin{proposition}
Si $f$ est un coefficient de $\mathcal{A}_0(G_\mathbb{A}, \omega)$ et $\phi \in \mathcal{S}(M_\mathbb{A})$, la fonction $\zeta(f, \phi, .)$ peut être prolongée en une fonction entière et vérifie l'équation fonctionnelle
\begin{equation}
\zeta(f, \phi, s) = \zeta(\check{f}, \hat{\phi}, n-s),
\end{equation}
où $\check{f}(g) = f(g^{-1})$.
\end{proposition}

\begin{proof}
On utilise l'équation fonctionnelle (\ref{eqcusp}) et le fait que la transformée de Fourier de $\phi(h^{-1}.)$ est $|\det h|_\mathbb{A}^n\hat{\phi}(.h)$,
\begin{align}
\zeta(f, \phi, s) &= \int_{\mathbb{A}^\times G \backslash G_\mathbb{A}} \tilde{\varphi}(h) \zeta(\check{\varphi}, \hat{\phi}(.h), n-s)|\det h|_\mathbb{A}^{n-s} dh \\
&= \int_{\mathbb{A}^\times G \backslash G_\mathbb{A}} \tilde{\varphi}(h) \int_{G_\mathbb{A}}\hat{\phi}(gh)\varphi(g^{-1})|\det g|_\mathbb{A}^{n-s} dg |\det h|_\mathbb{A}^{n-s} dh.
\end{align}
On effectue maintenant le changement de variable $g \mapsto gh^{-1}$, ce qui donne
\begin{equation}
\int_{\mathbb{A}^\times G \backslash G_\mathbb{A}} \tilde{\varphi}(h) \int_{G_\mathbb{A}}\hat{\phi}(g)\varphi(hg^{-1})|\det g|_\mathbb{A}^{n-s} dg dh,
\end{equation}
qui est bien $\zeta(\check{f}, \hat{\phi}, n-s)$.
\end{proof}

Si l'on combine cette proposition avec les résultats locaux, on peut construire la fonction $L$ attachée à une représentation cuspidale irréductible.
\begin{definition}
Une représentation cuspidale est un $(\mathfrak{g}, O_n(\mathbb{R})) \times GL_n(\mathbb{A}_f)$-module qui est isomorphe à un sous-quotient de $\mathcal{A}_0(G_\mathbb{A}, \omega)$.
\end{definition}

Plus précisément, on montre le
\begin{theoreme}
Soit $\pi$ une représentation cuspidale irréductible.

Le produit $L(s, \pi) = \prod_v L(s, \pi_v)$, qui est défini pour $Re(s) > n$, se prolonge en une fonction entière. De plus, $L(s, \pi)$ vérifie l'équation fonctionnelle
\begin{equation}
L(s,\pi) = \epsilon(s,\pi)L(1-s,\tilde{\pi}),
\end{equation}
où $\epsilon(s,\pi) = \prod_v \epsilon(s, \pi_v, \psi_v)$, avec $\psi$ un caractère non trivial de $\mathbb{A}/\mathbb{Q}$.
\end{theoreme}

\begin{proof}
La représentation $\pi$ se décompose en facteurs locaux,
$\pi \simeq \otimes_v^{'} \pi_v$, où $\pi_v$ est une représentation admissible irréductible de $GL_n(\mathbb{Q}_v)$ (un $(\mathfrak{g}, O_n(\mathbb{R}))$-module irréductible pour la place archimédienne) et pour presque toutes les places $\pi_v$ est sphérique (contient la représentation unité de $GL_n(\mathbb{Z}_v)$).

D'après les résultats locaux, pour chaque place $v$, il existe un nombre fini $(\phi_{\alpha_v})_{\alpha_v \in I_v}$ d'éléments de $\mathcal{S}(M_v)$ et de coefficient $(f_{\alpha_v})_{\alpha_v \in I_v}$ de $\pi_v$ tel que
\begin{equation}
\sum_{\alpha_v \in I_v} \zeta(f_{\alpha_v}, \phi_{\alpha_v}, s + \frac{1}{2}(n-1)) = L(s, \pi_v).
\end{equation}
De plus, d'après l'équation fonctionnelle locale
\begin{equation}
\sum_{\alpha_v \in I_v} \zeta(\check{f}_{\alpha_v}, \hat{\phi}_{\alpha_v}, 1-s + \frac{1}{2}(n-1)) = \epsilon(s,\pi_v,\psi_v)L(1-s, \tilde{\pi}_v).
\end{equation}

Notons $I = \prod_v I_v$. Pour presque toutes les places $v$, $\pi_v$ est sphérique, $I_v$ est un singleton; donc $I$ est fini.

Pour $\alpha = (\alpha_v) \in I$, on pose
\begin{equation}
\phi_\alpha = \prod_v \phi_{\alpha_v}, \quad f_\alpha = \prod_v f_{\alpha_v}.
\end{equation}
Alors $\phi_\alpha \in \mathcal{S}(M_\mathbb{A})$ et $f_\alpha$ est un coefficient de $\pi$ qui est un sous-quotient de $\mathcal{A}_0(G_\mathbb{A}, \omega)$. De plus,
\begin{equation}
\zeta(f_\alpha, \phi_\alpha, s) = \prod_v \zeta(f_{\alpha_v}, \phi_{\alpha_v}, s).
\end{equation}

On en déduit que
\begin{align}
L(s, \pi) &= \prod_v L(s, \pi_v) = \prod_v \sum_{\alpha_v \in I_v} \zeta(f_{\alpha_v}, \phi_{\alpha_v}, s + \frac{1}{2}(n-1)) \\
&= \sum_{\alpha \in I} \zeta(f_\alpha, \phi_\alpha, s + \frac{1}{2}(n-1))
\end{align}
est une somme finie de fonction zêta, qui chacune se prolonge en une fonction entière. De plus,
\begin{align}
L(s, \pi) &= \sum_{\alpha \in I} \zeta(f_\alpha, \phi_\alpha, s + \frac{1}{2}(n-1)) \\
&= \sum_{\alpha \in I} \zeta(\check{f}_\alpha, \hat{\phi}_\alpha, 1 - s + \frac{1}{2}(n-1)) \\
&= \prod_v \sum_{\alpha_v \in I_v} \zeta(\check{f}_{\alpha_v}, \hat{\phi}_{\alpha_v}, 1-s + \frac{1}{2}(n-1)) \\
&= \prod_v \epsilon(s, \pi_v, \psi_v) L(1-s, \tilde{\pi}_v) \\
&= \epsilon(s, \pi)L(1-s, \tilde{\pi}).
\end{align}
\end{proof}