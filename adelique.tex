\section{Fonctions zêta sur $GL_n(\mathbb{A})$}

Dans cette dernière section, on présente la théorie globale de Godement-Jacquet \cite{godement-jacquet} et on termine par la définition de la fonction $L$ attachée à une représentation cuspidale. On commence par un paragraphe d'introduction sur les adèles.

\subsection{Adèles}

L'anneau des adèles, noté $\mathbb{A}$, est le produit restreint des complétions de $\mathbb{Q}$,
\begin{equation}
\mathbb{A} = \prod_v \mathbb{Q}_v,
\end{equation}
où $v$ décrit les places de $\mathbb{Q}$, $v = \infty$ ou $p$ un nombre premier, le produit restreint étant relatif aux sous-groupes $\mathbb{Z}_p$. Plus exactement,
\begin{equation}
\mathbb{A} = \left\lbrace (x_v), x_v \in \mathbb{Q}_v, x_p \in \mathbb{Z}_p \text{ sauf pour un nombre fini de p} \right\rbrace.
\end{equation}
De plus, $\mathbb{A}$ est un anneau topologique localement compact en prenant comme base d'ouverts les ensembles de la forme
\begin{equation}
U \times \prod_{p \not\in S} \mathbb{Z}_p
\end{equation}
où $S$ est un ensemble fini de places contenant $\infty$ et $U$ est un ouvert de $\prod_{v \in S} \mathbb{Q}_v$ muni de la topologie produit.

Le groupe multiplicatif des idèles, noté $\mathbb{A}^\times$, est
\begin{equation}
\mathbb{A}^\times = \left\lbrace (x_v), x_v \in \mathbb{Q}_v^\times, x_p \in \mathbb{Z}_p^\times \text{ sauf pour un nombre fini de p} \right\rbrace.
\end{equation}
Le groupe $\mathbb{A}^\times$ est groupe topologique localement compact en prenant comme base d'ouverts les ensembles de la forme
\begin{equation}
U \times \prod_{p \not\in S} \mathbb{Z}_p^\times
\end{equation}
où $S$ est un ensemble fini de places contenant $\infty$ et $U$ est un ouvert de $\prod_{v \in S} \mathbb{Q}_v^\times$ muni de la topologie produit. La topologie des idèles n'est pas la topologie induite des adèles.

On plonge l'anneau des rationnels $\mathbb{Q}$ diagonalement dans les adèles $\mathbb{A}$. Soit $q \in \mathbb{Q}$, alors $(q)_v \in \mathbb{A}$, c'est l'adèle dont toutes les composantes sont égales à $q$.

\begin{proposition}[Approximation forte \cite{goldfeld-hundley}]
Un domaine fondamental pour $\mathbb{Q}\backslash \mathbb{A}$ est donné par
\begin{equation}
[0, 1[ \prod_p \mathbb{Z}_p.
\end{equation}
Autrement dit, tout élément $x \in \mathbb{A}$ s'écrit de manière unique $x = q + d$, avec $q \in \mathbb{Q}$ et $d \in \mathbb{A}$ qui vérifie $d_\infty \in [0, 1[$ et $d_p \in \mathbb{Z}_p$ pour tout $p$ premier.
\end{proposition}

L'espace de Schwartz de $\mathbb{A}$, que l'on note $\mathcal{S}(\mathbb{A})$, est l'ensemble des combinaisons linéaires de fonctions $\phi : \mathbb{A} \rightarrow \mathbb{C}$ factorisables $\phi = \prod_v \phi_v$ telles que $\phi_\infty \in \mathcal{S}(\mathbb{R})$, $\phi_p \in \mathcal{S}(\mathbb{Q}_p)$ et $\phi_p = \mathbbm{1}_{\mathbb{Z}_p}$ sauf pour un nombre fini de $p$.

Soit $e : \mathbb{Q}\backslash \mathbb{A} \rightarrow \mathbb{C}^\times$ le caractère non trivial défini par $e(x) = \prod_v e_v(x_v)$ où $e_p(x_p) = e^{2i\pi \lambda_p(x_p)}$ ($\lambda_p$ est définit dans la section \ref{gl1}) et $e_\infty(x_\infty) = e^{-2i\pi x_\infty}$. Pour $\phi = \prod_v \phi_v \in \mathcal{S}(\mathbb{A})$, on définit la transformée de Fourier de $\phi$ par
\begin{equation}
\hat{\phi}(x) = \int_{\mathbb{}}\phi(y)e(-xy)dy,
\end{equation}
où $dy = \otimes dy_v$ avec $dy_p$ la mesure introduite dans la section \ref{gl1} et $dy_\infty$ la mesure de Lebesgue.

\begin{proposition}[\cite{goldfeld-hundley}]
Soit $\phi \in \mathcal{S}(\mathbb{A})$ une fonction $\mathbb{Q}$-périodique. Alors
\begin{equation}
\phi(x) = \sum_{\alpha \in \mathbb{Q}} \hat{\phi}_\alpha e(\alpha x),
\end{equation}
où l'on a posé
\begin{equation}
\hat{\phi}_\alpha = \int_{\mathbb{Q}\backslash \mathbb{A}} \phi(x)e(-\alpha x).
\end{equation}
\end{proposition}

\begin{proposition}[Formule de Poisson]
Soit $\phi \in \mathcal{S}(\mathbb{A})$. Alors
\begin{equation}
\sum_{\alpha \in \mathbb{Q}} \phi(\alpha) = \sum_{\alpha \in \mathbb{Q}} \hat{\phi}(\alpha).
\end{equation}
\end{proposition}

\begin{proof}
Pour $x \in \mathbb{A}$, on pose $h(x) = \sum_{\alpha \in \mathbb{Q}} \phi(\alpha+x)$. Cette somme converge absolument et uniformément sur tout compact. Alors
\begin{equation}
h(x) = \sum_{\alpha \in \mathbb{Q}} \hat{h}_\alpha e(\alpha x).
\end{equation}

Montrons que l'on a $\hat{h}_\alpha = \hat{\phi}(\alpha)$. En effet,
\begin{align}
\hat{h}_\alpha &= \int_{\mathbb{Q}\backslash \mathbb{A}} h(x)e(-\alpha x) \\
&=\int_{\mathbb{Q}\backslash \mathbb{A}} \sum_{\beta \in \mathbb{Q}} \phi(\beta+x)e(-\alpha x) \\
&= \int_{\mathbb{Q}\backslash \mathbb{A}} \sum_{\beta \in \mathbb{Q}} \phi(x)e(-\alpha x)e(\alpha \beta) \\
&= \int_{\mathbb{A}} \phi(x) e(-\alpha x) \\
&= \hat{\phi}(\alpha),
\end{align}
où l'on a utilisé le fait que $e(\alpha \beta) = 1$ car $e$ est trivial sur $\mathbb{Q}$. On a donc l'égalité
\begin{equation}
h(x) = \sum_{\alpha \in \mathbb{Q}} \phi(\alpha+x) = \sum_{\alpha \in \mathbb{Q}} \hat{\phi}_\alpha e(\alpha x).
\end{equation}
En prenant cette égalité en $x=0$, on obtient la formule de Poisson.
\end{proof}

\subsection{Formes cuspidales}
Dans la suite, on note $G = GL_n(\mathbb{Q})$, $G_\mathbb{A}=GL_n(\mathbb{A})$. On pose $K = O_n(\mathbb{R}) \times \prod_p GL_n(\mathbb{Z}_p)$, c'est un sous-groupe compact maximal de $G_\mathbb{A}$.

On commence par donner la définition des formes automorphes (et cuspidales), on renvoie à \cite{bump} et \cite{goldfeld-hundley} pour plus de détails.

On fixe un caractère unitaire $\omega : \mathbb{A}^\times/\mathbb{Q}^\times \rightarrow \mathbb{S}^1$.

\begin{definition}
Une forme automorphe de caractère central $\omega$ est une fonction $\varphi : G_\mathbb{A} \rightarrow \mathbb{C}$ lisse et $G$-invariante qui vérifie de plus :
\begin{itemize}
\item $\varphi$ est $K$-finie à droite,
\item $\varphi$ est $Z(U(\mathfrak{g}))$-finie,
\item \begin{equation}
\varphi(zg) = \omega(z)\varphi(g) \quad \forall g \in G_\mathbb{A}, z \in \mathbb{A}^\times,
\end{equation}
\item $\varphi$ est à croissance modérée.
\end{itemize}

On note $\mathcal{A}(G_\mathbb{A}, \omega)$ l'espace des formes automorphes de caractère central $\omega$.
\end{definition}

On rajoute aussi une condition d'annulation dont on aura besoin pour la preuve de l'équation fonctionnelle. Ce qui donne la
\begin{definition}
Une forme cuspidale $\varphi$ de caractère central $\omega$ est une forme automorphe de caractère central $\omega$ qui vérifie de plus les conditions :
\begin{equation}
\int_{U \backslash U_\mathbb{A}} \varphi(ug) du = 0
\end{equation}
pour tout radical unipotent $U$ d'un sous-groupe parabolique propre de $G_\mathbb{A}$ et tout $g \in G_\mathbb{A}$.

On note $\mathcal{A}_0(G_\mathbb{A}, \omega)$ l'espace des formes cuspidales de caractère central $\omega$.
\end{definition}

Dans le but de donner un sens à la proposition \ref{decrap}, on a besoin d'introduire une notion de "norme" sur $\mathbb{R}_{> 0}G\backslash G_\mathbb{A}$. Commençons par la
\begin{proposition}[approximation forte \cite{goldfeld-hundley}]
\label{approx}
Pour $g \in G_\mathbb{A}$, il existe $\gamma \in G$, $t \in \mathbb{R}_{>0}$, $g_\infty \in SL_n(\mathbb{R})$ et $k \in \prod_p GL_n(\mathbb{Z}_p)$ tels que $g = \gamma t g_\infty k$. Cette décomposition n'est pas unique. Autrement dit,
\begin{equation}
G_\mathbb{A} = \mathbb{R}_{> 0}G(\mathbb{Q}) SL_n(\mathbb{R}) \prod_p GL_n(\mathbb{Z}_p).
\end{equation}
\end{proposition}

Une classe de $\mathbb{R}_{> 0}G\backslash G_\mathbb{A}$ est représentée par un élément de la forme $g_\infty k$ avec $g_\infty \in SL_n(\mathbb{R})$ et $k \in \prod_p GL_n(\mathbb{Z}_p)$. L'élément $g_\infty$ est alors unique modulo $SL_n(\mathbb{Z})$. On pose alors $||g_\infty||_\infty = \inf_{\gamma \in SL_n(\mathbb{Z})} |||\gamma g|||$, où $|||.|||$ est la norme d'opérateur sur $M_n(\mathbb{R})$.

Soit $g \in G_\mathbb{A}$, d'après la proposition \ref{approx}, il s'écrit sous la forme $g=\gamma t g_\infty k$ avec $\gamma \in G$, $t \in \mathbb{R}_{>0}$, $g_\infty \in SL_n(\mathbb{R})$ et $k \in \prod_p GL_n(\mathbb{Z}_p)$. On pose alors $||g|| = ||g_\infty||_\infty$.

\begin{proposition}
\label{decrap}
Les formes cuspidales sont à décroissance rapide. Soit $\varphi$ une forme cuspidale. Pour tout $d \in \mathbb{N}$, il existe $C > 0$ tel que
\begin{equation}
|\varphi(g)| \leq C||g||^{-d},
\end{equation}
pour tout $g \in G_{\mathbb{A}}$.
\end{proposition}

L'espace de Schwartz de $M_n(\mathbb{A})$ est, par définition, $\mathcal{S}(M_n(\mathbb{A}))=\otimes_v^{'} \mathcal{S}(M_n(\mathbb{Q}_v)=Vect(\left\lbrace \phi = \otimes \phi_v, \phi_v \in \mathcal{S}(M_n(\mathbb{Q}_v)), \phi_v = \mathbbm{1}_{\mathbb{Z}_v} \text{ sauf pour un nombre fini de v} \right\rbrace)$.

\begin{definition}
Soient $\varphi \in \mathcal{A}_0(G_\mathbb{A}, \omega)$, $\phi \in \mathcal{S}(M_\mathbb{A})$ et $s \in \mathbb{C}$, on pose
\begin{equation}
\zeta(\varphi, \phi, s) = \int_{G_\mathbb{A}} \phi(g) \varphi(g) |\det g|_\mathbb{A}^s dg,
\end{equation}
où $dg = \otimes_v dg_v$ est une mesure de Haar sur $GL_n(\mathbb{A})$ et $|.|_\mathbb{A} = \prod_v |.|_v$ est la valeur absolue adélique.
\end{definition}

Notons $G^0_\mathbb{A}=\left\lbrace g \in G_\mathbb{A}, |\det g|_\mathbb{A} = 1 \right\rbrace$. Comme $\mathbb{R}_{> 0} \subset \mathbb{A}^\times=Z(G_\mathbb{A})$, l'application $|\det|_\mathbb{A} : G_\mathbb{A} \rightarrow \mathbb{R}_{> 0}$ est surjective de noyau $G^0_\mathbb{A}$.

La factorisation $G_\mathbb{A} = \mathbb{R}_{> 0}G^0_\mathbb{A}$ permet d'obtenir que
\begin{align}
\zeta(\varphi, \phi, s) &= \int_0^\infty \int_{G^0_\mathbb{A}} \phi(tg) \omega(t) \varphi(g) t^{ns} dg \frac{dt}{t} \\
&= \int_0^\infty \int_{G \backslash G^0_\mathbb{A}} \sum_{x \in G}{\phi(txg)} \varphi(g) \omega(t) t^{ns} dg \frac{dt}{t}.
\end{align}

\begin{lemme}
\label{lemme_convergence_globale}
L'intégrale définissant $\zeta(\varphi, \phi, s)$ converge absolument pour $Re(s)$ assez grand.
\end{lemme}

\begin{proof}
D'après la proposition \ref{approx}, on a $G^0_\mathbb{A} = GL_n(\mathbb{Q})SL_n(\mathbb{R})\prod_p GL_n(\mathbb{Z}_p)$, d'où $G \backslash G^0_\mathbb{A} = SL_n(\mathbb{Z}) \backslash SL_n(\mathbb{R}) \prod_p GL_n(\mathbb{Z}_p)$. On veut donc majorer
\begin{equation}
\label{convergence_globale}
\int_0^\infty \int_{SL_n(\mathbb{Z}) \backslash SL_n(\mathbb{R})} \int_{\prod_p GL_n(\mathbb{Z}_p)} \sum_{x \in G}{|\phi(txg_\infty k)|} |\varphi(g_\infty k)| |t|^{nRe(s)-1} dk dg_\infty dt.
\end{equation}

Comme $\phi$ et $\varphi$ sont $\prod_p GL_n(\mathbb{Z}_p)$-finis, l'intégrale sur $\prod_p GL_n(\mathbb{Z}_p)$ est une somme finie. De plus, on peut supposer que $\phi = \prod_v \phi_v$ avec $\phi_v \in \mathcal{S}(M_n(\mathbb{Q}_v)$ et $\phi_v = \mathbbm{1}_{\mathbb{Z}_v}$ sauf pour un nombre fini de $v$. Comme $\phi_v$ est à support compact et qu'il n'y a qu'un nombre fini de $v$ tel que $\phi_v \neq \mathbbm{1}_{\mathbb{Z}_v}$, il existe $N \in \mathbb{N}^*$ (uniforme en $v$) tel que $supp(\phi_v) \subset N^{-1}M_n(\mathbb{Z}_v)$. Alors
\begin{align}
\sum_{x \in N^{-1}M_n(\mathbb{Z}) \cap G} |\phi(txg_\infty)| &\leq \sum_{x \in N^{-1}M_n(\mathbb{Z}) \cap G} |\phi_\infty(txg_\infty)|\prod_p |\phi_p(x)| \\
&\leq C_d \sum_{x \in N^{-1}M_n(\mathbb{Z}) \cap G} |||txg_\infty|||^{-d} \\
&\leq C_d |t|^{-d}|||g_\infty|||^{-d}\sum_{x \in N^{-1}M_n(\mathbb{Z}) \cap G} |||x|||^{-d},
\end{align}
pour tout $d > 0$, où $C_d > 0$ est une constante. Cette dernière somme est convergente pour $d$ assez grand. De plus, la somme est invariante par la translation $x \mapsto x\gamma$ pour $\gamma \in SL_n(\mathbb{Z})$. On peut donc remplacer $g_\infty$ par $\gamma g_\infty$. On en déduit que
\begin{equation}
\sum_{x \in N^{-1}M_n(\mathbb{Z}) \cup G} |\phi(txg_\infty)| \leq C_d' ||g_\infty ||^{-d}_{\infty} |t|^{-d}.
\end{equation}

De plus, d'après la proposition \ref{decrap}, pour tout $d > 0$, il existe $C_d'' > 0$ tel que $|\phi(g_\infty)| \leq C_d'' ||g_\infty||^{-d}_\infty$. Ce qui montre que l'intégrale \ref{convergence_globale} est majorée par
\begin{equation}
C_d'C_d''\int_0^\infty \int_{SL_n(\mathbb{Z}) \backslash SL_n(\mathbb{R})} ||g_\infty||^{-2d}_\infty|t|^{nRe(s)-1-d} dg_\infty dt.
\end{equation}

Pour finir, il ne reste plus qu'à montrer que l'intégrale sur $SL_n(\mathbb{Z}) \backslash SL_n(\mathbb{R})$ est convergente pour $d$ assez grand. Pour ce faire on remarque que
\begin{equation}
||g_\infty||^{-2d}_\infty = \sup_{\gamma \in SL_n(\mathbb{Z})} |||\gamma g_\infty|||^{-2d} \leq \sum_{\gamma \in SL_n(\mathbb{Z})} |||\gamma g_\infty|||^{-2d}.
\end{equation}
L'intégrale sur $SL_n(\mathbb{Z}) \backslash SL_n(\mathbb{R})$ est donc majorée par
\begin{equation}
\int_{SL_n(\mathbb{R})} |||g_\infty|||^{-2d} dg_\infty,
\end{equation}
qui converge pour $d$ assez grand. Ce qui permet de conclure. On fixe $d > 0$ assez grand pour avoir la convergence des différentes sommes et intégrales. L'intégrale \ref{convergence_globale} est alors majorée par
\begin{equation}
C_d''' \int_0^\infty |t|^{nRe(s)-1-d} dt,
\end{equation}
qui converge pour $Re(s)$ assez grand.
\end{proof}

Comme dans la preuve de l'équation fonctionnelle de la fonction zêta de Riemann, on scinde l'intégrale en $1$ dans le but de faire apparaître une symétrie. Autrement dit,
\begin{equation}
\begin{split}
\zeta(\varphi, \phi, s) &= \int_0^1 \int_{G \backslash G^0_\mathbb{A}} \sum_{x \in G}{\phi(txg)} \varphi(g) \omega(t) t^{ns} dg \frac{dt}{t} \\
&+ \int_1^\infty \int_{G \backslash G^0_\mathbb{A}} \sum_{x \in G}{\phi(txg)} \varphi(g) \omega(t) t^{ns} dg \frac{dt}{t}.
\end{split}
\end{equation}

La seconde intégrale converge absolument pour tout $s \in \mathbb{C}$, c'est une fonction entière. En effet, on reprend le raisonnement du lemme \ref{lemme_convergence_globale} et on montre la convergence absolue de $\int_1^\infty$, cette intégrale est majorée par
\begin{equation}
C_d''' \int_1^\infty |t|^{nRe(s)-1-d} dt,
\end{equation}
pour tout $d > 0$ assez grand. On fixe maintenant $s \in \mathbb{C}$, cette dernière intégrale est alors convergente pour $d$ assez grand.

Pour la première intégrale, on fait le changement de variable $t \mapsto t^{-1}$, ce qui donne
\begin{equation}
\int_1^\infty \int_{G \backslash G^0_\mathbb{A}} \sum_{x \in G}{\phi(t^{-1}xg)} \varphi(g) \omega^{-1}(t) t^{-ns} dg \frac{dt}{t}.
\end{equation}

On va maintenant utiliser la formule de Poisson sur $M_n(\mathbb{A})$. Commençons par définir la transformée de Fourier adélique, on pose
\begin{equation}
\hat{\phi}(y) = \int_{M_n(\mathbb{A})} \phi(x)\psi(Tr(xy))dx, y \in M_n(\mathbb{A}),
\end{equation}
où $dx$ est une mesure de Haar normalisée pour avoir la formule $\hat{\hat{\phi}}(x)=\phi(-x)$ et $\psi : \mathbb{A}/\mathbb{Q} \rightarrow \mathbb{C}^\times$ est un caractère non trivial.

La formule de Poisson donne, pour la fonction $x \mapsto \phi(t^{-1}xg)$ :
\begin{equation}
\sum_{x \in M_n(\mathbb{Q})} \phi(t^{-1}xg) = t^{n^2}\sum_{x \in M_n(\mathbb{Q})} \hat{\phi}(txg^{-1}),
\end{equation}
on se rappelle que $g \in G^0_\mathbb{A}$, donc $|\det g|_\mathbb{A}=1$. On scinde la somme selon le rang de la matrice et on obtient :
\begin{equation}
\begin{split}
\sum_{x \in G} \phi(t^{-1}xg) &= t^{n^2}\sum_{x \in G} \hat{\phi}(txg^{-1}) \\
&+ \sum_{r < n, rg(x)=r} \left( t^{n^2}\hat{\phi}(txg^{-1}) - \phi(t^{-1}xg)\right).
\end{split}
\end{equation}

La contribution de la dernière somme s'avèrera nulle. Ce qui nous permet d'en déduire la
\begin{proposition}
Pour $\varphi \in \mathcal{A}_0(G_\mathbb{A}, \omega)$ et $\phi \in \mathcal{S}(M_n(\mathbb{A})$, la fonction $\zeta(\varphi, \phi, .)$ peut être prolongée en une fonction entière et vérifie l'équation fonctionnelle
\begin{equation}
\label{eqcusp}
\zeta(\varphi, \phi, s) = \zeta(\check{\varphi}, \hat{\phi}, n-s),
\end{equation}
où $\check{\varphi}(g)=\varphi(g^{-1})$.
\end{proposition}

\begin{proof}
Il suffit de prouver que la contribution dans la formule de Poisson des matrices de rang $r < n$ est effectivement nulle. On considère l'action de $G$ par translation à droite sur l'ensemble des matrices de rang $r$. Chaque orbite contient un représentant de la forme $\begin{pmatrix} 
* & 0 \\
* & 0 
\end{pmatrix}$, on note $X$ l'ensemble des matrices de cette forme. On pose $P$ le sous-groupe parabolique de $G$ des matrices de la forme $\begin{pmatrix} 
* & 0 \\
* & * 
\end{pmatrix}$ et $U$ son radical unipotent.

On réécrit la somme sur les matrices de rang $r$ grâce au système de représentant $X$,
\begin{equation}
\sum_{rg(x)=r}\phi(xg) = \sum_{\gamma \in P \backslash G} \sum_{x \in X} \phi(x \gamma g).
\end{equation}
On en déduit que la contribution des matrices de rang $r$ dans la seconde intégrale est
\begin{equation}
\int_{P \backslash G^0_\mathbb{A}} \sum_{x \in X}{\phi(t^{-1}x g)} \varphi(g) dg.
\end{equation}
De plus, on remarque que, $xu=x$, pour tout $x \in X$ et $u \in U_\mathbb{A}$. Ce qui nous permet de réécrire cette intégrale sous la forme
\begin{equation}
\int_{PU_\mathbb{A} \backslash G^0_\mathbb{A}} \sum_{x \in X}{\phi(t^{-1}xg)} \int_{U \backslash U_\mathbb{A}} \varphi(ug) du dg.
\end{equation}
Cette dernière intégrale s'annule, car $f$ est cuspidale. On montre de même de l'intégrale correspondant au terme en $\hat{\phi}$ sur les matrices de rang $r < n$ s'annule aussi. Ce qui nous donne, grâce à la formule de Poisson et le raisonnement précédent, la formule
\begin{equation}
\begin{split}
\zeta(\varphi, \phi, s) &= \int_1^\infty \int_{G \backslash G^0_\mathbb{A}} \sum_{x \in G}{\hat{\phi}(txg^{-1})} \varphi(g) \omega^{-1}(t) t^{n(n-s)} dg \frac{dt}{t} \\
&+ \int_1^\infty \int_{G \backslash G^0_\mathbb{A}} \sum_{x \in G}{\phi(txg)} \varphi(g) \omega(t) t^{ns} dg \frac{dt}{t},
\end{split}
\end{equation}
ce qui démontre l'équation fonctionnelle en effectuant le changement de variable $g \mapsto g^{-1}$ dans la première intégrale.
\end{proof}

\subsection{Représentations automorphes}

L'espace des formes cuspidales $\mathcal{A}_0(G_\mathbb{A}, \omega)$ est stable par l'action de $U(\mathfrak{g})$ par opérateurs différentiels et par translation à droite de $O_n(\mathbb{R})$ et $GL_n(\mathbb{A}_f)$, c'est un $(\mathfrak{g}, O_n(\mathbb{R})) \times GL_n(\mathbb{A}_f)$-module.

\begin{proposition}
\label{unitaire}
L'espace des formes cuspidales $\mathcal{A}_0(G_\mathbb{A}, \omega)$ est muni d'une forme hermitienne définie positive, définie par
\begin{equation}
(\varphi_1, \varphi_2) = \int_{\mathbb{A}^\times G \backslash G_\mathbb{A}} \varphi_1(g)\overline{\varphi_2(g)}dg,
\end{equation}
pour tous $\varphi_1, \varphi_2 \in \mathcal{A}_0(G_\mathbb{A}, \omega)$.

Cette forme hermitienne fait de l'espace $\mathcal{A}_0(G_\mathbb{A}, \omega)$ un $(\mathfrak{g}, O_n(\mathbb{R})) \times GL_n(\mathbb{A}_f)$-module unitaire.
\end{proposition}

Un coefficient $f$ de $\mathcal{A}_0(G_\mathbb{A}, \omega)$ est de la forme
\begin{equation}
f(g) = <\pi(g)\varphi, \tilde{\varphi}> = \int_{\mathbb{A}^\times G \backslash G_\mathbb{A}} \varphi(hg) \tilde{\varphi}(h) dh,
\end{equation}
où $\varphi \in \mathcal{A}_0(G_\mathbb{A}, \omega)$ et $\tilde{\varphi} \in \mathcal{A}_0(G_\mathbb{A}, \omega^{-1})$.

\begin{definition}
Soient $f$ un coefficient de $\mathcal{A}_0(G_\mathbb{A}, \omega)$, $\phi \in \mathcal{S}(M_\mathbb{A})$ et $s \in \mathbb{C}$, on pose
\begin{equation}
\zeta(f, \phi, s) = \int_{G_\mathbb{A}} \phi(g) f(g) |\det g|_\mathbb{A}^s dg.
\end{equation}
\end{definition}

On peut déduire les propriétés de cette fonction zêta grâce à ce que l'on vient de faire pour les formes cuspidales. Plus précisément, on a
\begin{align}
\zeta(f, \phi, s) &= \int_{G_\mathbb{A}}\phi(g)\int_{\mathbb{A}^\times G \backslash G_\mathbb{A}} \varphi(hg) \tilde{\varphi}(h) dh |\det g|_\mathbb{A}^s dg \\
&= \int_{\mathbb{A}^\times G \backslash G_\mathbb{A}} \tilde{\varphi}(h) \int_{G_\mathbb{A}}\phi(h^{-1}g)\varphi(g)|\det g|_\mathbb{A}^s dg |\det h|_\mathbb{A}^{-s} dh \\
&= \int_{\mathbb{A}^\times G \backslash G_\mathbb{A}} \tilde{\varphi}(h) \zeta(\varphi, \phi(h^{-1}.), s)|\det h|_\mathbb{A}^{-s} dh,
\end{align}
où la deuxième égalité s'obtient grâce au changement de variable $g \mapsto h^{-1}g$. Ceci nous permet de démontrer la
\begin{proposition}
\label{zetacusp}
Pour $f$ un coefficient de $\mathcal{A}_0(G_\mathbb{A}, \omega)$ et $\phi \in \mathcal{S}(M_\mathbb{A})$, la fonction $\zeta(f, \phi, .)$ peut être prolongée en une fonction entière et vérifie l'équation fonctionnelle
\begin{equation}
\zeta(f, \phi, s) = \zeta(\check{f}, \hat{\phi}, n-s),
\end{equation}
où $\check{f}(g) = f(g^{-1})$.
\end{proposition}

\begin{proof}
On utilise l'équation fonctionnelle (\ref{eqcusp}) et le fait que la transformée de Fourier de $\phi(h^{-1}.)$ est $|\det h|_\mathbb{A}^n\hat{\phi}(.h)$,
\begin{align}
\zeta(f, \phi, s) &= \int_{\mathbb{A}^\times G \backslash G_\mathbb{A}} \tilde{\varphi}(h) \zeta(\check{\varphi}, \hat{\phi}(.h), n-s)|\det h|_\mathbb{A}^{n-s} dh \\
&= \int_{\mathbb{A}^\times G \backslash G_\mathbb{A}} \tilde{\varphi}(h) \int_{G_\mathbb{A}}\hat{\phi}(gh)\varphi(g^{-1})|\det g|_\mathbb{A}^{n-s} dg |\det h|_\mathbb{A}^{n-s} dh.
\end{align}
On effectue maintenant le changement de variable $g \mapsto gh^{-1}$, ce qui donne
\begin{equation}
\int_{\mathbb{A}^\times G \backslash G_\mathbb{A}} \tilde{\varphi}(h) \int_{G_\mathbb{A}}\hat{\phi}(g)\varphi(hg^{-1})|\det g|_\mathbb{A}^{n-s} dg dh,
\end{equation}
qui est bien $\zeta(\check{f}, \hat{\phi}, n-s)$.
\end{proof}

Si l'on combine cette proposition avec les résultats locaux, on peut construire la fonction $L$ attachée à une représentation cuspidale irréductible.
\begin{definition}
Une représentation cuspidale est un $(\mathfrak{g}, O_n(\mathbb{R})) \times GL_n(\mathbb{A}_f)$-module qui est isomorphe à un sous-quotient de $\mathcal{A}_0(G_\mathbb{A}, \omega)$.
\end{definition}

Plus précisément, on montre le
\begin{theoreme}[Godement-Jacquet \cite{godement-jacquet}]
Soit $\pi$ une représentation cuspidale irréductible.

Le produit $L(s, \pi) = \prod_v L(s, \pi_v)$, qui est défini pour $Re(s) > n$, se prolonge en une fonction entière. De plus, $L(s, \pi)$ vérifie l'équation fonctionnelle
\begin{equation}
L(s,\pi) = \epsilon(s,\pi)L(1-s,\tilde{\pi}),
\end{equation}
où $\epsilon(s,\pi) = \prod_v \epsilon(s, \pi_v, \psi_v)$, avec $\psi$ un caractère non trivial de $\mathbb{A}/\mathbb{Q}$.
\end{theoreme}

\begin{proof}
La représentation $\pi$ se décompose en facteurs locaux,
$\pi \simeq \otimes_v^{'} \pi_v$, où $\pi_v$ est une représentation admissible irréductible de $GL_n(\mathbb{Q}_v)$ (un $(\mathfrak{g}, O_n(\mathbb{R}))$-module irréductible pour la place archimédienne) et pour presque toutes les places $\pi_v$ est sphérique (contient la représentation unité de $GL_n(\mathbb{Z}_v)$).

D'après le théorème \ref{thm_padique}, pour chaque place $v$, il existe un nombre fini $(\phi_{\alpha_v})_{\alpha_v \in I_v}$ d'éléments de $\mathcal{S}(M_v)$ et de coefficient $(f_{\alpha_v})_{\alpha_v \in I_v}$ de $\pi_v$ tels que
\begin{equation}
\sum_{\alpha_v \in I_v} \zeta(f_{\alpha_v}, \phi_{\alpha_v}, s + \frac{1}{2}(n-1)) = L(s, \pi_v).
\end{equation}
De plus, d'après l'équation fonctionnelle locale
\begin{equation}
\sum_{\alpha_v \in I_v} \zeta(\check{f}_{\alpha_v}, \hat{\phi}_{\alpha_v}, 1-s + \frac{1}{2}(n-1)) = \epsilon(s,\pi_v,\psi_v)L(1-s, \tilde{\pi}_v).
\end{equation}

Notons $I = \prod_v I_v$. Pour presque toutes les places $v$, $\pi_v$ est sphérique, $I_v$ est un singleton (lemme \ref{lemmespherique}); donc $I$ est fini.

Pour $\alpha = (\alpha_v) \in I$, on pose
\begin{equation}
\phi_\alpha = \prod_v \phi_{\alpha_v}, \quad f_\alpha = \prod_v f_{\alpha_v}.
\end{equation}
Alors $\phi_\alpha \in \mathcal{S}(M_\mathbb{A})$ et $f_\alpha$ est un coefficient de $\pi$ qui est un sous-quotient de $\mathcal{A}_0(G_\mathbb{A}, \omega)$. De plus,
\begin{equation}
\zeta(f_\alpha, \phi_\alpha, s) = \prod_v \zeta(f_{\alpha_v}, \phi_{\alpha_v}, s).
\end{equation}

On en déduit que
\begin{align}
L(s, \pi) &= \prod_v L(s, \pi_v) = \prod_v \sum_{\alpha_v \in I_v} \zeta(f_{\alpha_v}, \phi_{\alpha_v}, s + \frac{1}{2}(n-1)) \\
&= \sum_{\alpha \in I} \zeta(f_\alpha, \phi_\alpha, s + \frac{1}{2}(n-1))
\end{align}
est une somme finie de fonctions zêta, qui se prolongent en des fonctions entières (proposition \ref{zetacusp}). De plus,
\begin{align}
L(s, \pi) &= \sum_{\alpha \in I} \zeta(f_\alpha, \phi_\alpha, s + \frac{1}{2}(n-1)) \\
&= \sum_{\alpha \in I} \zeta(\check{f}_\alpha, \hat{\phi}_\alpha, 1 - s + \frac{1}{2}(n-1)) \\
&= \prod_v \sum_{\alpha_v \in I_v} \zeta(\check{f}_{\alpha_v}, \hat{\phi}_{\alpha_v}, 1-s + \frac{1}{2}(n-1)) \\
&= \prod_v \epsilon(s, \pi_v, \psi_v) L(1-s, \tilde{\pi}_v) \\
&= \epsilon(s, \pi)L(1-s, \tilde{\pi}).
\end{align}
\end{proof}